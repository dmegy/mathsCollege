

% - - - - - - - - - - - - - -
% PARAMETRAGE DU PACKAGE ANSWERS 
% POUR LES INDICATIONS ET CORRECTIONS
% - - - - - - - - - - - - - - 

\usepackage{answers}



\Newassociation{sol}{Soln}{solutions}
% ira dans le fichier d'identifiant 'solutions'
% et écrira les solutions dans un environnement 'Soln'
\Newassociation{hint}{Hint}{indications}

%\newenvironment{exo}{
%\begin{ex} \label{enonce.\theex}
%
%}
%{~\hyperref[indication.\theex]{[Indications]}
%~\hyperref[solution.\theex]{[Correction]}~\end{ex} }


\usepackage{xparse}
\NewDocumentEnvironment{exo}{o}
 {\label{enonce.\theex} \IfNoValueTF{#1}
   {\ex\addcontentsline{toc}{subsubsection}{\protect\numberline{\theex}  }}
   {\ex[#1]\addcontentsline{toc}{subsubsection}{\protect\numberline{\theex}#1}}%
   \ignorespaces}
 {~\hyperref[indication.\theex]{[Indications]}
~\hyperref[solution.\theex]{[Correction]}~\endex\medskip\hrule}
 
 


\renewenvironment{Hint}[1]{
\noindent \textbf{Indications pour l'exercice #1} (retour à l'\hyperref[enonce.#1]{énoncé}, voir la \hyperref[solution.#1]{correction}) 
\phantomsection\label{indication.#1}
}

\renewenvironment{Soln}[1]{
\noindent \textbf{Correction de l'exercice #1} (retour à l'\hyperref[enonce.#1]{énoncé}, retour à l'\hyperref[indication.#1]{indication}) 
\phantomsection\label{solution.#1}
}




% - - - - - - - - - - - - - - 
% FIN PARAMETRAGE ANSWERS
% - - - - - - - - - - - - - - 

%-----------------------------
% MACROS POUR LES FEUILLES DE TD



\newcommand{\debut}{
	\Opensolutionfile{indications}[\jobname_hints]
	\Opensolutionfile{solutions}[\jobname_sol]
	
}
\newcommand{\fin}{
	\Closesolutionfile{indications}
	\Closesolutionfile{solutions}
	\section{Indications pour la résolution des exercices}
	\Readsolutionfile{indications}
	\newpage
	\section{Correction des exercices}
	\Readsolutionfile{solutions}
}
