\begin{Soln}{1}
$21$ et $33$.
\end{Soln}
\begin{Soln}{2}
\begin{enumerate}
\item La suite semble être construite en itérant alternativement les opérations $+2$ et $\times 2$. Dans ce cas, le prochain nombre devrait être $36$.
\item La suite semble être construite en itérant le groupe d'opérations $+1$ et $\times 2$ à chaque étape (ou $\times 2$ et $+2$, c'est pareil). Dans ce cas, le prochain nombre devrait être $126$.
\item
Chaque terme semble être la somme des deux précédents. Dans ce cas, le prochain nombre devrait être $21$.
\end{enumerate}
\end{Soln}
\begin{Soln}{3}
On fait la liste des diviseurs de $36$, et on calcule les sommes possibles de trois nombres dont le produit vaut trente-six.

S'il manque des données, c'est que cette somme à elle seule, qui est connue en théorie, ne permet pas de conclure. Ceci ne se produit que si les âges sont $1$, $6$ et $6$ ou bien $2$, $2$ et $9$.

Comme il y a un(e) ainé(e), on est forcément dans le deuxième cas.
\end{Soln}
\begin{Soln}{5}
Il y a $90$ matchs joués : chaque équipe joue $18$ matchs. Il y a dix équipes, mais multiplier le nombre de maths par équipes par le nombre d'équipes revient à compter chaque match deux fois.
\end{Soln}
\begin{Soln}{6}
Entre chacun des dix chiffres, c'est-à-dire à neuf emplacements possibles, on doit choisir si oui ou non on insère un signe $+$ ou pas. Pour chaque emplacement, on a donc deux choix, et donc au final, on a $2\times 2\times 2\times2\times2\times2\times2\times2\times 2 =512 (=2^9)$ choix. Mais parmi ces $512$ choix, il y a celui de ne rien insérer du tout, or l'énoncé est formulé de telle sorte qu'on doit au moins insérer un signe $+$. Il y a donc finalement $511$ choix.
\end{Soln}
\begin{Soln}{7}
Vingt-quatre heures avant qu'il ne soit totalement recouvert, c'est-à-dire le 30 juin à midi.
\end{Soln}
\begin{Soln}{8}
Notons $n$ le nombre de participants\footnote{Méthode : nommer les objets permet de raisonner ou de faire des calculs dessus plus clairement.}. Pour chaque entier $k$ compris entre $1$ et $n$, notons $a_n$ le nombre personnes que connaît le $n$-ème participant. Alors les $a_i$ sont $n$ entiers entre $0$ et $n-1$ et il s'agit de montrer que deux d'entre eux sont identiques. (Notons que le principe des tiroirs ne permet pas de conclure immédiatement, puisqu'il y a $n$ entiers à distribuer dans $n$ tiroirs.)

Supposons que tous les $a_i$ soient distincts. Alors ils valent forcément (dans le désordre) $0$, $1$, $2$, .. $n-1$. Ceci signifie qu'un participant connaît tout le monde, et qu'un autre ne connaît personne, ce qui est absurde. On en déduit que deux des entiers $a_i$ sont égaux.
\end{Soln}
\begin{Soln}{9}
Groupons les entiers de $1$ à $20$ deux par deux : $1$ et $2$, puis $3$ et $4$, etc jusqu'à $19$ et $20$. Ceci donne dix \og lots\fg{} de deux entiers consécutifs.

Or on a choisi onze nombres entiers, donc comme il n'y a que dix lots, on a forcément choisi deux nombres dans un même lot au moins une fois, sinon on aurait choisi moins de dix nombres.
\end{Soln}
\begin{Soln}{10}
Comme chaque coefficient est compris entre $-1$ et $1$, la somme de trois coefficients est comprise entre $-3$ et $3$, ce qui fait sept valeurs entières possibles.

D'autre part il y a huit sommes à calculer (les trois lignes, les trois colonnes et les deux diagonales).

Il y a donc au moins deux des huit sommes qui sont identiques. (C'est le principe des tiroirs, mais ici le résultat est intuitif.)
\end{Soln}
\begin{Soln}{11}
\begin{enumerate}
\item
Comme il y a plus d'élèves que de jours dans l'année, il y a au moins deux élèves qui fêtent leur anniversaire le même jour.

\item À partir de $367$ élèves, on peut conclure de la même manière.

\item S'il y a $3\times 366=1098$ élèves, il est possible qu'exactement trois d'entre eux fêtent leur anniversaire chaque jour. À partir de $3\times 366+1=1099$ élèves, il y en a forcément quatre qui fêtent leur anniversaire le même jour.
\end{enumerate}
\end{Soln}
\begin{Soln}{12}
On trace une corde et sa médiatrice, qui doit contenir le centre du cercle.

Ensuite, soit on recommence avec une autre corde, soit, puisque la première médiatrice fournit un diamètre, on construit la médiatrice de ce diamètre.
\end{Soln}
\begin{Soln}{13}
Comme $BA=BC$, $B$ est sur la médiatrice de $[AC]$.

De même, $D$ est sur la médiatrice de $[AC]$.

Donc la droite $(BD)$ est la médiatrice de $[AC]$, ce dont on déduit que ces deux droites se croisent à angle droit au milieu de $[AC]$.
\end{Soln}
\begin{Soln}{14}
Comme dans le cas du cerf-volant, les égalités $BA=BC$ et $DA=DC$ montrent que $B$ et $D$ sont sur la médiatrice de $[AC]$ et donc que la droite $(BD)$ est la médiatrice de $[AC]$.

On montre de la même manière que la droite $(AC)$ est la médiatrice du segment $[DB]$.
\end{Soln}
\begin{Soln}{15}
L'exercice ressemble comme deux gouttes d'eau à la preuve que les médiatrices sont concourantes, mais rédigé avec des réflexions.

Comme $B$ est le symétrique de $A$ par rapport à $\mathcal D$, la droite $\mathcal D$ est la médiatrice de $[AB]$. On en déduit que pour tout point $P$ de $\mathcal D$, on a $PA=PB$. En particulier, comme $O$ est sur $\mathcal D$, on a
\[ OA=OB.\]

On prouve de la même manière que
\[OA=OB'.\]
On en déduit donc  l'égalité de distances
\[OB=OB',\]
ce qui montre que \fbox{$O$ est sur la médiatrice de $[BB']$.}
\end{Soln}
\begin{Soln}{16}
Les cercles ont même rayon, donc les cordes communes sont les médiatrices des segments reliant les centres des cercles. Ces médiatrices sont donc concourantes.
\end{Soln}
\begin{Soln}{17}
Il s'agit de montrer que $C'$ appartient au cercle circonscrit de $ABC$.

Comme $C$ et $C'$ sont symétriques par rapport à $\mathcal D$, tout point $P$ de cette droite vérifie $PC = PC'$.

En particulier, le centre $O$, du cercle circonscrit à $ABC$, qui est l'intersection des trois médiatrices, vérifie $OC=OC'$.

On en déduit que $C'$ est sur le cercle de centre $O$ et de rayon $OC$. Ce cercle est le cercle circonscrit à $ABC$, ce qu'il fallait démontrer.

\begin{center}
\definecolor{uuuuuu}{rgb}{0.26666666666666666,0.26666666666666666,0.26666666666666666}
\definecolor{qqwuqq}{rgb}{0.,0.39215686274509803,0.}
\definecolor{qqqqff}{rgb}{0.,0.,1.}
\begin{tikzpicture}[line cap=round,line join=round,>=triangle 45,x=1.0cm,y=1.0cm]
\clip(-2.72,-1.62) rectangle (5.42,4.92);
\draw[color=qqwuqq,fill=qqwuqq,fill opacity=0.10000000149011612] (2.2194272122991046,3.40417457639394) -- (2.285994396756149,3.67907239368877) -- (2.0110965794613187,3.7456395781458145) -- (1.9445293950042741,3.4707417608509843) -- cycle;
\draw[color=qqwuqq,fill=qqwuqq,fill opacity=0.10000000149011612] (1.5048978172948304,0.4534328155429559) -- (1.5714650017518745,0.7283306328377863) -- (1.296567184457044,0.7948978172948304) -- (1.23,0.52) -- cycle;
\draw [line width=1.6pt,dash pattern=on 2pt off 2pt,domain=-2.72:5.42] plot(\x,{(-4.9242--4.46*\x)/1.08});
\draw (2.98,3.22)-- (3.46,-0.02);
\draw (2.98,3.22)-- (-1.,1.06);
\draw (0.9090587900085476,3.721483521701966)-- (1.9445293950042746,3.4707417608509847);
\draw (1.4479756408967428,3.683584591110625) -- (1.405612544116079,3.5086406914423254);
\draw (1.9445293950042746,3.4707417608509847)-- (2.98,3.22);
\draw (2.483446245892468,3.4328428302596423) -- (2.441083149111804,3.257898930591342);
\draw (-1.,1.06)-- (1.23,0.52);
\draw (0.10216467901038476,0.8857092186526129) -- (0.059801582229720825,0.7107653189843128);
\draw (0.1701984177702793,0.8692346810156878) -- (0.12783532098961536,0.6942907813473878);
\draw (1.23,0.52)-- (3.46,-0.02);
\draw (2.3321646790103854,0.34570921865261267) -- (2.28980158222972,0.17076531898431263);
\draw (2.4001984177702798,0.3292346810156876) -- (2.357835320989615,0.1542907813473875);
\draw(1.4272093023255812,1.3344013781223087) circle (2.442670897524113cm);
\draw (1.4272093023255812,1.3344013781223087)-- (2.98,3.22);
\draw (2.089631366800829,2.2803772362035155) -- (2.228580806236984,2.165952355690673);
\draw (2.1341299314447126,2.3344131293175754) -- (2.273079370880867,2.2199882488047327);
\draw (2.178628496088596,2.3884490224316353) -- (2.3175779355247506,2.2740241419187925);
\draw (1.4272093023255812,1.3344013781223087)-- (0.9090587900085476,3.721483521701966);
\draw (1.095030928902691,2.4404442542379905) -- (1.2709346031470836,2.4786266767295353);
\draw (1.0801822090448685,2.508851238666365) -- (1.256085883289261,2.5470336611579096);
\draw (1.0653334891870458,2.5772582230947396) -- (1.2412371634314383,2.6154406455862844);
\draw (1.4272093023255812,1.3344013781223087)-- (3.46,-0.02);
\draw (2.435253454598866,0.7709119792479728) -- (2.335447849058994,0.6211159809609037);
\draw (2.4935074539327253,0.7320986882046893) -- (2.3937018483928543,0.5823026899176201);
\draw (2.551761453266586,0.6932853971614057) -- (2.4519558477267154,0.5434893988743364);
\draw (1.4272093023255812,1.3344013781223087)-- (-1.,1.06);
\draw (0.29327186109722975,1.1156339332027652) -- (0.27305127054936706,1.2944945709012141);
\draw (0.2237149464367219,1.1077703702119297) -- (0.20349435588885922,1.2866310079103789);
\draw (0.15415803177621407,1.0999068072210945) -- (0.13393744122835138,1.2787674449195434);
\begin{scriptsize}
\draw [fill=qqqqff] (-1.,1.06) circle (2.5pt);
\draw[color=qqqqff] (-1.44,1.33) node {$A$};
\draw [fill=qqqqff] (3.46,-0.02) circle (2.5pt);
\draw[color=qqqqff] (3.6,0.35) node {$B$};
\draw [fill=qqqqff] (2.98,3.22) circle (2.5pt);
\draw[color=qqqqff] (3.12,3.59) node {$C$};
\draw [fill=qqqqff] (0.9090587900085476,3.721483521701966) circle (2.5pt);
\draw[color=qqqqff] (1.1,4.09) node {$C'$};
\draw [fill=uuuuuu] (1.4272093023255812,1.3344013781223087) circle (1.5pt);
\draw[color=uuuuuu] (1.02,1.65) node {$O$};
\end{scriptsize}
\end{tikzpicture}
\end{center}

Note : on peut aussi poser cet exercice de la façon suivante \og Montrer qu'un trapèze est inscriptible si et seulement s'il est isocèle\fg. En général, on montre ce résultat à l'aide du théorème de l'angle inscrit mais la preuve élémentaire ci-dessus marche aussi et n'est pas très longue.

\end{Soln}
\begin{Soln}{18}
Bel exercice qui demande d'utiliser deux caractérisations des triangles isocèles, ainsi que de la médiatrice.

\begin{center}
\definecolor{qqwuqq}{rgb}{0.,0.39215686274509803,0.}
\definecolor{ccqqqq}{rgb}{0.8,0.,0.}
\definecolor{uuuuuu}{rgb}{0.26666666666666666,0.26666666666666666,0.26666666666666666}
\definecolor{qqqqff}{rgb}{0.,0.,1.}
\begin{tikzpicture}[line cap=round,line join=round,>=triangle 45,x=1.0cm,y=1.0cm]
\clip(-2.72,-1.06) rectangle (4.4,4.5);
\draw [shift={(-2.2933658978078877,3.840182343311194)},color=qqwuqq,fill=qqwuqq,fill opacity=0.10000000149011612] (0,0) -- (-73.62011222157199:0.6) arc (-73.62011222157199:-30.60982553724811:0.6) -- cycle;
\draw [shift={(2.1,2.82)},color=qqwuqq,fill=qqwuqq,fill opacity=0.10000000149011612] (0,0) -- (-175.53608221615343:0.6) arc (-175.53608221615343:-132.52579553182954:0.6) -- cycle;
\draw [shift={(2.1,2.82)},color=qqwuqq,fill=qqwuqq,fill opacity=0.10000000149011612] (0,0) -- (-132.52579553182954:0.6) arc (-132.52579553182954:-89.51550884750567:0.6) -- cycle;
\draw [shift={(3.479331008575662,0.4248825897600912)},color=qqwuqq,fill=qqwuqq,fill opacity=0.10000000149011612] (0,0) -- (149.3901744627519:0.6) arc (149.3901744627519:192.40046114707576:0.6) -- cycle;
\draw [line width=1.6pt] (-1.,-0.56)-- (0.24,4.78);
\draw [line width=1.6pt] (-1.,-0.56)-- (3.48,2.02);
\draw (-2.2933658978078877,3.840182343311194)-- (3.479331008575662,0.4248825897600912);
\draw [domain=-2.72:4.4] plot(\x,{(--1.644--3.38*\x)/3.1});
\draw (-0.2578750441772161,2.635925212978763)-- (2.1,2.82);
\draw (2.1,2.82)-- (2.113419968703404,1.2329963212622284);
\draw [dash pattern=on 2pt off 2pt] (2.1,2.82)-- (-2.2933658978078877,3.840182343311194);
\draw [dash pattern=on 2pt off 2pt] (2.1,2.82)-- (3.479331008575662,0.4248825897600912);
\draw [line width=1.6pt,color=ccqqqq] (-1.,-0.56)-- (-2.2933658978078877,3.840182343311194);
\draw [line width=1.6pt,color=ccqqqq] (-1.7395281702117957,1.5658970689229936) -- (-1.5284573048877046,1.6279381022101624);
\draw [line width=1.6pt,color=ccqqqq] (-1.7649085929201827,1.6522442411010312) -- (-1.5538377275960917,1.7142852743882);
\draw [line width=1.6pt,color=ccqqqq] (-1.,-0.56)-- (3.479331008575662,0.4248825897600912);
\draw [line width=1.6pt,color=ccqqqq] (1.1720935788644282,0.030211611922458685) -- (1.2193370802114338,-0.18465590907705226);
\draw [line width=1.6pt,color=ccqqqq] (1.2599939283642287,0.049538498837143244) -- (1.3072374297112344,-0.1653290221623677);
\draw [shift={(-2.2933658978078877,3.840182343311194)},color=qqwuqq] (-73.62011222157199:0.6) arc (-73.62011222157199:-30.60982553724811:0.6);
\draw[color=qqwuqq] (-1.9617632237660787,3.4139902896917933) -- (-1.888073740645677,3.319280944443037);
\draw [shift={(2.1,2.82)},color=qqwuqq] (-175.53608221615343:0.6) arc (-175.53608221615343:-132.52579553182954:0.6);
\draw[color=qqwuqq] (1.6145234602374336,2.583541696402589) -- (1.506639784734641,2.5309954067142746);
\draw [shift={(2.1,2.82)},color=qqwuqq] (-132.52579553182954:0.6) arc (-132.52579553182954:-89.51550884750567:0.6);
\draw[color=qqwuqq] (1.9062996055774284,2.315936356002002) -- (1.8632550734835236,2.2039222128913356);
\draw [shift={(3.479331008575662,0.4248825897600912)},color=qqwuqq] (149.3901744627519:0.6) arc (149.3901744627519:192.40046114707576:0.6);
\draw[color=qqwuqq] (2.946134534012404,0.5103315190296419) -- (2.8276464285539014,0.5293201699784313);
\draw [line width=1.6pt,color=ccqqqq] (-1.,-0.56)-- (2.1,2.82);
\draw [line width=1.6pt,color=ccqqqq] (0.4385164656825278,1.1711876386320568) -- (0.6006505476666322,1.022484782374446);
\draw [line width=1.6pt,color=ccqqqq] (0.49934945233336825,1.237515217625554) -- (0.6614835343174726,1.0888123613679432);
\begin{scriptsize}
\draw [fill=qqqqff] (-1.,-0.56) circle (2.5pt);
\draw[color=qqqqff] (-1.5,-0.25) node {$O$};
\draw [fill=qqqqff] (2.1,2.82) circle (2.5pt);
\draw[color=qqqqff] (2.08,3.39) node {$A$};
\draw [fill=qqqqff] (-2.2933658978078877,3.840182343311194) circle (2.5pt);
\draw[color=qqqqff] (-2.16,4.21) node {$B$};
\draw [fill=qqqqff] (3.479331008575662,0.4248825897600912) circle (2.5pt);
\draw[color=qqqqff] (3.62,0.79) node {$C$};
\draw [fill=uuuuuu] (-0.2578750441772161,2.635925212978763) circle (1.5pt);
\draw[color=uuuuuu] (-0.9,2.49) node {$E$};
\draw [fill=uuuuuu] (2.113419968703404,1.2329963212622284) circle (1.5pt);
\draw[color=uuuuuu] (2.04,0.83) node {$D$};
\end{scriptsize}
\end{tikzpicture}
\end{center}

Comme $A$ et $B$ sont symétriques l'un de l'autre par rapport au bord de l'angle, on a $OA=OB$. On a de même $OA = OC$, donc \uline{$OB=OC$, et donc $OBC$ est isocèle.}

Par l'autre caractérisation des triangles isocèles, on en déduit \uline{$\widehat{OBC}=\widehat{OCB}$.}

Or, comme les symétries axiales préservent les angles (non orientés), on a $\widehat{OBC}=\widehat{OBE}=\widehat{OAE}$, et d'autre part, $\widehat{OCB}=\widehat{OCD}=\widehat{OAD}$.

Finalement, \fbox{$\widehat{OAE}=\widehat{OAD}$, et donc $[AO)$ est la bissectrice de $\widehat{EAD}$.}
\end{Soln}
\begin{Soln}{20}
C'est la moitié de l'aire du parallélogramme.
\end{Soln}
\begin{Soln}{22}
On peut découper en plusieurs petits triangles, ou bien voir que $(GF)//(HE)$. On peut alors calculer l'aire de la façon suivante:
\begin{center}
\definecolor{zzttqq}{rgb}{0.6,0.2,0.}
\definecolor{uuuuuu}{rgb}{0.26666666666666666,0.26666666666666666,0.26666666666666666}
\definecolor{xdxdff}{rgb}{0.49019607843137253,0.49019607843137253,1.}
\definecolor{qqqqff}{rgb}{0.,0.,1.}
\begin{tikzpicture}[line cap=round,line join=round,>=triangle 45,x=1.0cm,y=1.0cm]
\clip(-2.12,-0.96) rectangle (4.54,3.32);
\fill[color=zzttqq,fill=zzttqq,fill opacity=0.10000000149011612] (-1.56,1.31) -- (1.04,0.) -- (2.34,1.965) -- cycle;
\fill[color=zzttqq,fill=zzttqq,fill opacity=0.10000000149011612] (-1.56,1.31) -- (1.04,0.) -- (1.04,2.62) -- cycle;
\draw (-1.56,0.)-- (3.64,0.);
\draw (-1.56,2.62)-- (3.64,2.62);
\draw (-1.56,0.)-- (-1.56,2.62);
\draw (3.64,0.)-- (3.64,2.62);
\draw (1.04,2.62)-- (3.64,1.31);
\draw [color=zzttqq] (-1.56,1.31)-- (1.04,0.);
\draw [color=zzttqq] (1.04,0.)-- (2.34,1.965);
\draw [color=zzttqq] (2.34,1.965)-- (-1.56,1.31);
\draw [color=zzttqq] (-1.56,1.31)-- (1.04,0.);
\draw [color=zzttqq] (1.04,0.)-- (1.04,2.62);
\draw [color=zzttqq] (1.04,2.62)-- (-1.56,1.31);
\begin{scriptsize}
\draw [fill=qqqqff] (-1.56,0.) circle (2.5pt);
\draw[color=qqqqff] (-1.9,-0.17) node {$A$};
\draw [fill=qqqqff] (3.64,0.) circle (2.5pt);
\draw[color=qqqqff] (3.92,-0.01) node {$B$};
\draw [fill=xdxdff] (3.64,2.62) circle (2.5pt);
\draw[color=xdxdff] (3.78,2.99) node {$C$};
\draw [fill=uuuuuu] (-1.56,2.62) circle (1.5pt);
\draw[color=uuuuuu] (-1.82,2.97) node {$D$};
\draw [fill=uuuuuu] (1.04,0.) circle (1.5pt);
\draw[color=uuuuuu] (1.04,-0.35) node {$E$};
\draw [fill=uuuuuu] (3.64,1.31) circle (1.5pt);
\draw[color=uuuuuu] (3.78,1.59) node {$F$};
\draw [fill=uuuuuu] (1.04,2.62) circle (1.5pt);
\draw[color=uuuuuu] (1.18,2.91) node {$G$};
\draw [fill=uuuuuu] (-1.56,1.31) circle (1.5pt);
\draw[color=uuuuuu] (-1.94,1.55) node {$H$};
\draw [fill=uuuuuu] (2.34,1.965) circle (1.5pt);
\draw[color=uuuuuu] (2.48,2.25) node {$I$};
\end{scriptsize}
\end{tikzpicture}
\end{center}
On en déduit que l'aire du triangle vaut un quart de celle du rectangle.
\end{Soln}
\begin{Soln}{24}
On considère la symétrie centrale  de centre $O$ (le centre de $ABCD$). Comme une symétrie centrale préserve les distances, et de plus envoie une droite sur une droite parallèle, on voit que l'image de $I$ est $K$, et l'image de $J$ est $L$. Ceci montre que $IJKL$ est un parallélogramme.

%Pour la réciproque, on considère la symétrie centrale centrée sur le centre de $IJKL$, et on montre qu'elle envoie $A$ sur $D$ et $B$ sur $D$.
\end{Soln}
\begin{Soln}{25}
\begin{center}
\includegraphics{images/img007081-2}
\end{center}
\underline{Première solution} : soit $\sigma$ la symétrie centrale de centre $I$ et  $A''$ (respectivement $B''$) l'image de $A'$ (resp. $B'$) par $\sigma$.

Par construction, $A'B'A''B''$ est un parallélogramme de centre $I$, et par construction également, on a $B=\sigma(A)$.

Montrons que $A'B'A''B''$ est un rectangle. Comme une symétrie centrale envoie une droite sur une droite parallèle, l'image de la droite $(AA')$ lui est parallèle, et doit forcément contenir $\sigma(A)$ c'est-à-dire $B$. C'est donc la droite $(BB')$. Ceci montre que $A''$ est le point d'intersection des droites $(A'I)$ et $(BB')$, et donc que $A'B'A''B''$ est un rectangle.

Comme les diagonales d'un rectangle on même longueur, on a terminé.

\underline{Deuxième solution} : une projection orthogonale sur une droite préserve les milieux : on peut par exemple le prouver en considérant un repère orthonormé et des coordonnées. On voit alors que la  projection orthogonale $I'$ de $I$ sur $(A'B')$ est le milieu de $[A'B']$. Ceci signifie que dans le triangle $A'IB'$, la hauteur issue de $I$ est également la médiane issue de $I$. Le triangle $A'IB'$ est donc isocèle, d'où $A'I = B'I$.
\end{Soln}
\begin{Soln}{26}
Le quadrilatère $AMBM'$ est un parallélogramme car ses diagonales se croisent en leur milieu. Les segments $[AM]$ et $[M'B]$ sont donc parallèles et de même longueur.

De même, $AMCM''$ est un parallélogramme, et donc les segments $[AM]$ et $[M''C]$ sont donc parallèles et de même longueur.

On en déduit que $M'BCM''$ est un parallélogramme, et donc ses côtés opposés $[M'M'']$ et $[BC]$ sont (parallèles et) de même longueur.
\end{Soln}
\begin{Soln}{27}
Procédons par conditions nécessaires, puis suffisantes. (On dit aussi que l'on raisonne par analyse-synthèse.)

Si de tels points existent et que $APCQ$ est un losange, alors ses diagonales $[AC]$ et $[PQ]$ se croisent en leur milieu et sont orthogonales. Donc $(PQ)$ doit être la médiatrice de $[AC]$.

Faisons maintenant la synthèse : traçons la médiatrice de $[AC]$ : elle passe par le centre $O$ du rectangle, et coupe $[AB]$ et $[CD]$ en deux points $P$ et $Q$. De plus, $OP=OQ$, en effet, toute droite passant par le centre du rectangle coupe le rectangle en deux points qui sont à même distance du centre, par symétrie centrale.
\end{Soln}
\begin{Soln}{28}
Soit $O$ le centre de $ABCD$.

Comme $BA'=DC'$ et $(BA')//(DC')$, on en déduit que $BA'DC'$ est un parallélogramme.

\marginpar{1}

Son centre est le milieu de la diagonale $[BD]$ c'est-à-dire $O$, et donc $O$ est le milieu de $[A'C']$. (En d'autres termes, $A'$ et $C'$ sont symétriques para rapport à $O$.)

\marginpar{1}

On montre de même que $O$ est le milieu de $[B'D']$.

\marginpar{1}

Donc $A'B'C'D'$ est un parallélogramme, soit parce que ses diagonales se croisent en leur milieu, soit, en reformulant avec des symétries, parce que la symétrie centrale de centre $O$ envoie $A'$ sur $C'$ et $B'$ sur $D'$.

\marginpar{1}

Montrons que son aire est cinq fois celle de $ABCD$.

Le triangle $BA'B'$ a la même aire que $BAB'$, car ils ont des bases $[AB]$ et $[BA']$ de même longueur et la même hauteur.

\marginpar{1}

Ensuite, le triangle $BAB'$ a la même aire que $ABCD$, car sa base $[BB']$ est deux fois plus grande que la base $[BC]$ du parallélogramme, et les hauteurs relativement à ces bases sont égales.

\marginpar{1}

On peut faire un peu plus rapide avec la théorème des milieux, mais cette preuve-ci évite de faire appel à ce théorème.
\end{Soln}
\begin{Soln}{29}
La droite $(AO)$ coupe $[BC]$ en un point $A'$.
\begin{center}
\definecolor{uuuuuu}{rgb}{0.26666666666666666,0.26666666666666666,0.26666666666666666}
\definecolor{qqqqff}{rgb}{0.,0.,1.}
\begin{tikzpicture}[line cap=round,line join=round,>=triangle 45,x=1.0cm,y=1.0cm]
\clip(-2.8,-0.86) rectangle (3.42,3.38);
\draw (-1.24,2.92)-- (-1.98,-0.36);
\draw (-1.98,-0.36)-- (2.56,0.72);
\draw (2.56,0.72)-- (-1.24,2.92);
\draw [dash pattern=on 2pt off 2pt] (0.62,0.92)-- (-1.98,-0.36);
\draw [dash pattern=on 2pt off 2pt,domain=-2.8:3.42] plot(\x,{(--2.9512-2.*\x)/1.86});
\begin{scriptsize}
\draw [fill=qqqqff] (-1.24,2.92) circle (2.5pt);
\draw[color=qqqqff] (-1.74,3.01) node {$A$};
\draw [fill=qqqqff] (-1.98,-0.36) circle (2.5pt);
\draw[color=qqqqff] (-2.5,-0.27) node {$B$};
\draw [fill=qqqqff] (2.56,0.72) circle (2.5pt);
\draw[color=qqqqff] (2.9,0.81) node {$C$};
\draw [fill=qqqqff] (0.62,0.92) circle (2.5pt);
\draw[color=qqqqff] (0.76,1.29) node {$O$};
\draw [fill=uuuuuu] (1.123747204386408,0.378336339369454) circle (1.5pt);
\draw[color=uuuuuu] (1.32,0.67) node {$A'$};
\end{scriptsize}
\end{tikzpicture}
\end{center}
On applique l'inégalité triangulaire sur $AA'C$ et $OA'B$:
\[ AO+OA' \leq AC+CA',\]
\[ OB \leq OA'+A'B\]
En sommant, on obtient le résultat.
\end{Soln}
\begin{Soln}{31}
Considérons la symétrie axiale d'axe $\mathcal D$, et soit $B'$ l'image de $B$ par cette symétrie axiale.  Les points $A$ et  $B'$ sont situés de part et d'autre de la droite $\mathcal D$.

Si un chemin va de $A$ à $B$ en touchant la droite, alors à partir du moment où on touche la droite, il est équivalent d'aller vers $B$ ou d'aller vers $B'$, puisque la droite $\mathcal D$ est la médiatrice de $[BB']$.


Considérons alors le segment $[AB']$ : c'est le plus court chemin entre $A$ et $B'$. Il coupe $\mathcal D$ en un point $C$. Alors le plus court chemin de $A$ à $B$ qui touche la droite $\mathcal D$ consiste à aller de $A$ à $C$ en ligne droite, puis de $C$ à $B$ en ligne droite.

En utilisant qu'une symétrie axiale conserve les angles (non orientés), ainsi que des angles opposés par le sommet ou bien alternes-internes, on en déduit que les angles d'incidence et de réflexion sur la droite $\mathcal D$ sont égaux.

\end{Soln}
\begin{Soln}{33}
Il y a deux triangles vérifiant ces conditions : ce sont les deux triangles isocèles de base $AB$ et d'aire $\mathcal A$. Pour le démontrer, considérons les deux droites parallèles à $(AB)$ et à une distance $h =\frac{\mathcal A}{ AB}$ de la droite $(AB)$. Un triangle $ABC$ a une aire égale à $\mathcal A$ si et seulement si le point $C$ appartient à un de ces deux droites (suivant s'il est direct ou indirect).

Pour simplifier, supposons que l'on cherche uniquement les triangles $ABC$ directs, et appelons $\mathcal D$ la droite formée des points $C$ tels que $ABC$ soit direct et d'aire $\mathcal A$.

Il s'agit donc de trouver un tel point $C$ pour que le périmètre de $ABC$ soit minimal. Comme la distance $AB$ est fixe, il s'agit donc de minimiser $AC+CB$. On peut appliquer le résultat de l'exercice \ref{riviere}, d'après lequel le point $C$ qui convient est l'intersection de $\mathcal D$ et du segment joignant $A$ à l'image de $B$ par symétrie axiale d'axe $\mathcal D$. Ceci produit un triangle isocèle (utilisation d'angles opposés, ou alors alternes-internes).
\end{Soln}
\begin{Soln}{34}


Ce sont les deux triangles isocèles de base $[AB]$. Le problème est dual du précédent.

Autre solution, de plus haut niveau : fixer le périmètre revient à fixer la longueur $AC+CB$. Le point $C$ décrit donc une ellipse.

D'autre part,  par la formule $\mathcal A = AB \cdot h$ avec $h$ la longueur de la hauteur issue de $C$, le  ou les triangles d'aire maximale sont ceux pour lesquels $C$ est le plus éloigné de la droite $(AB)$.

On voit alors que l'aire maximale est atteinte lorsque $C$ est sur a médiatrice de $[AB]$.


\end{Soln}
\begin{Soln}{35}

Notons $O$ le milieu de $[BC]$, et considérons le symétrique $A'$ de $A$ par rapport à $O$. Alors $ABA'C$ est un rectangle : en effet c'est un parallélogramme car ses côtés sont deux à deux de même longueur (ou aussi : parallèles deux à deux, puisqu'une symétrie centrale envoie une droite sur une droite parallèle). Or, un parallélogramme qui possède un angle droit est un rectangle.

On en déduit que les médiatrices de ses côtés se croisent au centre de ce rectangle, qui est le milieu de ses diagonales, donc le milieu de $[BC]$, et donc $O$. Ceci montre que $O$ est le centre du cercle circonscrit à $ABC$.

Preuve de la réciproque:
\end{Soln}
\begin{Soln}{36}
On relie les points $A$ et $B$. On construit le milieu $I$ de $[AB]$. On a donc $AI = \frac{1}{2}AB$. On trace le cercle $\mathcal{C}$ de centre $A$ et de rayon $AI$. Ceci fait, On trace le cercle $\mathcal{C}'$ centr\'e en $I$ de diam\`etre $AB$. Nommons $C$ l'un des deux points d'intersection de $\mathcal{C}$ et $\mathcal{C}'$. Il s'agit de l'une des deux solutions possibles.

En effet, par construction $AC = \frac{1}{2} AB$, puisque $C$ appartient \`a $\mathcal{C}$. En outre $ACB$ est rectangle en $C$. En effet, $ACB$ est inscrit dans le cercle $\mathcal{C}'$ et $AB$ est un diam\`etre de $\mathcal{C}'$.
\end{Soln}
\begin{Soln}{37}
Comme par hypoth\`ese $DA = DC = DB$, les points $A$, $B$ et $C$ sont situ\'es sur le cercle de centre $A$ et de diam\`etre $AB$. Donc, $ACB$ est rectangle en $C$.

Puisque $ADC$ est \'equilat\'eral, $\widehat{DAC} = \widehat{BAC}= 60^{\circ}$. Comme $ACB$ est rectangle en $C$, donc $\widehat{ACB}=90^{\circ}$.

De la relation $180^{\circ} = \widehat{BAC} + \widehat{ACB} + \widehat{CBA}$ on tire $\widehat{CBA} = 30^{\circ}$.
\end{Soln}
\begin{Soln}{38}

%%%%%%%%%%%%
%% FIGURE %%
%%%%%%%%%%%%

D'après les hypothèses, la droite $(BC)$ est perpendiculaire \`a la droite $(AH)$ et $ADA'$ est rectangle en $D$. Donc, les droites $(AH)$ et $(A'D)$ sont perpendiculaires. Or, deux droites perpendiculaires à une même droite sont parallèles.
\end{Soln}
\begin{Soln}{39}
Les angles en $M$ et $N$ sont droits.

Dans le premier cas, les angles sont égaux car ils ont le même complémentaire.

Dans le second cas, ils sont supplémentaires car leur comme plus deux angles droits vaut $2\pi$ (dans un quadrilatère convexe, la somme des angles vaut $2\pi$).
\end{Soln}
\begin{Soln}{41}
Prendre un deuxième point $N$ sur le cercle de telle sorte que $(AM)$ et $(BN)$ se coupent en un point $C$. On peut alors construire l'orthocentre de $ABC$. La troisième hauteur fournit une droite orthogonale à $(AB)$, coupant le cercle en deux points $P$ et $Q$. On peut alors compléter $MPQ$ en un trapèze (isocèle) $MPQR$, en utilisant les diagonales d'un tel trapèze. La droite $(MR)$ est orthogonale à $(AB)$.
\end{Soln}
\begin{Soln}{42}
Les trois droites sont les hauteurs de $ABC$.
%%%%%%%%%%%%
%% FIGURE %%
%%%%%%%%%%%%

\end{Soln}
\begin{Soln}{43}
  Le point $A$ est l'orthocentre de $PQB$. L'angle est donc droit.
\end{Soln}
\begin{Soln}{44}
Dans un parallélogramme, les angles opposés sont égaux, et deux angles consécutifs sont supplémentaires. On en déduit que deux bissectrices consécutives sont perpendiculaires. La réciproque est vraie aussi : si les bissectrices consécutives sont perpendiculaires, alors deux angles consécutifs sont supplémentaires et donc on a un parallélogramme.
\end{Soln}
\begin{Soln}{45}
Méthodologie : tracer la figure avec le point $M$ construit (même approximativement), ce qui permet de réfléchir. Ensuite, tracer ce qui peut être tracé, par exemple ici la droite $(BN)$. On voit que cette droite doit être la bissectrice de l'angle $\widehat{CBA}$, bissectrice qui est constructible  priori. Une fois construite, elle coupe $[AC]$ en un point $N$. On construit alors $M$ en menant la parallèle.

Noter qu'en appelant $M'$ le symétrique de $M$ par rapport à la bissectrice $(BN)$, alors $M'$ est sur $[BC]$ et en plus $BMNM'$ est un losange, ce qui montre que la construction est correcte.
\end{Soln}
\begin{Soln}{47}
Notons $ABC$ un triangle quelconque, $I$ (resp. $J$) le pied de la médiane issue de $A$ (resp. $B$). Soit $M$ (resp. $N$) le symétrique de $G$ par rapport à  $I$ (resp. $J$).
\begin{center}
\definecolor{uuuuuu}{rgb}{0.26666666666666666,0.26666666666666666,0.26666666666666666}
\definecolor{qqqqff}{rgb}{0.,0.,1.}
\begin{tikzpicture}[line cap=round,line join=round,>=triangle 45,x=1.0cm,y=1.0cm]
\clip(-1.22,-4.32) rectangle (10.38,5.06);
\draw [line width=1.6pt] (-0.68,4.28)-- (4.08,4.28);
\draw [line width=1.6pt] (4.08,4.28)-- (3.06,0.4);
\draw [line width=1.6pt] (3.06,0.4)-- (-0.68,4.28);
\draw (-0.68,4.28)-- (3.57,2.34);
\draw (4.08,4.28)-- (1.19,2.34);
\draw [dash pattern=on 2pt off 2pt] (-0.68,4.28)-- (0.22666666666666646,1.6933333333333336);
\draw [dash pattern=on 2pt off 2pt] (0.22666666666666646,1.6933333333333336)-- (3.06,0.4);
\draw [dash pattern=on 2pt off 2pt] (4.08,4.28)-- (4.986666666666667,1.6933333333333336);
\draw [dash pattern=on 2pt off 2pt] (4.986666666666667,1.6933333333333336)-- (3.06,0.4);
\draw [dash pattern=on 2pt off 2pt] (0.22666666666666646,1.6933333333333336)-- (1.19,2.34);
\draw [dash pattern=on 2pt off 2pt] (3.57,2.34)-- (4.986666666666667,1.6933333333333336);
\draw [dash pattern=on 2pt off 2pt] (2.1533333333333338,2.9866666666666672)-- (3.06,0.4);
\begin{scriptsize}
\draw [fill=qqqqff] (-0.68,4.28) circle (2.5pt);
\draw[color=qqqqff] (-0.86,4.75) node {$A$};
\draw [fill=qqqqff] (4.08,4.28) circle (2.5pt);
\draw[color=qqqqff] (4.22,4.65) node {$B$};
\draw [fill=qqqqff] (3.06,0.4) circle (2.5pt);
\draw[color=qqqqff] (3.38,0.27) node {$C$};
\draw [fill=uuuuuu] (3.57,2.34) circle (1.5pt);
\draw[color=uuuuuu] (3.94,2.69) node {$I$};
\draw [fill=uuuuuu] (1.19,2.34) circle (1.5pt);
\draw[color=uuuuuu] (1.14,1.95) node {$J$};
\draw [fill=uuuuuu] (2.1533333333333338,2.9866666666666672) circle (1.5pt);
\draw[color=uuuuuu] (1.84,3.45) node {$G$};
\draw [fill=uuuuuu] (4.986666666666667,1.6933333333333336) circle (1.5pt);
\draw[color=uuuuuu] (5.12,1.99) node {$M$};
\draw [fill=uuuuuu] (0.22666666666666646,1.6933333333333336) circle (1.5pt);
\draw[color=uuuuuu] (-0.46,1.87) node {$N$};
\end{scriptsize}
\end{tikzpicture}
\end{center}

Alors, $AGCN$ et $BGCM$ sont des parallélogrammes car leurs diagonales se croisent en leur milieu.

On en déduit que $ABMN$ est un parallélogramme. Donc ses diagonales se coupent en leur milieu, ce qui montre que $AG = 2GI$ et $BG = 2GJ$. (Note : on aurait aussi peu utiliser que $GMCN$ est un parallélogramme.)

On a donc montré qu'étant donné un triangle, deux médianes quelconques se coupent toujours aux deux-tiers à partir des sommets. Ceci montre que les trois médianes sont forcément concourantes, puisqu'on peut échanger le rôle des deux dernières médianes.

(Autre façon de conclure, en utilisant le théorème des milieux : montrons que si $K$ est le milieu de $[AB]$, alors $(CK)$ contient le point $G$. Par le théorème des milieux appliqué au triangle $ABM$, le segment $[GK]$ est parallèle à $(BM)$, donc par ce qui précède, à $(GC)$. On en déduit que $K$, $G$ et $C$ sont alignés.)
\end{Soln}
\begin{Soln}{48}
\marginpar{1} Le point $K$ est le centre de gravité de $ABD$.

\marginpar{1} On en déduit que $(AK)$ est une autre médiane de ce triangle, et donc qu'elle coupe $[BD]$ en son milieu $O$.

\marginpar{1}  Et donc, $(AK)$ contient le centre du parallélogramme $ABCD$, c'est donc une diagonale et donc elle contient $C$.
\begin{center}
\definecolor{uuuuuu}{rgb}{0.26666666666666666,0.26666666666666666,0.26666666666666666}
\definecolor{qqqqff}{rgb}{0.,0.,1.}
\begin{tikzpicture}[line cap=round,line join=round,>=triangle 45,x=1.0cm,y=1.0cm]
\clip(-2.5444651162790692,0.027720930232561652) rectangle (4.839348837209292,4.7551627906976766);
\draw (-1.44,0.58)-- (-0.9447441860465123,3.213860465116279);
\draw (-0.9447441860465123,3.213860465116279)-- (3.9839069767441835,3.8321860465116298);
\draw (3.9839069767441835,3.8321860465116298)-- (3.4886511627906955,1.198325581395351);
\draw (3.4886511627906955,1.198325581395351)-- (-1.44,0.58);
\draw [dash pattern=on 3pt off 3pt] (-0.9447441860465123,3.213860465116279)-- (1.0243255813953478,0.8891627906976756);
\draw [dash pattern=on 3pt off 3pt] (-0.9447441860465123,3.213860465116279)-- (3.4886511627906955,1.198325581395351);
\draw [dash pattern=on 3pt off 3pt] (-1.44,0.58)-- (3.9839069767441835,3.8321860465116298);
\begin{scriptsize}
\draw [fill=qqqqff] (-1.44,0.58) circle (2.5pt);
\draw[color=qqqqff] (-1.9816744186046515,0.8944186046511661) node {$A$};
\draw [fill=qqqqff] (3.4886511627906955,1.198325581395351) circle (2.5pt);
\draw[color=qqqqff] (3.89386046511627,1.0970232558139565) node {$B$};
\draw [fill=qqqqff] (3.9839069767441835,3.8321860465116298) circle (2.5pt);
\draw[color=qqqqff] (4.254046511627898,4.338697674418607) node {$C$};
\draw [fill=uuuuuu] (-0.9447441860465123,3.213860465116279) circle (1.5pt);
\draw[color=uuuuuu] (-1.3963720930232568,3.775906976744188) node {$D$};
\draw [fill=uuuuuu] (1.0243255813953478,0.8891627906976756) circle (1.5pt);
\draw[color=uuuuuu] (1.192465116279065,1.322139534883724) node {$M$};
\draw [fill=uuuuuu] (0.36796899224806134,1.6640620155038768) circle (1.5pt);
\draw[color=uuuuuu] (0.40455813953487996,2.2451162790697703) node {$K$};
\draw [fill=uuuuuu] (1.2719534883720915,2.206093023255815) circle (1.5pt);
\draw[color=uuuuuu] (1.2825116279069717,2.8079069767441887) node {$O$};
\end{scriptsize}
\end{tikzpicture}
\end{center}
\end{Soln}
\begin{Soln}{50}
Les deux triangles coupés par la médiane ont la même hauteur et des bases de même longueur (égales à la moitié de la base du triangle d'origine).
\end{Soln}
\begin{Soln}{51}

\begin{center}
\definecolor{qqwuqq}{rgb}{0.,0.39215686274509803,0.}
\definecolor{xdxdff}{rgb}{0.49019607843137253,0.49019607843137253,1.}
\definecolor{uuuuuu}{rgb}{0.26666666666666666,0.26666666666666666,0.26666666666666666}
\definecolor{qqqqff}{rgb}{0.,0.,1.}
\begin{tikzpicture}[line cap=round,line join=round,>=triangle 45,x=1.0cm,y=1.0cm]
\clip(-3.6,-1.06) rectangle (4.3,4.2);
\draw [shift={(1.1,3.42)},color=qqwuqq,fill=qqwuqq,fill opacity=0.1] (0,0) -- (-157.7619672844355:0.6) arc (-157.7619672844355:-107.95975501222517:0.6) -- cycle;
\draw [shift={(1.1,3.42)},color=qqwuqq,fill=qqwuqq,fill opacity=0.1] (0,0) -- (-107.95975501222517:0.6) arc (-107.95975501222517:-58.15754274001482:0.6) -- cycle;
\draw (-2.96,1.76)-- (1.1,3.42);
\draw (2.257061141239912,1.556935450545904)-- (1.1,3.42);
\draw (2.257061141239912,1.556935450545904)-- (-2.96,1.76);
\draw [dash pattern=on 5pt off 5pt] (-0.93,2.59)-- (3.414122282479824,-0.30612909890819173);
\draw [dash pattern=on 5pt off 5pt] (2.257061141239912,1.556935450545904)-- (3.414122282479824,-0.30612909890819173);
\draw [dash pattern=on 5pt off 5pt] (3.414122282479824,-0.30612909890819173)-- (-2.96,1.76);
\draw (1.1,3.42)-- (0.5180407608266075,1.6246236336972695);
\begin{scriptsize}
\draw [fill=qqqqff] (-2.96,1.76) circle (2.5pt);
\draw[color=qqqqff] (-2.82,2.12) node {$A$};
\draw [fill=qqqqff] (1.1,3.42) circle (2.5pt);
\draw[color=qqqqff] (1.24,3.78) node {$B$};
%\draw [fill=uuuuuu] (-0.93,2.59) circle (1.5pt);
%\draw[color=uuuuuu] (-1.32,2.92) node {$I$};
\draw [fill=xdxdff] (2.257061141239912,1.556935450545904) circle (2.5pt);
\draw[color=xdxdff] (2.4,1.92) node {$C$};
\draw [fill=qqqqff] (3.414122282479824,-0.30612909890819173) circle (2.5pt);
\draw[color=qqqqff] (3.56,0.06) node {$D$};
\draw [fill=uuuuuu] (0.5180407608266075,1.6246236336972695) circle (1.5pt);
\draw[color=uuuuuu] (0.08,1.3) node {$M$};
\end{scriptsize}
\end{tikzpicture}
\end{center}
\marginpar{1} Soit $D$ le symétrique de $B$ par rapport à $C$.

\marginpar{1} Alors $AB=BD$ donc $ABD$ est isocèle en $B$.

\marginpar{1} De plus, $M$ est son centre de gravité puisqu'il est aux deux tiers d'une des médianes.

\marginpar{1} La droite $(BM)$ est donc la médiane issue de $B$, et donc également la bissectrice de $ABD$ issue de $B$. On en déduit que
\[\widehat{ABM} = \widehat{MBC}.\]
\end{Soln}
\begin{Soln}{52}
\begin{enumerate}
\item C'est le cas si et seulement s'il existe un entier $n$ tel que $225 = n+(n+1)+...+(n+4) = 5n + (0+1+2+3+4) = 5n+10$. Or, cette équation est équivalente à $215=5n$ donc à $n = 215/5$. Ici, comme $215$ est bien divisible par $5$, il y a bien une solution entière, c'est $n=215/5=43$. On a donc $225=43+44+45+46+47$.
\item Le même raisonnement aboutit à $219=4n$, qui n'a pas de solution entière car $219$ n'est pas pair, et donc a fortiori n'est pas divisible par quatre.
\end{enumerate}
Pour un corrigé différent, voir \url{http://www.animath.fr/IMG/pdf/coupe_animath_automne_17_corrige.pdf}

Ouverture pour le lycée : on peut se demander pour quels $k$ est-ce que $225$ est la somme de $k$ entiers consécutifs.
\end{Soln}
\begin{Soln}{53}
On trouve, en résolvant un système linéaire ou bien en étant un peu astucieux, qu'il y a eu 25 phases. Amandine a donc été $13$ fois gardienne, et comme le gardien change à chaque phase, elle a forcément été gardienne lors des phases $1$, $3$, $5$, ... et $25$. Ceci montre que c'est elle qui a marqué le sixième but, puisqu'elle était la gardienne durant la septième phase.

Pour un corrigé plus détaillé, voir \url{http://www.animath.fr/IMG/pdf/coupe_animath_automne_17_corrige.pdf}.
\end{Soln}
\begin{Soln}{54}
Appliquer la translation au cercle. (Si on n'a pas donné le centre du cercle, commencer par construire le centre.)

Les points d'intersection des deux cercles fournissent les (ou la) solutions du problème.
\end{Soln}
\begin{Soln}{56}
Soit $O$ le pt d'intersection. On note $\phi$ la translation qui envoie $A$ sur $B$, et $\psi$ celle qui envoie $B$ sur $C$. Alors $\phi\psi = \psi\phi$. L'image de $A$ est $C$ et l'image de $A'$ est $C'$, d'où le  parallélisme demandé.
\'Ecrire la solution de façon élémentaire avec des parallélogrammes.
\end{Soln}
\begin{Soln}{57}


\begin{center}
\begin{tikzpicture}[line cap=round,line join=round,>=triangle 45,x=1.0cm,y=1.0cm]
\clip(-3.94,-0.88) rectangle (5.24,4.22);
\draw(0.47,1.75) circle (1.7800280896660026cm);
\draw [dash pattern=on 5pt off 5pt] (-0.79572113832392,0.49842099729981115)-- (-0.7815790027001891,3.0157211383239195);
\draw [dash pattern=on 5pt off 5pt] (-0.7815790027001891,3.0157211383239195)-- (1.7357211383239195,3.0015790027001885);
\draw [dash pattern=on 5pt off 5pt] (1.7357211383239195,3.0015790027001885)-- (1.7215790027001878,0.4842788616760799);
\draw [dash pattern=on 5pt off 5pt] (1.7215790027001878,0.4842788616760799)-- (-0.79572113832392,0.49842099729981115);
\draw (-1.32,-0.02)-- (-1.3,3.54);
\draw (-1.3,3.54)-- (2.26,3.52);
\draw (2.26,3.52)-- (2.24,-0.04);
\draw (2.24,-0.04)-- (-1.32,-0.02);
\draw (-1.31,1.76)-- (0.48,3.53);
\draw (0.48,3.53)-- (2.25,1.74);
\draw (2.25,1.74)-- (0.46,-0.03);
\draw (-1.31,1.76)-- (0.46,-0.03);
\end{tikzpicture}
\end{center}
\end{Soln}
\begin{Soln}{58}
Tracer une figure en prolongeant les segments $[OP]$ et $[OR]$, et considérer une rotation de centre $O$ et d'angle $\pi/2$.
\end{Soln}
\begin{Soln}{61}
On construit l'image du carré par une rotation d'angle $\pi/4$. Les points d'intersection des deux carrés forment un octogone régulier qui répond à la question.
\end{Soln}
\begin{Soln}{62}
Les triangles $BGF$ et $BGC$ ont même aire. Ensuite, les triangles $BGC$ et $BAH$ ont même aire car ils se déduisent l'un de l'autre par rotation de centre $B$ et d'angle $\pi/2$. (On peut aussi utiliser le critère d'égalité des triangles et regarder les angles. Mais il est clair que la rotation fixe $B$, et envoie $G$ sur $A$, et $C$ sur $H$.)

Ensuite, on remarque que les triangles $BAH$ et $BJH$ ont même aire.

On fait de même pour le second triangle.

Ceci permet de conclure, en multipliant les aires par deux,  que le carré $BAFG$ a la même aire que le rectangle $BHKJ$, et que le carré $ACDE$ a même aire que le rectangle $CIKJ$, d'où le résultat.

C'est la preuve que l'on trouve dans Euclide, prop. 37 du libre I.
\end{Soln}
\begin{Soln}{63}
Pour montrer que $H$ est l'orthocentre du triangle $DMN$, il suffit de montrer que  $(AN)$ et $(CM)$ sont des hauteurs de ce triangle : leur point d'intersection $H$ sera alors l'orthocentre.\\

Soit $\rho$ la rotation de centre $O$ (le centre du carré) et d'angle $\pi/2$. Par définition d'un carré direct, on a $\rho(A)=B$, $\rho(B)=C$, $\rho(C)=D$ et $\rho(D)=A$.

On a de plus \underline{$\rho(M)=N$}. En effet, comme $M \in [AB]$, on a $\rho(M) \in [\rho(A)\rho(B)] = [BC]$, et d'autre part, comme $\rho$ est une isométrie, on a $AM = \rho(A)\rho(M) =  B\rho(M)$. Or il n'y a qu'un point sur $[BC]$ à distance $AM$ de $B$, et d'après l'énoncé c'est $N$.

La rotation $\rho$ envoie donc le triangle $DAM$ sur $ABN$. Comme c'est une rotation d'angle $\pi/2$, on en déduit que $(DM)\bot (AN)$ et donc que $(AN)$ est une hauteur de $DMN$. On procède de même pour la deuxième hauteur.

\emph{Remarque: on peut rédiger la solution sans rotations, juste en utilisant des angles complémentaires, mais c'est plus laborieux et moins éclairant, donc (fortement) déconseillé.}
\end{Soln}
\begin{Soln}{67}
\begin{enumerate}
\item Soit $\rho$ la rotation de centre $O$ et d'angle $\pi/2$. D'après l'énoncé, on a $\rho(B)=A$ et $\rho(D)=C$. Donc $[AC]$ est l'image de $[AD]$ par $\rho$, d'où on déduit que $(AC)\bot (BD)$ et que $AC=BD$.

\item Le quadrilatère $IJKL$ est toujours un parallélogramme, même sans hypothèses sur $ABCD$ (c'est le théorème de Varignon). En effet, par le théorème de Thalès (ou simplement le théorème des milieux) :
\[ \overrightarrow{IL} = \frac12 \overrightarrow{BD} = \overrightarrow{JK}.\]
Ceci signifie que les côtés $[IL]$ et $[JK]$ ont même longueur et sont parallèles, donc $IJKL$ est un parallélogramme.

Pour voir que c'est un carré, remarquons qu'on montre de même que
\[ \overrightarrow{IJ} = \frac12 \overrightarrow{AC} = \overrightarrow{LK},\]
et d'après la première question, $(AC)\bot (BD)$ et que $AC=BD$, donc $IJKL$ est un parallélogramme ayant un angle droit et des cotés consécutifs de même longueur. C'est donc un carré.
\end{enumerate}
\end{Soln}
\begin{Soln}{69}
Considérer une rotation de centre $O$ et d'angle $\pi/4$, et les images itérées de $E$ par cette rotation. Ce sont les sommets d'un carré.
\end{Soln}
\begin{Soln}{70}
Voici une solution dans une configuration \og générique\fg.

Commençons par analyser le problème. On suppose que $ABCD$ est direct et que $E\in [AB]$, $F \in [BC]$ etc.
Considérons la rotation de centre $O$ (le centre du carré), et d'angle $\pi/4$. L'image du segment $[EG]$ est un segment $[E'G']$, que l'on suppose distinct de $[FH]$. En fait, on suppose pour simplifier $E'\neq F$. Considérons alors la translation de vecteur $\overrightarrow{E'F}$. Elle envoie $G'$ sur un point $G''$ appartenant à la droite $(DA)$. Si on suppose que ce point est différent de $H$, alors on a $(DA) = (G''H)$.

Voici comment obtenir le point $G''$. On construit la perpendiculaire à $(EG)$ passant par $F$, et sur cette droite, on place le point $G''$ tel que $FG''=EG$ et $(\overrightarrow{EG},\overrightarrow{FG''}) = \pi/2$. Ceci permet de tracer la droite $(HG'')$ c'est-à-dire $(AD)$.

\begin{center}
\includegraphics{images/img007104-1}
\end{center}

On projette ensuite les points $E$ et $G$ sur cette droite, ce qui donne $A$ et $D$. On peut ensuite terminer la construction du carré.

Il existe d'autres solutions qui utilisent le théorème de l'angle au centre.
\end{Soln}
\begin{Soln}{71}
On peut supposer, sans perte de généralité, que $B$ et $C$ sont fixes et vérifient $BC=a$. On cherche alors le point $A$ tel que $CA=b$, tel que l'aire de $ABC$ soit maximale.

A priori, le point $A$ doit appartenir au cercle de centre $C$ et de rayon $b$.

Si $A$ est sur la perpendiculaire à $(BC)$ passant par $C$, alors l'aire est $\mathcal A = BC\cdot CA = ab$.

Si $A'$ est un autre point du cercle, en notant $H$ son projeté orthogonal sur $(BC)$, l'aire est $BC\cdot AH$ (base fois hauteur), or par Pythagore dans le triangle rectangle $AHC$ rectangle en $H$, on a $A'H \leq A'C=AC$.

Note : on peut aussi calculer l'aire avec un sinus et utiliser que $\sin\theta$ est maximal pour $\theta=\pi/2$, mais ceci compliqué au niveau collège et de toute façon la preuve est celle donnée plus haut avec Pythagore.
\end{Soln}
\begin{Soln}{72}
(Note : l'aire est forcément inférieure à $1$, puisque par Pythagore $AB\leq EB=1$.)

De $EB=EF$ on tire que $EFB$ est isocèle en $E$, et on conclut de la même façon que $DFE$ est isocèle en $F$.

Soit $E'$ (respectivement $F'$) le pied de la hauteur de $BEF$ (resp. $DFE$) issue de $E$ (resp. $F$).

Alors comme ces deux hauteurs sont aussi des médianes, on a $BE'=E'F$ et d'autre part $EF'=F'D$.
D'autre part, comme $ABE'E$, $EE'FF'$ et $F'FCD$ sont des rectangles (ils ont trois angles droits), on a finalement:
\[ AE=EF'=F'D  =  BE'=E'F=FC \]
et ces quantités  valent donc le tiers du côté du carré.

On applique alors Pythagore dans un des triangles rectangles, et on obtient $AE=\frac{1}{\sqrt 10}$. On en déduit que le carré a une aire de $9/10$.

Voir la correction d'Animath, ainsi que des autres exercices de 2014, ici : \url{http://www.animath.fr/IMG/pdf/2014-10-test-OFM-corrige.pdf}


\end{Soln}
\begin{Soln}{73}
Pour tout autre point $H'$ de $\mathcal D$, on écrit Pythagore dans le triangle $AHH'$, ce qui donne $AH'^2=AH^2+HH'^2$, d'où on déduit que $AH' > AH$.
\end{Soln}
\begin{Soln}{74}
Si $A'=B'$, alors le résultat est vrai.

Sinon, cela signifie que $(AA')$ et $(BB')$ sont deux droites différentes. Notons $H$ le projeté orthogonal de $A$ sur $(BB')$, c'est-à-dire le point tel que $(AH)\bot (BB')$.

Notons que $AHB'A'$ a trois angles droits donc est un rectangle, et donc en particulier $AH=A'B'$.

Si $H=B$, cela signifie que $AB=A'B'$ et on a fini.

Sinon, le triangle $ABH$ est rectangle en $H$ et d'après le théorème de Pythagore, on a $AB^2=BH^2+AH^2$. Comme $AH=A'B'$, ceci donne donc $AB^2=A'B'^2+BH^2$ donc $AB\geq A'B'$, ce qu'il fallait démontrer.

Note : avec des vecteurs et du produit scalaire, on n'a pas besoin de séparer les différents cas, les points peuvent être sur la droite, être confondus etc : une projection orthogonale diminue toujours les distances (au sens large).
\end{Soln}
\begin{Soln}{75}
\'Ecrire chacune des distances à l'aide d'aires de triangles.

Ou bien dessiner les trois petits triangles, chaque distance est une hauteur d'un petit triangle équilatéral, faire tourner ces hauteurs.
\end{Soln}
\begin{Soln}{76}
On a et $MM' = 2IJ$ et d'autre part $I$ et $J$ sont les projetés de $O$ et $O'$ sur la droite $D$, donc $IJ \leq OO'$ avec égalité ssi $(OO') // (IJ)$, dans ce cas le maximum est donc $2OO'$.
\end{Soln}
\begin{Soln}{81}
Après avoir suivi l'indication, translater ce cercle de manière à ce qu'il contienne le point $A$.
\end{Soln}
\begin{Soln}{82}
\begin{enumerate}
\item Par définition, $(IA)$ et $(IB)$ sont tangentes au cercle $\mathcal C$, donc $IA=IB$. On a de même $IA=IC$ et donc $I$ est le milieu de $[BC]$. Le triangle $ABC$ est donc un triangle d'écolier et il est rectangle en $A$.
\end{enumerate}
\end{Soln}
\begin{Soln}{83}
%%%%%%%%%%%%
%% FIGURE %%
%%%%%%%%%%%%

Pour montrer $CA=CP$, on va montrer que le triangle $CAP$ est donc isocèle en $C$.

On a les égalités d'angles:
\begin{align*}
\widehat{CAP} &= \widehat{OAB} \text{ car les angles sont opposés par le sommet}\\
&= \pi/2 - \widehat{ABO} \\
&= \pi/2 -\widehat{OPB}  \text{ car $POB$ est isocèle en $O$}\\
&= \widehat{APC}
\end{align*}

Le triangle $CAP$ est donc isocèle en $C$, et donc $CA=CP$.
\end{Soln}
\begin{Soln}{85}
Traçons la figure, où on a placé $I$ le milieu de $[AB]$, de telle sorte que $\frac12(OA,OB)=(OA,OI)$.

\begin{center}
\includegraphics{images/img007118-2}
\end{center}


Les angles $(AO,\mathcal T)$ et $(AI,IO)$ sont droits.
On a d'une part :
\[ 0=(\mathcal T,\mathcal T) =  (\mathcal T,AI) +(AI,AO)+ \pi/2, \]
et d'autre part, dans le triangle $AIO$:
\[ 0=(AI,AO)+(IO,IA)+(OA,OI)=(AI,AO)+\pi/2+(OA,OI).\]
Finalement, on a donc:
\[ (\mathcal T,AB) = (\mathcal T, AI) = -(AI,AO)-\pi/2 = (OA,OI)=\frac{1}{2}(OA,OB),\]
ce qu'il fallait démontrer.$\qed$
\end{Soln}
\begin{Soln}{86}
Considérons deux bissectrices du triangle (celles des angles $\widehat A$ et $\widehat B$ par exemple), ainsi que leur point d'intersection $I$. Alors $I$ est à égale distance des trois côtés, donc il est sur la troisième bissectrice, et c'est aussi le centre d'un cercle qui est tangent aux trois côtés.


\end{Soln}
\begin{Soln}{88}
En général (si les trois droites sont distinctes), quatre : les droites forment un triangle, et les cercles qui conviennent sont le cercle inscrit et les trois cercles exinscrits.
\end{Soln}
\begin{Soln}{90}
Deuxième indication: un des angles du triangle a une mesure $\geq \pi/3$, et un autre a une mesure $\leq \pi/3$.
\end{Soln}
\begin{Soln}{94}
Pour le cercle de centre $P$, il suffit de montrer que $P$ est le centre du cercle inscrit du triangle $MAB$. Pour cela, en notant $C$ le projeté orthogonal de $P$ sur $(MA)$, il suffit de montrer que $PC=PH$, ou de montrer que $AC=AH$. Or, on a $AC=AH = \cos(\widehat{AMO}) / OA$.

D'autres solutions sont possibles, par exemple avec des homothéties.
\end{Soln}
\begin{Soln}{95}
\end{Soln}
\begin{Soln}{96}
\marginpar{1} On a $MB=DN$ donc $MBND$ est un parallélogramme.

\marginpar{1} On en déduit que $(DM)//(BN)$.

\marginpar{1} Comme $M$ est le milieu de $[AB]$, on a par le théorème des milieux que $AK=KL$.

\marginpar{1} De la même façon, le théorème de Thalès appliqué dans $DCK$ entraîne que $KL=LC$, d'où le résultat.

\underline{Variante, demande plus d'initiative} :\\
Soit $Q$ le symétrique de $M$ par rapport à $B$. On a $MB=BQ$, donc $BQ=NC$ et $(BQ)//(NC)$. Donc $BQCM$ est un parallélogramme et donc $(NB)//(CQ)$.

Comme $AM=MB=BQ$ et que $(MK)//(BL)//(QC)$, le théorème de Thalès donne $AK=KL=LC$.

\underline{Autre preuve, avec centre de gravité} :\\
Dans le triangle $ABD$, le point $K$ est l'intersection des deux médianes $(DM)$ et $(AO)$. C'est donc le centre de gravité de $ABD$.

On en déduit que $KA = 2KO$.

Par symétrie centrale de centre $O$, on a $AK=CL$ et $KO=LO$, et finalement $AK = KO+OL = KL = LC$.
\end{Soln}
\begin{Soln}{98}
\begin{enumerate}
\item
Dans le triangle $ABC$, en notant $I$ est le milieu de $[AB]$ et $J$ le milieu de $[BC]$, le théorème de Thalès dit que $(IJ)$ est parall\`ele \`a $(AC)$ et $IJ = \frac{1}{2} AC$. On raisonne pareillement avec le triangle $ACD$, ce qui donne $(KL)$ parall\`ele \`a $AC$ et $KL = \frac{1}{2} AC$. Or, un quadrilat\`ere qui a deux c\^ot\'es parall\`eles et de m\^eme longueur est un parall\'elogramme.

\item La preuve la plus élémentaire utilise uniquement qu'une médiane d'un triangle donné le partage en deux triangles de même aire.

Soit $O$ le point d'intersection des diagonales du quadrilat\`ere $ABCD$. On consid\`ere le triangle $AOB$. Soit $O_1$ le point d'intersection de la diagonale $[AC]$ avec $[IL]$ et soit $O_2$ le point d'intersection de $[IJ]$ avec la diagonale $[BD]$.

% Par construction, le quadrilat\`ere $IO_1OO_2$ est un parall\'elogramme.

Par le théorème de Thalès,  $O_1$ est le milieu de $[AO]$ et $O_2$ le milieu de $[BO]$. Les triangles $IO_1A$ et $IO_1B$ ont même aire, de même que les triangles $I0_2O$ et $IO_2B$.

La somme des aires des triangles $AIO_1$ et $IO_2B$ est donc exactement \'egale \`a l'aire du parall\'elogramme $IO_1OO_2$.

On applique le m\^eme raisonnement aux triangles $BCO$, $CDO$ et $ADO$, ce qui signifie que, dans le quadrilat\`ere $ABCD$, la partie compl\'ementaire de $IJKL$ a une aire qui est exactement \'egale \`a celle de $IJKL$, ce qui permet de conclure. \end{enumerate}

\end{Soln}
\begin{Soln}{99}
Soit $\phi$ l'application du segment $[AB]$ dans lui-même qui a un point $D$ sur le segment associe le point $G$ comme construit dans l'énoncé. On veut montrer qu'appliquer deux fois de suite la fonction $\phi$ à un point revient à ne rien faire.

Pour comprendre l'application $\phi$, calculons les images de quelques points.
SI $D=A$, on voit on effectuant les trois projections que $\phi(A)=B$. On voit de la même manière que $\phi(B)=A$. L'application $\phi$ échange donc les deux extrémités du segment. D'autre part, on voit en utilisant le théorème des milieux trois fois de suite que l'image du milieu de $[AB]$ par $\phi$ est toujours le milieu de $[AB]$. Ceci porte à croire que l'application $\phi$ est la symétrie du segment par rapport à son milieu, autrement dit que si $D$ est un point de $[AB]$, alors $\phi(D)$ (autrement dit $G$ dans les notations de l'énoncé) est le point qui est à la même distance de $B$ que $D$ de $A$.

Autrement dit, on veut montrer:
\[ AD=BG\text{ ou bien, de façon équivalente: } BD =AG.\]
On prouve cette égalité en appliquant trois fois le théorème de Thalès (une fois pour chaque projection).
\end{Soln}
\begin{Soln}{102}
Le livre coûte sept kopecks.
\end{Soln}
\begin{Soln}{105}
Le nombre dont $a_n$ est le dernier chiffre est simplement $7n$. En considérant la table de multiplication par sept, on voit que les premières valeurs sont $7$, $4$, $1$, $8$, $5$, $2$, $9$, $6$, $3$, $0$, $7$ et à partir de ce moment la suite \og boucle \fg. On en déduit que les $n$ pour lesquels $a_n=0$ sont les multiples de $10$, et que la somme demandée vaut $45\cdot 200+7+4+1+8+5 = 9025$. Pour un corrigé plus détaillé, voir la correction d'Animath : \url{www.animath.fr/IMG/pdf/coupe_animath_automne_17_corrige.pdf}.
\end{Soln}
\begin{Soln}{106}
Elles sont parties à six heures, et se sont croisées au $2/5$ème du trajet pour la plus lente, au $3/5$ème pour la plus rapide.

Cet exercice tombe tout de suite si on trace le graphe des positions des deux dames au cours du temps et qu'on applique... le théorème de Thalès. On obtient immédiatement, en notant $t$ le temps écoulé entre le départ et midi :
\[ \frac{t+4}{t+9} = \frac{4}{t} = \frac{t}{9},\]
d'où $t=6$.
\end{Soln}
\begin{Soln}{107}
 Une de plus. On peut noter $f$ le nombre de filles et $g$ le nombre de garçons. Alors, Vasya a $f$ sœurs et $g-1$ frères, et la première phrase se traduit donc par $f = (g-1)+2$.
\end{Soln}
\begin{Soln}{108}
On déplie le patron du cube et on trace une ligne droite. Si le cube est de côté $1$, on trouve $\sqrt 5$.
\end{Soln}
\begin{Soln}{113}
La somme demandée vaut $n^2$.
Au lycée, on peut prouver le résultat demandé par récurrence, ou bien encore, si on connaît la formule $1+2+...+n = n(n+1)/2$, on peut écrire $1+3+5+...+(2n+1) = 1+2+3+4+...+2(n+1) -(2+4+6+(2(n+1)) = 1+2+...+2(n+1) -2(1+2+...+(n+1))$ et trouver le résultat.

Au collège, on peut expliquer la \og preuve sans mots\fg avec le dessin du carré quadrillé.
\end{Soln}
\begin{Soln}{114}
Méthode de Gau\ss : on écrit deux fois la somme en commençant par la fin la deuxième fois. On obtient $(101\times 100)/2 = 5050$.
\end{Soln}
\begin{Soln}{116}
Tracer des rayons de $\mathcal C$ et $\mathcal C'$ parallèles entre eux.
\end{Soln}
\begin{Soln}{117}

Analyse. Traçons comme suggéré une figure avec le carré déjà construit : on trace un carré puis on trace un triangle adéquat autour. On constate qu'un des côtés du carré, notons-le $[IJ]$, est parallèle à $[BC]$. Il y a une homothétie $h$ de centre $A$ qui envoie $[IJ]$ sur $[BC]$. Alors, l'image du carré $IJKL$ par $h$ est un carré dont un des côtés est $[BC]$. Notons $BCDE$ ce carré et traçons-le. On constate que $h(K)=D$ et $h(L)=E$, c'est-à-dire $K=h^{-1}(D)$ et $L = h^{-1}(E)$. Il ne reste plus qu'à faire la synthèse.


\end{Soln}
\begin{Soln}{118}
\begin{enumerate}
\item Le triangle des milieux est l'image de $ABC$ par l'homothétie de centre $G$ et de rapport $-1/2$. On en déduit que $\mathcal C$ est l'image du cercle circonscrit par cette homothétie, et donc que son centre est l'image de $\Omega$ par cette homothétie : il est donc sur la droite $(G\Omega)$.

\item Montrons que $\mathcal C'$ est l'image de $\mathcal C$ par l'homothétie de centre $H$ et de rapport $1/2$. Considérons la composition de l'homothétie de centre $G$ et de rapport $-1/2$ avec l'homothétie de rapport $-1$ et de centre $J$. C'est une homothétie de rapport $1/2$ qui envoie $\mathcal C$ sur $\mathcal C'$. Comme elle envoie de plus $\Omega$ sur $J$, son centre est le point $M$ tel que $\overrightarrow{MJ}=\frac12\overrightarrow{M\Omega}$. Or on sait déjà, par exemple en considérant l'homothétie de centre $G$ et de rapport $-\frac12$, que $\overrightarrow{G\Omega}=-\frac12\overrightarrow{GH}$. On en déduit que $M=H$.
\end{enumerate}
\end{Soln}
\begin{Soln}{123}
Soit $O$ le pt d'intersection. On note $\phi$ l'homothétie qui envoie $A$ sur $B$, et $\psi$ celle qui envoie $B$ sur $C$. Alors $\phi\psi = \psi\phi$. L'image de $A$ est $C$ et l'image de $A'$ est $C'$, d'où le  parallélisme demandé.
\end{Soln}
\begin{Soln}{125}
Ces triangles ont les mêmes angles, donc sont semblables.
\end{Soln}
\begin{Soln}{126}
Les triangles $ABC$, $ABH$ et $ACH$ sont semblables car ils ont à chaque fois deux (donc trois) angles identiques. Les rapports de longueurs de côtés homologues sont donc égaux, ce qui donne
\[ \frac{AB}{AC} = \frac{HB}{HA} = \frac{HA}{HC}\]
d'où on tire $HA^2 = HB\cdot HC$, ce qu'il fallait démontrer.\\

Remarque (autre preuve, plus sophistiquée) : Soit $A'$ le symétrique de $A$ par rapport à $[BC]$. Alors, $BACA'$ est inscriptible, et la puissance de $H$ par rapport à son cercle circonscrit est
\[ p_{\mathcal C}(H) = HB\cdot HC=  HA\cdot HA' = HA^2.\]
\end{Soln}
\begin{Soln}{127}
Les triangles ont les mêmes angles (alternes-internes).

Ou alors, on applique directement Thalès \og inversé\fg.
\end{Soln}
\begin{Soln}{128}
Les angles $\widehat{AIB}$ et $\widehat{DIC}$ sont opposés par le sommet donc leurs mesures sont égales.

On en déduit que les triangles $AIB$ et $DIC$ sont semblables car ils ont deux paires d'angles égaux.

On en déduit que leurs troisièmes angles sont égaux.
\end{Soln}
\begin{Soln}{131}
Ces nombres sont les neuf premiers multiples de $1001$, donc il suffit de vérifier pour celui-ci, puisque $143=11\times 13$.
\end{Soln}
\begin{Soln}{133}
$n-5$ doit être multiple de $36$.
$977$.
\end{Soln}
\begin{Soln}{134}
Comme les deux quantités sont positives, il suffit de vérifier que les carrés des deux quantités vérifient la même inégalité.

Ou alors, comme $a$ et $b$ sont positifs, on peut écrire $a=x^2$ et $b=y^2$ et on reconnait une identité remarquable.
\end{Soln}
\begin{Soln}{135}
Soient $a$ et $b$ les mesures des côtés du rectangle. L'aire du rectangle vaut donc:
\[ A=ab.\]

D'autre part, en  calculant le périmètre en fonction de $a$ et $b$ on obtient la contrainte :
\[ 2(a+b)=p.\]

Il s'agit donc de déterminer $a$ et $b$ tels que $a+b=p/2$, de telle façon à maximiser la quantité $ab$. Or, l'inégalité arithmético-géométrique donne
\[ \sqrt ab \leq \frac{a+b}{2} \]
avec égalité si et seulement si $a=b$, autrement dit, en élevant au carré et en écrivant le résultat en fonction de $p$ et de $A$:
\[ A \leq \frac{p^2}{16},\]
avec égalité ssi $a=b$.

Ceci montre que l'aire maximale est atteinte lorsque les deux côtés du rectangle sont égaux, c'est-à-dire lorsque le rectangle est un carré. Dans ce cas, le périmètre vaut $4a=4b$ et l'aire vaut $A=p^2/16 = a^2$.
\end{Soln}
\begin{Soln}{136}
Notons $a$ et $b$ les nombres de l'énoncé. On a $ab=100$.

On voit assez vite que la somme de $a$ et $b$ peut être aussi grande que l'on veut, par exemple l'on désire avoir une somme supérieure à un million, il suffit de choisir $a=1000000$, puis  $b=\frac{1}{10000}$. Plus généralement, pour avoir une somme supérieure à un nombre arbitraire $M>0$ il suffit de prendre $a=M$ et $b=100/M$.

Essayons donc de voir si la somme a une valeur minimale.

L'inégalité arithmético-géométrique fournit :
\[ \frac{a+b}{2}\geq \sqrt{ab}=10\]
avec égalité ssi $a=b$, donc la somme est supérieure à $20$, avec égalité ssi $a=b=10$.
\end{Soln}
\begin{Soln}{137}
L'exercice se résout assez simplement en utilisant trois triangles isocèles, mais on peut remarquer que les trois tangentes sont les trois axes radicaux, qui s'intersectent tous trois  au centre radical des trois cercles.
\end{Soln}
\begin{Soln}{138}
\begin{enumerate}
\item Les droites sont perpendiculaires.
\item Pour les tangentes communes extérieures, le cercle de centre $O'$ et de rayon $r'-r$ intersecte le cercle de diamètre $[OO']$ en deux points $C$ et $D$. Les droites $(O'C)$ et $(O'D)$ coupent $\mathcal C'$ en deux points $A'$ et $B'$. Les tangentes extérieures sont les parallèles à $(OC)$ et $(OD)$ passant par $A'$ et $B'$. Pour les tangentes intérieures, utiliser le cercle de centre $O'$ et de rayon $r'+r$.
\end{enumerate}
\end{Soln}
\begin{Soln}{142}
 Tracer le lieu des points à distance $R$ des cercles et droites en présence. Leurs éventuels points d'intersection fournissent des solutions.
\end{Soln}
\begin{Soln}{145}
Commencer par tracer la bissectrice, puis $(OA)$. Ensuite, tracer un cercle quelconque tangent aux deux droites (dans le même secteur angulaire), et utiliser une homothétie.

Note : les exercices faisant intervenir des homothéties se résolvent plus facilement en \og partant de la fin\fg, c'est-à-dire en procédant par analyse-synthèse et en faisant une figure approximative de ce que sera la solution.
\end{Soln}
\begin{Soln}{146}
Deuxième indication: un des angles du triangle a une mesure $\geq \pi/3$, et un autre a une mesure $\leq \pi/3$.
\end{Soln}
\begin{Soln}{147}
Soit $A'$ le point d'intersection de $(AD)$ avec $(BC)$. Alors $(BD)$ est à la fois une bissectrice et une hauteur de $BAA'$, donc $BAA'$ est isocèle en $B$.

On en déduit par le théorème des milieux (ou Thalès) que la droite passant par $D$ et le milieu de $[AB]$ est parallèle à $(BC)$.

Le même raisonnement pour le triangle $ACE$ montre que $(DE) // (BC)$, et que $(DE)$ passe par les milieux de $[AB]$ et $[AC]$.
\end{Soln}
\begin{Soln}{148}
Par le théorème de l'angle inscrit, c'est un arc de cercle, dont le centre est sur la médiatrice de $[AB]$.

Par le cas limite du théorème de l'angle inscrit, on sait aussi que si $\mathcal T$ est la tangente à ce cercle en $A$, alors $(\mathcal T,AB)=\alpha$.

On trace donc la droite $\mathcal T$ faisant un angle $\alpha$ avec $(AB)$ en $A$, puis la perpendiculaire à $\mathcal T$ passant par $A$. Cette droite coupe la médiatrice en un point $O$ qui est donc le centre du cercle recherché.
\end{Soln}
\begin{Soln}{149}
Construisons un triangle $AIB$ isocèle rectangle en $I$ et le cercle de centre $I$ et de rayon $IA$. Ce cercle intersecte la médiatrice de $[AB]$ en un point $O$ qui vérifie $\widehat{AOB}=\pm \pi/4$, par le théorème de l'angle au centre. C'est donc le centre d'un octogone appuyé sur $[AB]$. En traçant le cercle de centre $O$ et de rayon $OA$, on peut terminer la construction de cet octogone.
\end{Soln}
\begin{Soln}{150}
Commençons par rappeler deux points:
\begin{enumerate}
\item dans un trapèze, deux angles non adjacents à une même base sont supplémentaires, puisque les deux bases sont parallèles.
\item un quadrilatère non croisé est inscriptible ssi les angles opposés sont supplémentaires.
\end{enumerate}

Un trapèze est isocèle ssi les angles adjacents à une même base sont égaux, donc (par le premier point ci-dessus) ssi les angles opposés sont supplémentaires, donc (par le deuxième point) ssi il est inscriptible.

\begin{center}
\includegraphics{images/img007121-1}
\end{center}

\end{Soln}
\begin{Soln}{152}
Traçons une figure. \emph{On marque dès à présent quelques égalités d'angles obtenues par le théorème de l'angle inscrit:}

\begin{center}
\includegraphics{images/img007123-1}
\end{center}

\emph{Les égalités d'angles repérées sur la figure permettent de voir la solution, au moins dans la configuration particulière dessinée. On voit en effet que les angles $\widehat{ECD}$ et $\widehat{AEF}$ sont égaux. Attention toutefois, les angles géométriques sont trompeurs et les égalités que l'on voit sur une figure peuvent dépendre de la façon de tracer la figure. Sur la figure ci-dessous par exemple, les angles en question ne sont pas égaux mais supplémentaires.}

\begin{center}
\includegraphics{images/img007123-2}
\end{center}


\emph{Il ne reste plus qu'à rédiger rigoureusement la solution  avec des angles de droites, en s'appuyant sur l'intuition donnée par la figure.}

Pour montrer que $(CD)$ et $(EF)$ sont parallèles, il suffit par exemple de montrer qu'elles forment le même angle avec la droite $(CA)$. Or on a la suite d'égalités d'angles de droites :

\begin{align*}
(CD,CA) &= (BD,BA) \text{ car $CDAB$ est inscriptible}\\
&= (BF,BA) \text{ car $(BD)=(BF)$}\\
&=(EF,EA) \text{ car $BFAE$ est inscriptible}\\
&=(EF,CA)  \text{ car $(EA)=(CA)$.}
\end{align*}

\end{Soln}
\begin{Soln}{153}

Pour montrer le résultat, il suffit de montrer que $IBC$ et $JBC$ sont isocèles en $I$ et $J$.

On commence par prouver le résultat pour $I$ :

\begin{center}
\includegraphics{images/img007124-1}
\end{center}

Pour montrer que $BCI$ est isocèle en $I$, il suffit de montrer que $(BC,BI) = (CI,CB)$. Or, on a
\begin{align*}
(BC,BI) &= (AC,AI) \text{ car $ABIC$ est inscriptible}\\
&= (AI,AB) \text{ car $(AI)$ est une bissectrice de $(AC)$ et $(AB)$}\\
&= (CI,CB) \text{ car $ABIC$ est inscriptible.}
\end{align*}

On remarque qu'en rédigeant avec des angles de droites, on n'a pas eu besoin (ni en fait la possibilité) de préciser si la bissectrice était intérieure ou extérieure, ce qui implique que la preuve sera la même pour $J$. Traçons juste une figure pour visualiser la deuxième situation.


\begin{center}
\includegraphics{images/img007124-2}
\end{center}



\end{Soln}
\begin{Soln}{154}
Soit $\mathcal T$ la tangente commune  aux deux cercles.

\begin{center}
\includegraphics{images/img007125-2}
\end{center}

Par le cas limite du théorème des angles inscrits, on a
\[ (AB,AT) = (BT,\mathcal T)=(B'T,\mathcal T)=(A'B',A'T)\]

Comme $(AT) = (A'T)$, on en déduit que
\[ (AB,AT) = (A'B',AT),\]
et donc que $(AB)//(A'B')$.

\underline{Autre preuve:} considérer une homothétie de centre $T$ qui envoie un cercle sur l'autre.

\end{Soln}
\begin{Soln}{155}



Soient $\mathcal C$ et $\mathcal C'$ les cercles circonscrits à $ARQ$ et $BPR$.
Ils se coupent en $R$ et en un deuxième point $T$.

\begin{center}
\includegraphics{images/img007126-1}
\end{center}

Il s'agit de montrer que $T, P, C, Q$ sont cocycliques.


Par le cours,  il suffit de montrer l'égalité d'angles de droites $(QT,QC)=(PT,PC)$. Or on a :
\begin{align*}
(QT,QC)&=(QT,QA) \text{ car $(QC)=(QA)$}\\
&=(RT,RA) \text{ car $AQTR$ est inscriptible} \\
&= (RT,RB) \text{ car $(RA)=(RB)$} \\
&=(PT,PB) \text{ car $PTRB$ est inscriptible} \\
&=(PT,PC) \text{ car $(PB)=(PC)$.}
\end{align*}

Attention, si on utilise des angles géométriques au lieu des angles de droites pour rédiger la solution, on peut être amené à distinguer plusieurs configurations possibles, par exemple celle-ci:
\begin{center}
\includegraphics{images/img007126-2}
\end{center}

(Les angles $\widehat{BRT}$ et $\widehat{BPT}$ sont supplémentaires dans la première figure, et égaux dans la seconde.)

\end{Soln}
\begin{Soln}{156}

Le quadrilatère $ABA'B'$ est inscriptible dans un cercle de diamètre $[AB]$. En effet, les triangles $ABA'$ et $ABB'$ sont par définition rectangles en $A'$ et $B'$, et ont même hypoténuse $[AB]$.

De même, les quadrilatères $BCB'C'$ et $CAC'A'$ sont inscriptibles dans des cercles de diamètre $[BC]$ et $[CA]$.

\begin{center}
\includegraphics{images/img007131-1}
\end{center}

Montrons que la hauteur $(BB')$ est une bissectrice des droites $(B'C')$ et $(B'A')$. Pour cela, on montre que $(B'C',B'B)=(B'B,B'A')$.

On a :
\begin{align*}
(B'C',B'B)
&= (CC',CB) \text{ (car $BCB'C'$ est inscriptible)}\\
&= (CC',CA') \text{ (mêmes droites)}\\
&= (AC'AA') \text{ (car $ACA'C'$ est inscriptible)}\\
&= (AB,AA') \text{ (mêmes droites)}\\
&= (B'B,B'A') \text{ (car $ABA'B'$ est inscriptible)}\\
\end{align*}

\end{Soln}
\begin{Soln}{157}
Traçons une figure :

\begin{center}
\includegraphics{images/img007128-1}
\end{center}


\emph{[Sur la figure, on voit que les angles $\widehat{GFD}$ et $\widehat{GED}$ sont supplémentaires, car $\widehat{GFD}=\widehat{BFA}$ et $\widehat{GED}=\widehat{GEB}+\widehat{BED} = \widehat{FBA}+\widehat{BAF}$. Il ne reste plus qu'à rédiger cette preuve un peu plus rigoureusement avec des angles de droites.]}

Montrons que $(FD,FG)=(ED,EG)$, ce qui prouve que $EDFG$ est inscriptible.

Tout d'abord, comme $(FD)=(FA)$ et $(FG)=(FB)$, on a
\[(FD,FG)=(FA,FB).\]

Ensuite, la somme des angles du triangle $ABF$ vaut $\pi$, donc en termes d'angles de droites on a la relation
$(FA,FB)+(AB,AF)+(BF,BA)=0$, c'est-à-dire:
\[
(FA,FB) = (AF,AB)+(BA,BF).
\]
Calculons chacun de ces deux angles. D'une part, on a :
\begin{align*}
(AF,AB) &= (AD,AB) \text{ car $(AD)=(AF)$}\\
&=(ED,EB) \text{ car $ABDE$ est inscriptible}.
\end{align*}
Et d'autre part :
\begin{align*}
(BA,BF) &= (BA,BC) \text{ car $(BF)=(BC)$} \\
&= (CB,CA) \text{ car $ABC$ est isocèle en $A$}\\
&= (EB,EA) \text{ car $ABCE$ est inscriptible.}
\end{align*}

Finalement, on obtient donc:
\begin{align*}
(FD,FG)&=(FA,FB) \\
&= (AF,AB)+(BA,BF) \\
&= (ED,EB) + (EB,EA)\\
&= (ED,EA)\\
&= (ED,EG) \text{ car $(EG)=(EA)$,}
\end{align*}
ce qu'il fallait démontrer.
\begin{center}
\includegraphics{images/img007128-1}
\includegraphics{images/img007128-2}
\end{center}

\end{Soln}
\begin{Soln}{158}
  Traçons une figure. \emph{[Le fait de marquer toutes les égalités d'angles disponibles donne le résultat. Sur la figure, on ne marque que celles utilisées dans la rédaction proposée.]}
\begin{center}
\includegraphics{images/img007129-1}
\end{center}

Montrons que $(PA,PC)=(DQ,BQ)$. On a:
\begin{align*}
(PA,PC)&= (PA,PQ)+(PQ,PC) \\
&= (BA,BQ)+(DQ,DC) \text{ par cocyclicité dans chaque cercle}\\
&= (BA,BQ)+(DQ,BA) \text{ car $(DC)=(BA)$}\\
&= (DQ,BQ).
\end{align*}

\end{Soln}
\begin{Soln}{162}
La somme des angles d'un quadrilatère convexe vaut $2\pi$ :
\begin{align*}
2\pi &=
\widehat{ABC}+\widehat{BCD}+\widehat{CDA}+\widehat{DAB}\\
&= 2\widehat{ABI}
+2\widehat{KCD}+2\widehat{CDK}+2\widehat{IAB}
\end{align*}
d'où \[\widehat{ABI}
+\widehat{KCD}+\widehat{CDK}+\widehat{IAB}=\pi,\]
autrement dit la somme des demi-angles vaut $\pi$.

On termine alors la preuve en utilisant le critère de cocyclicité.

\end{Soln}
\begin{Soln}{163}
On trouve un parallélogramme dont les angles sont les supplémentaires de ceux de $ABCD$.
\end{Soln}
\begin{Soln}{165}

Si $ABC$ est rectangle, l'orthocentre coïncide avec un des sommets et la vérification de l'assertion est relativement facile. Dans la suite on suppose qu'on n'est pas dans ce cas.

Par définition, $H'$ est le symétrique de $H$ par rapport à $(AC)$ si $(AC)$ est la médiatrice de $[HH']$. C'est cela qu'on doit montrer.



D'autre part, par définition, on a $(AC) \bot (HH')$, donc $(AC)$ est la hauteur de $AHH'$ issue de $A$.

Donc si $AHH'$ est isocèle en $A$, alors cette hauteur de $AHH'$  est aussi la médiane issue de $A$ et c'est encore la médiatrice du côté opposé à $A$ c'est-à-dire $[HH']$.

Il suffit donc de montrer que $AHH'$ est isocèle en $A$. Pour cela, il suffit de montrer que les angles adjacents à la base sont égaux, autrement dit $\widehat{AHH'} = \widehat{AH'H}$ avec des angles géométriques non orientés, ou plus précisément avec des angles orientés $(H'A,H'H)=(HH',AH)$.

Suivant la méthodologie habituelle, on marque de façon systématique les angles égaux (ou complémentaires, supplémentaires etc) sur la figure. Ceci indique la marche à suivre pour la preuve.


\begin{center}
\includegraphics{images/img007139-2}
\end{center}

Montrons que $(H'A,H'H)=(HH',AH)$. On a :
\begin{align*}
(H'A,H'H) &= (H'A,H'B) \text{ (car $(H'H)=(HB)$}\\
&= (CA,CB) \text{ (car $ABCH'$ est inscriptible)}\\
&= (CA,AH)+(AH,CB) \text{ (par Chasles)}\\
&= (CA,AH)+\pi/2 \text{ (car $(AH)$ est une hauteur de $ABC$)}\\
&= (CA,AH)+(HH',CA)\\
&= (HH',AH) \text{ (par Chasles)}
\end{align*}


\end{Soln}
\begin{Soln}{166}

Rappelons la figure :

\begin{center}
\includegraphics{images/img007140-1}
\end{center}

\begin{enumerate}
\item On rappelle que le milieu de l'hypoténuse d'un triangle rectangle est le centre de son cercle circonscrit (une autre façon de le dire est que l'hypoténuse est un diamètre du cercle circonscrit).

Si $IO=IA$, cela signifie que $I$ est sur la médiatrice de $[OA]$. D'autre part, $AOB$ est rectangle en $O$ et $I$ est par définition sur l'hypoténuse $[AB]$. Donc $I$ est  l'intersection de l'hypoténuse et d'une médiatrice d'un autre côté, c'est donc le milieu de l'hypoténuse par la propriété rappelée plus haut. Il est donc suffisant de montrer que $IO=IA$.


\item Pour montrer que $IO=IA$, il suffit de montrer que $IOA$ est isocèle en $I$, c'est-à-dire que $(AI,AO)=(OA,OI)$. Or, on a :
\begin{align*}
(AI,AO)& = (AB,AC) \text{ (mêmes droites)}\\
&= (DB,DC) \text{ (car $ABCD$ est inscriptible}\\
&= (DO,DH) \text{ (mêmes droites)}\\
&= (DO,OH)+(OH,DH) \text{ (par Chasles)}\\
&= (DO,OH)+\pi/2 \text{ (par définition de $H$)}\\
&= (OB,OI)+\pi/2 \text{ (mêmes droites)}\\
&= (OB,OI) + (OA,OB) \text{ (car $(OA)\bot (OB)$ d'après l'énoncé)}\\
&= (OA,OI) \text{ (par Chasles)}
\end{align*}
\end{enumerate}

\end{Soln}
\begin{Soln}{167}
Il suffit de montrer que $\frac{PA}{PB'} = \frac{PA'}{PB}$. C'est le cas car $PAB'$ et $PBA'$ sont semblables.
\end{Soln}
\begin{Soln}{168}
\begin{enumerate}
\item Par construction, la droite $\Delta$ est la hauteur et la médiane de $ABD$. Le triangle est donc isocèle en $A$ et $\Delta$ est également sa bissectrice, ce qui montre que $\widehat{BAA_1} = \widehat{A_1AD}$ et donc que $D\in (AC)$.
\item Il suffit de prouver que $(A_2A_1,A_2C)=(DA_1,DC)$.
\begin{center}
\includegraphics{images/img007168-2}
\end{center}
Or on a :
\begin{align*}
(A_2A_1,A_2C)
&= (A_2A,A_2C)\\
&= (BA,BC) \text{ (car $ABCA_2$ est inscriptible)}\\
&= (BA,BA_1) \text{ (mêmes droites)}\\
&= (DA,DA_1) \text{ (par réflexion suivant $\Delta$)}\\
&= (DC,DA_1) \text{ (mêmes droites)},
\end{align*}
ce qu'il fallait démontrer.
\item Remarquons déjà que $AB=AD$.  D'autre part, il suffit de montrer que
\[ \frac{AA_1}{AD} = \frac{AC}{AA_2}.\]

La question précédente montre que les triangles $ADA_1$ et $AA_2C$ sont (inversement) semblables, puisqu'ils ont deux (et donc trois) angles en commun. Les rapports entre les côtés sont donc égaux, c'est-à-dire précisément
\[ \frac{AA_1}{AD} = \frac{AC}{AA_2}.\]

Remarque : si on connaît la notion de puissance d'un point par rapport à un cercle, on peut conclure plus vite : les deux produits égaux sont la puissance de $A$ par rapport au cercle $A_1A_2CD$.
\end{enumerate}
\end{Soln}
