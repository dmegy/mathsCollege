\documentclass[11pt,a4paper]{article}
\usepackage[francais]{babel}
\usepackage[utf8]{inputenc}
\usepackage{amsmath,amssymb,amsthm}
\usepackage{mathrsfs,stmaryrd}
\usepackage{fancybox,multicol,comment,datetime}
\usepackage{framed} % pour encadrer
\usepackage{fontawesome}
\usepackage{fourier}

\usepackage{hyperref}
\hypersetup{
    colorlinks=true,       % false: boxed links; true: colored links
    linkcolor=[rgb]{0,0.2,0.6},          % color of internal links
    citecolor=[rgb]{0,0.2,0.6},        % color of links to bibliography
    filecolor=[rgb]{0,0.2,0.6},      % color of file links
    urlcolor=[rgb]{0.7,0.2,0.2}           % color of external links
}

\usepackage[all]{hypcap} % règle pb cible des liens hyperref
\usepackage{pgf,pgfmath,tikz}
\usetikzlibrary{arrows}
\usetikzlibrary[patterns]
\tikzset{every picture/.style={execute at begin picture={
   \shorthandoff{:;!?};}
}}

\usepackage[margin=2.5cm]{geometry}



\usepackage[normalem]{ulem} % pour souligner avec changements de ligne
\usepackage{esvect} % pour jolis vecteurs avec \vv

\everymath{\displaystyle} 



\theoremstyle{definition}
\newtheorem{theoreme}{Théorème}[section]
\newtheorem{definition}[theoreme]{Définition}
\newtheorem{lemme}[theoreme]{Lemme}
\newtheorem{proposition}[theoreme]{Proposition}
\newtheorem{corollaire}[theoreme]{Corollaire}
\newtheorem{remarque}[theoreme]{Remarque}
\newtheorem{ex}{Exercice}


\newcommand{\N}{\mathbb N}
\newcommand{\Z}{\mathbb Z}
\newcommand{\Q}{\mathbb Q}
\newcommand{\R}{\mathbb R}
\newcommand{\C}{\mathbb C}
\newcommand{\U}{\mathbb U}
\newcommand{\F}{\mathbb F}


%%%%%%%%%%%%%%%%%%%%%%%%%%%%%%%%%
%%%%%% MISE EN FORME CLUB %%%%%%%
%%%%%%%%%%%%%%%%%%%%%%%%%%%%%%%%%

%\pagestyle{empty}



% En-tête des feuilles :

\newcommand{\enTete}[1]{
\noindent \textbf{\textsf{\href{http://depmath-nancy.univ-lorraine.fr/club/}{Club Mathématique de Nancy} \hfill Institut Élie Cartan}}
\hrule
\begin{center}
{\Huge \textbf{#1}}
\end{center}
\hrule
\vspace{1em}
}


%----- Structure des exercices ------

\newenvironment{prerequis}{(\bfseries Prérequis:}{)\newline}




% - - - - - - - - - - - - - -
% PARAMETRAGE DU PACKAGE ANSWERS 
% POUR LES INDICATIONS ET CORRECTIONS
% - - - - - - - - - - - - - - 

\usepackage{answers}



\Newassociation{sol}{Soln}{solutions}
% ira dans le fichier d'identifiant 'solutions'
% et écrira les solutions dans un environnement 'Soln'
\Newassociation{hint}{Hint}{indications}

%\newenvironment{exo}{
%\begin{ex} \label{enonce.\theex}
%
%}
%{~\hyperref[indication.\theex]{[Indications]}
%~\hyperref[solution.\theex]{[Correction]}~\end{ex} }


\usepackage{xparse}
\NewDocumentEnvironment{exo}{o}
 {\label{enonce.\theex} \IfNoValueTF{#1}
   {\ex\addcontentsline{toc}{subsubsection}{\protect\numberline{\theex}  }}
   {\ex[#1]\addcontentsline{toc}{subsubsection}{\protect\numberline{\theex}#1}}%
   \ignorespaces}
 {~\hyperref[indication.\theex]{[Indications]}
~\hyperref[solution.\theex]{[Correction]}~\endex\medskip\hrule}
 
 


\renewenvironment{Hint}[1]{
\noindent \textbf{Indications pour l'exercice #1} (retour à l'\hyperref[enonce.#1]{énoncé}, voir la \hyperref[solution.#1]{correction}) 
\phantomsection\label{indication.#1}
}

\renewenvironment{Soln}[1]{
\noindent \textbf{Correction de l'exercice #1} (retour à l'\hyperref[enonce.#1]{énoncé}, retour à l'\hyperref[indication.#1]{indication}) 
\phantomsection\label{solution.#1}
}




% - - - - - - - - - - - - - - 
% FIN PARAMETRAGE ANSWERS
% - - - - - - - - - - - - - - 

%-----------------------------
% MACROS POUR LES FEUILLES DE TD



\newcommand{\debut}{
	\Opensolutionfile{indications}[\jobname_hints]
	\Opensolutionfile{solutions}[\jobname_sol]
	
}
\newcommand{\fin}{
	\Closesolutionfile{indications}
	\Closesolutionfile{solutions}
	\section{Indications pour la résolution des exercices}
	\Readsolutionfile{indications}
	\newpage
	\section{Correction des exercices}
	\Readsolutionfile{solutions}
}
 % les corrections sont regroupées à la fin du document
%


\newenvironment{hint}{$ $\\ \textbf{Indication:}}{}
\newenvironment{sol}{$ $\\ \textbf{Correction:}}{}

\newcommand{\debut}{}
\newcommand{\fin}{}

% Environnement normal (sans answers) pour les exercices :
%\newenvironment{exo}{\begin{ex}}{\end{ex}\medskip\hrule}


% Alternativement, pour mettre les exos dans la table des matières :(nécessite xparse)
\usepackage{xparse}
\NewDocumentEnvironment{exo}{o}
 {\IfNoValueTF{#1}
   {\ex\addcontentsline{toc}{subsubsection}{\protect\numberline{\theex}  }}
   {\ex[#1]\addcontentsline{toc}{subsubsection}{\protect\numberline{\theex}#1}}%
   \ignorespaces}
 {\endex\medskip\hrule}

 % sans le package answers : les correction suivent les énoncés



%\excludecomment{prerequis} % décommenter pour maxquer les prérequis
%\makeindex
% et ne pas oulbier de faire un vrai "makeindex"

\begin{document}


\debut % pour le début des macros answers, si applicable

\enTete{Défis mathématiques  pour le collège\\
\faCogs\\
\small{(classés par niveau, programmes 2016)\\
Version préliminaire, compilée le \today{}}}


\begin{multicols}{2}
\tableofcontents
\end{multicols}



\section{Introduction}

\subsection{Présentation du document}

Ce document est une liste d'exercices niveau collège mais pouvant être assez difficiles (pour élèves motivé(e)s / pour motiver les élèves).


Il a été écrit dans l'intention d'être conforme aux programmes de 2016. La référence est le BO et, lorsque ça ne permettait pas d'être totalement sûr, les manuels Sésamath 2016, qui sont disponibles gratuitement en version électronique (à l'adresse \url{http://manuel.sesamath.net/index.php?page=telechargement}). 
Les exercices sont classés par année, de la sixième à la troisième, puis par thèmes. Toute suggestion est la bienvenue.

En complément, on propose quelques exercices sur des notions ayant récemment disparu des programmes de collège (typiquement la notion de droite tangente à un cercle ou de centre de gravité). Ces exercices sont clairement distingués de ceux qui peuvent être traités en se limitant au programme et sont placés dans des sections à part. Les anciennes éditions des manuels Sesamath, toujours accessibles gratuitement en ligne, sont de bonnes sources d'exercices sur ces thèmes.
 
Quasiment tous les exercices sont corrigés, mais les corrections destinées aux enseignants : le style de rédaction est inégal, certaines solutions sont incomplètes ou utilisent des formulations et notations non étudiées au collège (par exemple, certains exercices sur les angles sont rédigés à l'aide d'angles orientés de droites).


Dans la mesure du possible, on a privilégié les exercices qui demandent un raisonnement un minimum complexe, articulé en plusieurs points. La géométrie prédomine donc largement, mais il existe aussi de beaux exercices d'arithmétique et de combinatoire abordables au collège et on s'est efforcé d'en inclure le plus possible. 


Certains exercices demandent plus d'initiative que d'autres : compléter une figure, placer des points supplémentaires des points pour voir une symétrie, etc. Ces exercices sont signalés par le symbole \faLightbulbO. L'icône \faCogs{} permet de distinguer des exercices plus difficiles et l'icône $\heartsuit$ orne les exercices qui ont particulièrement plu.


%\paragraph{Commentaires techniques sur le document lui-même :}

Ce document ainsi que son fichier source modifiable (en \LaTeX) sont disponibles à l'adresse :

\url{http://depmath-nancy.univ-lorraine.fr/club}.
 
%Pour masquer les prérequis, décommenter simplement la ligne
%\verb+\excludecomment{prerequis}+

\section{Sources d'exercices}

Les manuels et cahiers Sésamath actuels ainsi que les anciennes éditions toujours disponibles sont de bonnes sources d'exercices. Citons également :

\begin{enumerate}
\item Le concours Kangourou.
\item Les rallyes mathématiques. Exemples : 
\begin{itemize}
\item \url{http://apmeplorraine.fr/doc/Rallye%20Math%C3%A9matique%20de%20Lorraine%202016.pdf}, 
\item \url{http://apmeplorraine.fr/doc/Rallye%20Math%C3%A9matique%20de%20Lorraine%202017.pdf},
\end{itemize}
Voir aussi les rallyes d'autres académies. En général, certains exercices sont abordables au collège.
\item Les anciens manuels de mathématiques de collège, par exemple la série Lebossé-Hémery (années 60) dont sont tirés quelques exercices présentés ici. 
\item Les exercices des olympiades académiques sont parfois en partie accessibles au collège;
\item Les exercices estampillés \og collège \fg{} dans les coupes Animath et leurs éliminatoires, ainsi que des anciens test d'entrée de l'OFM. On en a inclus cinq dans ce document.
\item \href{http://fr.wikipedia.org}{Wikipedia}, \href{http://mathworld.wolfram.com/}{MathWorld}, \href{https://www.cut-the-knot.org/}{Cut the Knot}, \href{https://www.brilliant.org}{Brillant.org}, ...
\item Le jeu \href{https://www.euclidea.xyz}{Euclidea} sur ordinateur ou smartphone.
\item Sans oublier les nombreux blogs et sites web personnels de collègues. (\href{http://serge.mehl.free.fr/base/index_concr1.html}{Chronomath}, \href{http://debart.pagesperso-orange.fr/}{Descartes et les mathématiques}, \href{http://fandegeometrie.blogspot.fr/}{Fan de géométrie}, \href{http://mathafou.free.fr/}{Maths en folie}, ...) 
\end{enumerate}


\section{Classe de 6ème}
%Pièges classiques à éviter : pas de nombre négatifs ni de calcul littéral.

\emph{Contenu du Sesamath: nombres naturels, fractions, nombres décimaux, distances et cercles, angles, milieux, droites parallèles et perpendiculaires, triangles et quadrilatères, symétrie axiale, axe de symétrie, médiatrice d'un segment (équivalence des deux définitions) et bissectrice d'un angle, triangles isocèles et équilatéraux, quadrilatères particuliers.}



\subsection{Divers}

\begin{exo}[Somme et différence]
% source : Huffington Post, 10 problèmes de CM2
Jacques et Jean comptent leurs CD. Ils en ont $54$ à eux deux. Jean en a $12$ de moins que Jacques.

Combien possèdent-ils de CD chacun?
\begin{sol}
$21$ et $33$.
\end{sol}
\end{exo}

\begin{exo}[Suites de nombres]
On considère les trois suites de nombres:
\[1,\quad 3,\quad 6,\quad 8,\quad 16,\quad 18, ...
\]
\[0,\quad 2,\quad 6,\quad 14,\quad 30,\quad 62, ...
\]
\[1,\quad 2,\quad 3,\quad 5,\quad 8,\quad 13, ...
\]
À chaque fois, quel devrait être le nombre suivant?
\begin{sol}
\begin{enumerate}
\item La suite semble être construite en itérant alternativement les opérations $+2$ et $\times 2$. Dans ce cas, le prochain nombre devrait être $36$.
\item La suite semble être construite en itérant le groupe d'opérations $+1$ et $\times 2$ à chaque étape (ou $\times 2$ et $+2$, c'est pareil). Dans ce cas, le prochain nombre devrait être $126$.
\item 
Chaque terme semble être la somme des deux précédents. Dans ce cas, le prochain nombre devrait être $21$.
\end{enumerate}
\end{sol}
\end{exo}

\begin{exo}[Les enfants du voisin]
% un classique
Mon voisin vient d'emménager. Il me dit avoir trois enfants dont les âges sont des nombres entiers.

Si on fait le produit de leurs âges, on trouvera $36$ et leur somme sera justement le numéro de la maison d'en face.

Intrigué, je rentre faire quelques petits calculs et m'aperçoit qu'il me manque une donnée.

Je sonne donc chez mon nouveau voisin. Effectivement me dit-il, j'avais oublié de vous dire que mon aîné(e) a les cheveux bouclés.

\begin{sol}
On fait la liste des diviseurs de $36$, et on calcule les sommes possibles de trois nombres dont le produit vaut trente-six.

S'il manque des données, c'est que cette somme à elle seule, qui est connue en théorie, ne permet pas de conclure. Ceci ne se produit que si les âges sont $1$, $6$ et $6$ ou bien $2$, $2$ et $9$.

Comme il y a un(e) ainé(e), on est forcément dans le deuxième cas.
\end{sol}
\end{exo}



\begin{exo}[Petit Poucet]
% source : Huffington Post, 10 problèmes de CM2
Le Petit Poucet s’amuse dans un escalier. Il a $55$ cailloux dans la poche de son pantalon.

Il vide sa poche en posant les cailloux de la manière suivante :

-Un caillou sur la première marche.

-Deux cailloux sur la deuxième marche.

-Trois cailloux sur la troisième marche.

Et ainsi de suite… Sur quelle marche pose-t-il le dernier caillou ?
\end{exo}

\begin{exo}[Parties dans un tournoi]
% source : Huffington Post, 10 problèmes de CM2
Une compétition de football réunit dix équipes. Chaque équipe dispute deux matchs contre chacune des neuf autres.

Combien de matchs vont être joués pendant cette compétition ?
\begin{sol}
Il y a $90$ matchs joués : chaque équipe joue $18$ matchs. Il y a dix équipes, mais multiplier le nombre de maths par équipes par le nombre d'équipes revient à compter chaque match deux fois.
\end{sol} 
\end{exo}



\begin{exo}[Dénombrement]
[Éliminatoires de la coupe Animath d'automne 2017]
% http://www.animath.fr/IMG/pdf/eliminatoires-3.pdf
Combien de façon y a-t-il d'insérer un ou plusieurs signes $+$ entre les caractères de l'expression $0123456789$ de telle sorte que l'expression garde un sens ?
\begin{sol}
Entre chacun des dix chiffres, c'est-à-dire à neuf emplacements possibles, on doit choisir si oui ou non on insère un signe $+$ ou pas. Pour chaque emplacement, on a donc deux choix, et donc au final, on a $2\times 2\times 2\times2\times2\times2\times2\times2\times 2 =512 (=2^9)$ choix. Mais parmi ces $512$ choix, il y a celui de ne rien insérer du tout, or l'énoncé est formulé de telle sorte qu'on doit au moins insérer un signe $+$. Il y a donc finalement $511$ choix.
\end{sol}
\end{exo}

\begin{exo}[Nénuphar]
(Traduit de : Arnold : \emph{77 problems for children 5 to 15})
% compétence : prendre le problème à l'envers, commencer par la fin

En Amérique du Sud, il y a un étang circulaire. Tous les ans, le premier juin, un nénuphar apparaît au centre du lac et commence à grandir. Toutes les $24$ heures, la superficie du lac couverte par le nénuphar double, et le premier juillet à midi, le lac est tout juste entièrement couvert par le nénuphar.

À quel moment le lac est-il recouvert exactement à la moitié par le nénuphar ?


\begin{sol}
Vingt-quatre heures avant qu'il ne soit totalement recouvert, c'est-à-dire le 30 juin à midi.
\end{sol}

\end{exo}

\begin{exo}[Même nombre d'amis]
% (cas particulier du principe des tiroirs)
Le jour de la rentrée, certains élèves d'une même classe se connaissent et d'autres non. (On suppose que le fait, pour deux élèves, de se connaître est symétrique : si Pierre connaît Paul alors Paul connaît Pierre.)

Montrer qu'il y a deux élèves de la classe qui connaissent chacun autant de monde.
\begin{sol}
(Note : la solution qui suit n'est pas adaptée à la 6ème car il y a du calcul littéral, mais on peut faire sans.) 

Notons $n$ le nombre de participants. Pour chaque entier $k$ compris entre $1$ et $n$, notons $a_n$ le nombre personnes que connaît le $n$-ème participant. Alors les $a_i$ sont $n$ entiers entre $0$ et $n-1$ et il s'agit de montrer que deux d'entre eux sont identiques. (Notons que le principe des tiroirs ne permet pas de conclure immédiatement, puisqu'il y a $n$ entiers à distribuer dans $n$ tiroirs.)

Supposons que tous les $a_i$ soient distincts. Alors ils valent forcément (dans le désordre) $0$, $1$, $2$, .. $n-1$. Ceci signifie qu'un participant connaît tout le monde, et qu'un autre ne connaît personne, ce qui est absurde. On en déduit que deux des entiers $a_i$ sont égaux.
\end{sol}
\end{exo}

\begin{exo}[Forcément consécutifs]
% car particulier du principe des tiroirs
On choisit onze nombres entiers compris entre $1$ et $20$. Montrer que deux d'entre eux sont consécutifs.

\begin{sol}
Groupons les entiers de $1$ à $20$ deux par deux : $1$ et $2$, puis $3$ et $4$, etc jusqu'à $19$ et $20$. Ceci donne dix \og lots\fg{} de deux entiers consécutifs.

Or on a choisi onze nombres entiers, donc comme il n'y a que dix lots, on a forcément choisi deux nombres dans un même lot au moins une fois, sinon on aurait choisi moins de dix nombres.
\end{sol}
\end{exo}



\begin{exo}[Carré un peu magique]
% cas particulier du principe des tiroirs
On remplit un tableau $3\times 3$  avec les nombres $-1$, $0$ et $1$, puis on calcule la somme des nombres dans chaque ligne, chaque colonne, et chacune des deux diagonales. Montrer que parmi les sommes obtenues, il y en a deux qui sont égales.
\begin{sol}
Comme chaque coefficient est compris entre $-1$ et $1$, la somme de trois coefficients est comprise entre $-3$ et $3$, ce qui fait sept valeurs entières possibles.

D'autre part il y a huit sommes à calculer (les trois lignes, les trois colonnes et les deux diagonales).

Il y a donc au moins deux des huit sommes qui sont identiques. (C'est le principe des tiroirs, mais ici le résultat est intuitif.)
\end{sol}
\end{exo}



\begin{exo}[Même anniversaire]
% principe des tiroirs
\begin{enumerate}
\item
Le lycée Dirichlet compte $400$ élèves. Montrer qu'il existe (au moins) deux élèves qui fêtent leur anniversaire le même jour.

\item Même avec un peu moins d'élèves, on aurait pu avoir la même conclusion. Quel est le nombre minimal d'élèves à partir duquel on peut obtenir la même conclusion ?

\item À partir de combien d'élèves dans un lycée peut-on affirmer qu'il en existe  (au moins) quatre avec la même date d'anniversaire ?
\end{enumerate}
\begin{sol}
\begin{enumerate}
\item
Comme il y a plus d'élèves que de jours dans l'année, il y a au moins deux élèves qui fêtent leur anniversaire le même jour.

\item À partir de $367$ élèves, on peut conclure de la même manière.

\item S'il y a $3\times 366=1098$ élèves, il est possible qu'exactement trois d'entre eux fêtent leur anniversaire chaque jour. À partir de $3\times 366+1=1099$ élèves, il y en a forcément quatre qui fêtent leur anniversaire le même jour.
\end{enumerate}
\end{sol}
\end{exo}

\subsection{Symétrie axiale, bissectrices (sans cercle inscrit), médiatrices, triangles isocèles}

% sans médiatrices concourantes : fait en cinquième
% Mais on peut le faire démontrer en sixième, ou des exercices semblables.


\begin{exo}[Construction du centre]
% tags : corde, médiatrice, symétrique, milieu
On donne un cercle $\mathcal C$ (sans son centre).
Tracer son centre, si possible de plusieurs façons. 
\begin{hint}
Utiliser une ou plusieurs cordes.
\end{hint}
\begin{sol}
On trace une corde et sa médiatrice, qui doit contenir le centre du cercle.

Ensuite, soit on recommence avec une autre corde, soit, puisque la première médiatrice fournit un diamètre, on construit la médiatrice de ce diamètre.
\end{sol}
\end{exo}  

\begin{exo}[Cerf-volant]
%Sesamath 2016, classe de sixième (p. 96)
Un quadrilatère non croisé est un \emph{cerf-volant} si $AB=BC$ et $CD=DA$. Montrer que les deux diagonales d'un cerf-volant se croisent à angle droit au milieu de $[AC]$.
\begin{sol}
Comme $BA=BC$, $B$ est sur la médiatrice de $[AC]$.

De même, $D$ est sur la médiatrice de $[AC]$.

Donc la droite $(BD)$ est la médiatrice de $[AC]$, ce dont on déduit que ces deux droites se croisent à angle droit au milieu de $[AC]$.
\end{sol}
\end{exo}

\begin{exo}[Losange]
Un quadrilatère non croisé est un \emph{losange} si ses quatre côtés ont la même longueur. Montrer que les deux diagonales d'un losange se croisent à angle droit et en leur milieu.
\begin{sol}
Comme dans le cas du cerf-volant, les égalités $BA=BC$ et $DA=DC$ montrent que $B$ et $D$ sont sur la médiatrice de $[AC]$ et donc que la droite $(BD)$ est la médiatrice de $[AC]$.

On montre de la même manière que la droite $(AC)$ est la médiatrice du segment $[DB]$.
\end{sol}
\end{exo}





%%%%%%%%%%%%%%%%%%%%%%%%%%%%%%%%%%%
\section{Classe de 5ème}
%%%%%%%%%%%%%%%%%%%%%%%%%%%%%%%%%%%



\subsection{Cercle circonscrit, triangles isocèles}



\begin{exo}[Deux réflexions]
% def équiv de médiatrice, complexité 2

Soient $\mathcal D$ et $\mathcal D'$ deux droites sécantes en un point $O$, et $A$ un point hors de ces droites.

On appelle $B$ et $B'$ les symétriques de $A$ par rapport aux droites $\mathcal D$ et $\mathcal D'$. Montrer que la médiatrice du segment $[BB']$ contient le point $O$.
\begin{center}
\definecolor{qqqqff}{rgb}{0.,0.,1.}
\begin{tikzpicture}[line cap=round,line join=round,>=triangle 45,x=1.0cm,y=1.0cm]
\clip(-1.88,-0.6) rectangle (3.54,4.6);
\draw [domain=-1.88:3.54] plot(\x,{(--4.9876-3.42*\x)/2.});
\draw [domain=-1.88:3.54] plot(\x,{(-0.7096--3.2*\x)/1.54});
\draw [dash pattern=on 2pt off 2pt] (-1.218131546087001,0.586004943808772)-- (2.26,2.62);
\draw [dash pattern=on 2pt off 2pt] (2.26,2.62)-- (0.9976193345808624,3.22752069523296);
\begin{scriptsize}
\draw [fill=qqqqff] (0.78,1.16) circle (2.5pt);
\draw[color=qqqqff] (0.72,0.63) node {$O$};
\draw [fill=qqqqff] (2.26,2.62) circle (2.5pt);
\draw[color=qqqqff] (2.4,2.99) node {$A$};
\draw[color=black] (-0.68,3.51) node {$\mathcal D$};
\draw[color=black] (3.14,3.97) node {$\mathcal D'$};
\draw [fill=qqqqff] (-1.218131546087001,0.586004943808772) circle (2.5pt);
\draw[color=qqqqff] (-1.08,0.95) node {$B$};
\draw [fill=qqqqff] (0.9976193345808624,3.22752069523296) circle (2.5pt);
\draw[color=qqqqff] (1.2,3.59) node {$B'$};
\end{scriptsize}
\end{tikzpicture}
\end{center}
\begin{sol}
L'exercice ressemble comme deux gouttes d'eau à la preuve que les médiatrices sont concourantes, mais rédigé avec des réflexions.

Comme $B$ est le symétrique de $A$ par rapport à $\mathcal D$, la droite $\mathcal D$ est la médiatrice de $[AB]$. On en déduit que pour tout point $P$ de $\mathcal D$, on a $PA=PB$. En particulier, comme $O$ est sur $\mathcal D$, on a
\[ OA=OB.\]

On prouve de la même manière que 
\[OA=OB'.\]
On en déduit donc  l'égalité de distances
\[OB=OB',\]
ce qui montre que \fbox{$O$ est sur la médiatrice de $[BB']$.}
\end{sol}
\end{exo}


\begin{exo}[Cordes concourantes]
% Lebossé-Hemery 5ème, exercice 169, leçon 11 : médiatrices
Soient $\mathcal C$, $\mathcal C'$  et $\mathcal C''$ trois cercles de même rayon, sécants deux à deux et dont on note $A$, $B$ et $C$ les centres.

Montrer que les trois sécantes communes (aux trois paires de cercles) sont trois droites concourantes.
\begin{center}
\definecolor{qqqqff}{rgb}{0.,0.,1.}
\begin{tikzpicture}[line cap=round,line join=round,>=triangle 45,x=1.0cm,y=1.0cm]
\clip(-3.28,-1.42) rectangle (4.9,5.);
\draw(-0.42,2.74) circle (2.cm);
\draw(2.12,2.7) circle (2.cm);
\draw(-0.88,1.24) circle (2.cm);
\draw [dash pattern=on 2pt off 2pt] (-2.408879926554141,2.52938984414327)-- (1.1088799265541405,1.4506101558567304);
\draw [dash pattern=on 2pt off 2pt] (0.13723342784583936,2.9619861071660845)-- (1.1027665721541613,0.9780138928339162);
\draw [dash pattern=on 2pt off 2pt] (0.8256739639888225,1.1752967132902243)-- (0.8743260360111778,4.264703286709777);
\begin{scriptsize}
\draw [fill=qqqqff] (-0.42,2.74) circle (2.5pt);
\draw[color=qqqqff] (-0.82,3.05) node {$A$};
\draw [fill=qqqqff] (2.12,2.7) circle (2.5pt);
\draw[color=qqqqff] (2.54,2.73) node {$B$};
\draw [fill=qqqqff] (-0.88,1.24) circle (2.5pt);
\draw[color=qqqqff] (-0.58,1.63) node {$C$};
\end{scriptsize}
\end{tikzpicture}
\end{center}
\begin{sol}
Les cercles ont même rayon, donc les cordes communes sont les médiatrices des segments reliant les centres des cercles. Ces médiatrices sont donc concourantes.
\end{sol}
\end{exo}

\begin{exo}[Triangles inscrits dans le même cercle]
% def equivalentes des médiatrices, def d'un cercle, symétries

Soit $ABC$ un triangle, $\mathcal D$ la médiatrice du segment $[AB]$ et $C'$ le symétrique de $C$ par rapport à $\mathcal D$. Montrer que les deux triangles $ABC$ et $ABC'$ ont le même cercle circonscrit.

\begin{center}
\definecolor{qqwuqq}{rgb}{0.,0.39215686274509803,0.}
\definecolor{qqqqff}{rgb}{0.,0.,1.}
\begin{tikzpicture}[line cap=round,line join=round,>=triangle 45,x=1.0cm,y=1.0cm]
\clip(-2.72,-1.62) rectangle (5.42,4.92);
\draw[color=qqwuqq,fill=qqwuqq,fill opacity=0.10000000149011612] (2.2194272122991046,3.40417457639394) -- (2.285994396756149,3.67907239368877) -- (2.0110965794613187,3.7456395781458145) -- (1.9445293950042741,3.4707417608509843) -- cycle; 
\draw[color=qqwuqq,fill=qqwuqq,fill opacity=0.10000000149011612] (1.5048978172948304,0.4534328155429559) -- (1.5714650017518745,0.7283306328377863) -- (1.296567184457044,0.7948978172948304) -- (1.23,0.52) -- cycle; 
\draw [line width=1.6pt,dash pattern=on 2pt off 2pt,domain=-2.72:5.42] plot(\x,{(-4.9242--4.46*\x)/1.08});
\draw (2.98,3.22)-- (3.46,-0.02);
\draw (2.98,3.22)-- (-1.,1.06);
\draw (0.9090587900085476,3.721483521701966)-- (1.9445293950042746,3.4707417608509847);
\draw (1.4479756408967428,3.683584591110625) -- (1.405612544116079,3.5086406914423254);
\draw (1.9445293950042746,3.4707417608509847)-- (2.98,3.22);
\draw (2.483446245892468,3.4328428302596423) -- (2.441083149111804,3.257898930591342);
\draw (-1.,1.06)-- (1.23,0.52);
\draw (0.10216467901038476,0.8857092186526129) -- (0.059801582229720825,0.7107653189843128);
\draw (0.1701984177702793,0.8692346810156878) -- (0.12783532098961536,0.6942907813473878);
\draw (1.23,0.52)-- (3.46,-0.02);
\draw (2.3321646790103854,0.34570921865261267) -- (2.28980158222972,0.17076531898431263);
\draw (2.4001984177702798,0.3292346810156876) -- (2.357835320989615,0.1542907813473875);
\begin{scriptsize}
\draw [fill=qqqqff] (-1.,1.06) circle (2.5pt);
\draw[color=qqqqff] (-1.44,1.33) node {$A$};
\draw [fill=qqqqff] (3.46,-0.02) circle (2.5pt);
\draw[color=qqqqff] (3.6,0.35) node {$B$};
\draw [fill=qqqqff] (2.98,3.22) circle (2.5pt);
\draw[color=qqqqff] (3.12,3.59) node {$C$};
\draw [fill=qqqqff] (0.9090587900085476,3.721483521701966) circle (2.5pt);
\draw[color=qqqqff] (1.1,4.09) node {$C'$};
\end{scriptsize}
\end{tikzpicture}
\end{center}

\begin{sol}
Il s'agit de montrer que $C'$ appartient au cercle circonscrit de $ABC$.

Comme $C$ et $C'$ sont symétriques par rapport à $\mathcal D$, tout point $P$ de cette droite vérifie $PC = PC'$.

En particulier, le centre $O$, du cercle circonscrit à $ABC$, qui est l'intersection des trois médiatrices, vérifie $OC=OC'$.

On en déduit que $C'$ est sur le cercle de centre $O$ et de rayon $OC$. Ce cercle est le cercle circonscrit à $ABC$, ce qu'il fallait démontrer.

\begin{center}
\definecolor{uuuuuu}{rgb}{0.26666666666666666,0.26666666666666666,0.26666666666666666}
\definecolor{qqwuqq}{rgb}{0.,0.39215686274509803,0.}
\definecolor{qqqqff}{rgb}{0.,0.,1.}
\begin{tikzpicture}[line cap=round,line join=round,>=triangle 45,x=1.0cm,y=1.0cm]
\clip(-2.72,-1.62) rectangle (5.42,4.92);
\draw[color=qqwuqq,fill=qqwuqq,fill opacity=0.10000000149011612] (2.2194272122991046,3.40417457639394) -- (2.285994396756149,3.67907239368877) -- (2.0110965794613187,3.7456395781458145) -- (1.9445293950042741,3.4707417608509843) -- cycle; 
\draw[color=qqwuqq,fill=qqwuqq,fill opacity=0.10000000149011612] (1.5048978172948304,0.4534328155429559) -- (1.5714650017518745,0.7283306328377863) -- (1.296567184457044,0.7948978172948304) -- (1.23,0.52) -- cycle; 
\draw [line width=1.6pt,dash pattern=on 2pt off 2pt,domain=-2.72:5.42] plot(\x,{(-4.9242--4.46*\x)/1.08});
\draw (2.98,3.22)-- (3.46,-0.02);
\draw (2.98,3.22)-- (-1.,1.06);
\draw (0.9090587900085476,3.721483521701966)-- (1.9445293950042746,3.4707417608509847);
\draw (1.4479756408967428,3.683584591110625) -- (1.405612544116079,3.5086406914423254);
\draw (1.9445293950042746,3.4707417608509847)-- (2.98,3.22);
\draw (2.483446245892468,3.4328428302596423) -- (2.441083149111804,3.257898930591342);
\draw (-1.,1.06)-- (1.23,0.52);
\draw (0.10216467901038476,0.8857092186526129) -- (0.059801582229720825,0.7107653189843128);
\draw (0.1701984177702793,0.8692346810156878) -- (0.12783532098961536,0.6942907813473878);
\draw (1.23,0.52)-- (3.46,-0.02);
\draw (2.3321646790103854,0.34570921865261267) -- (2.28980158222972,0.17076531898431263);
\draw (2.4001984177702798,0.3292346810156876) -- (2.357835320989615,0.1542907813473875);
\draw(1.4272093023255812,1.3344013781223087) circle (2.442670897524113cm);
\draw (1.4272093023255812,1.3344013781223087)-- (2.98,3.22);
\draw (2.089631366800829,2.2803772362035155) -- (2.228580806236984,2.165952355690673);
\draw (2.1341299314447126,2.3344131293175754) -- (2.273079370880867,2.2199882488047327);
\draw (2.178628496088596,2.3884490224316353) -- (2.3175779355247506,2.2740241419187925);
\draw (1.4272093023255812,1.3344013781223087)-- (0.9090587900085476,3.721483521701966);
\draw (1.095030928902691,2.4404442542379905) -- (1.2709346031470836,2.4786266767295353);
\draw (1.0801822090448685,2.508851238666365) -- (1.256085883289261,2.5470336611579096);
\draw (1.0653334891870458,2.5772582230947396) -- (1.2412371634314383,2.6154406455862844);
\draw (1.4272093023255812,1.3344013781223087)-- (3.46,-0.02);
\draw (2.435253454598866,0.7709119792479728) -- (2.335447849058994,0.6211159809609037);
\draw (2.4935074539327253,0.7320986882046893) -- (2.3937018483928543,0.5823026899176201);
\draw (2.551761453266586,0.6932853971614057) -- (2.4519558477267154,0.5434893988743364);
\draw (1.4272093023255812,1.3344013781223087)-- (-1.,1.06);
\draw (0.29327186109722975,1.1156339332027652) -- (0.27305127054936706,1.2944945709012141);
\draw (0.2237149464367219,1.1077703702119297) -- (0.20349435588885922,1.2866310079103789);
\draw (0.15415803177621407,1.0999068072210945) -- (0.13393744122835138,1.2787674449195434);
\begin{scriptsize}
\draw [fill=qqqqff] (-1.,1.06) circle (2.5pt);
\draw[color=qqqqff] (-1.44,1.33) node {$A$};
\draw [fill=qqqqff] (3.46,-0.02) circle (2.5pt);
\draw[color=qqqqff] (3.6,0.35) node {$B$};
\draw [fill=qqqqff] (2.98,3.22) circle (2.5pt);
\draw[color=qqqqff] (3.12,3.59) node {$C$};
\draw [fill=qqqqff] (0.9090587900085476,3.721483521701966) circle (2.5pt);
\draw[color=qqqqff] (1.1,4.09) node {$C'$};
\draw [fill=uuuuuu] (1.4272093023255812,1.3344013781223087) circle (1.5pt);
\draw[color=uuuuuu] (1.02,1.65) node {$O$};
\end{scriptsize}
\end{tikzpicture}
\end{center}

Note : on peut aussi poser cet exercice de la façon suivante \og Montrer qu'un trapèze est inscriptible si et seulement s'il est isocèle\fg. En général, on montre ce résultat à l'aide du théorème de l'angle inscrit mais la preuve élémentaire ci-dessus marche aussi et n'est pas très longue.

\end{sol}
\end{exo}

\begin{exo}[Deux réflexions $\heartsuit$]
% Lebossé-Hemery 5ème, p. 204 
% triangles isocèles, bissectrices

Soit $A$ un point intérieur à un angle $xOy$. On note $B$ et $C$ les symétriques de $A$ par rapport aux bords de l'angle. Le segment $[BC]$ coupe les bords de l'angle en $D$ et $E$. Montrer que $[AO)$ est la bissectrice de l'angle $\widehat{DAE}$.

\begin{center}
\definecolor{uuuuuu}{rgb}{0.26666666666666666,0.26666666666666666,0.26666666666666666}
\definecolor{qqqqff}{rgb}{0.,0.,1.}
\begin{tikzpicture}[line cap=round,line join=round,>=triangle 45,x=1.0cm,y=1.0cm]
\clip(-2.72,-1.06) rectangle (4.4,4.5);
\draw [line width=1.6pt] (-1.,-0.56)-- (0.24,4.78);
\draw [line width=1.6pt] (-1.,-0.56)-- (3.48,2.02);
\draw (-2.2933658978078877,3.840182343311194)-- (3.479331008575662,0.4248825897600912);
\draw [domain=-2.72:4.4] plot(\x,{(--1.644--3.38*\x)/3.1});
\draw (-0.2578750441772161,2.635925212978763)-- (2.1,2.82);
\draw (2.1,2.82)-- (2.113419968703404,1.2329963212622284);
\draw [dash pattern=on 2pt off 2pt] (2.1,2.82)-- (-2.2933658978078877,3.840182343311194);
\draw [dash pattern=on 2pt off 2pt] (2.1,2.82)-- (3.479331008575662,0.4248825897600912);
\begin{scriptsize}
\draw [fill=qqqqff] (-1.,-0.56) circle (2.5pt);
\draw[color=qqqqff] (-1.5,-0.25) node {$O$};
\draw [fill=qqqqff] (2.1,2.82) circle (2.5pt);
\draw[color=qqqqff] (2.08,3.39) node {$A$};
\draw [fill=qqqqff] (-2.2933658978078877,3.840182343311194) circle (2.5pt);
\draw[color=qqqqff] (-2.16,4.21) node {$B$};
\draw [fill=qqqqff] (3.479331008575662,0.4248825897600912) circle (2.5pt);
\draw[color=qqqqff] (3.62,0.79) node {$C$};
\draw [fill=uuuuuu] (-0.2578750441772161,2.635925212978763) circle (1.5pt);
\draw[color=uuuuuu] (-0.9,2.49) node {$E$};
\draw [fill=uuuuuu] (2.113419968703404,1.2329963212622284) circle (1.5pt);
\draw[color=uuuuuu] (2.04,0.83) node {$D$};
\end{scriptsize}
\end{tikzpicture}
\end{center}
\begin{hint}
Utiliser un triangle isocèle.
\end{hint}
\begin{sol}
Bel exercice qui demande d'utiliser deux caractérisations des triangles isocèles, ainsi que de la médiatrice.

\begin{center}
\definecolor{qqwuqq}{rgb}{0.,0.39215686274509803,0.}
\definecolor{ccqqqq}{rgb}{0.8,0.,0.}
\definecolor{uuuuuu}{rgb}{0.26666666666666666,0.26666666666666666,0.26666666666666666}
\definecolor{qqqqff}{rgb}{0.,0.,1.}
\begin{tikzpicture}[line cap=round,line join=round,>=triangle 45,x=1.0cm,y=1.0cm]
\clip(-2.72,-1.06) rectangle (4.4,4.5);
\draw [shift={(-2.2933658978078877,3.840182343311194)},color=qqwuqq,fill=qqwuqq,fill opacity=0.10000000149011612] (0,0) -- (-73.62011222157199:0.6) arc (-73.62011222157199:-30.60982553724811:0.6) -- cycle;
\draw [shift={(2.1,2.82)},color=qqwuqq,fill=qqwuqq,fill opacity=0.10000000149011612] (0,0) -- (-175.53608221615343:0.6) arc (-175.53608221615343:-132.52579553182954:0.6) -- cycle;
\draw [shift={(2.1,2.82)},color=qqwuqq,fill=qqwuqq,fill opacity=0.10000000149011612] (0,0) -- (-132.52579553182954:0.6) arc (-132.52579553182954:-89.51550884750567:0.6) -- cycle;
\draw [shift={(3.479331008575662,0.4248825897600912)},color=qqwuqq,fill=qqwuqq,fill opacity=0.10000000149011612] (0,0) -- (149.3901744627519:0.6) arc (149.3901744627519:192.40046114707576:0.6) -- cycle;
\draw [line width=1.6pt] (-1.,-0.56)-- (0.24,4.78);
\draw [line width=1.6pt] (-1.,-0.56)-- (3.48,2.02);
\draw (-2.2933658978078877,3.840182343311194)-- (3.479331008575662,0.4248825897600912);
\draw [domain=-2.72:4.4] plot(\x,{(--1.644--3.38*\x)/3.1});
\draw (-0.2578750441772161,2.635925212978763)-- (2.1,2.82);
\draw (2.1,2.82)-- (2.113419968703404,1.2329963212622284);
\draw [dash pattern=on 2pt off 2pt] (2.1,2.82)-- (-2.2933658978078877,3.840182343311194);
\draw [dash pattern=on 2pt off 2pt] (2.1,2.82)-- (3.479331008575662,0.4248825897600912);
\draw [line width=1.6pt,color=ccqqqq] (-1.,-0.56)-- (-2.2933658978078877,3.840182343311194);
\draw [line width=1.6pt,color=ccqqqq] (-1.7395281702117957,1.5658970689229936) -- (-1.5284573048877046,1.6279381022101624);
\draw [line width=1.6pt,color=ccqqqq] (-1.7649085929201827,1.6522442411010312) -- (-1.5538377275960917,1.7142852743882);
\draw [line width=1.6pt,color=ccqqqq] (-1.,-0.56)-- (3.479331008575662,0.4248825897600912);
\draw [line width=1.6pt,color=ccqqqq] (1.1720935788644282,0.030211611922458685) -- (1.2193370802114338,-0.18465590907705226);
\draw [line width=1.6pt,color=ccqqqq] (1.2599939283642287,0.049538498837143244) -- (1.3072374297112344,-0.1653290221623677);
\draw [shift={(-2.2933658978078877,3.840182343311194)},color=qqwuqq] (-73.62011222157199:0.6) arc (-73.62011222157199:-30.60982553724811:0.6);
\draw[color=qqwuqq] (-1.9617632237660787,3.4139902896917933) -- (-1.888073740645677,3.319280944443037);
\draw [shift={(2.1,2.82)},color=qqwuqq] (-175.53608221615343:0.6) arc (-175.53608221615343:-132.52579553182954:0.6);
\draw[color=qqwuqq] (1.6145234602374336,2.583541696402589) -- (1.506639784734641,2.5309954067142746);
\draw [shift={(2.1,2.82)},color=qqwuqq] (-132.52579553182954:0.6) arc (-132.52579553182954:-89.51550884750567:0.6);
\draw[color=qqwuqq] (1.9062996055774284,2.315936356002002) -- (1.8632550734835236,2.2039222128913356);
\draw [shift={(3.479331008575662,0.4248825897600912)},color=qqwuqq] (149.3901744627519:0.6) arc (149.3901744627519:192.40046114707576:0.6);
\draw[color=qqwuqq] (2.946134534012404,0.5103315190296419) -- (2.8276464285539014,0.5293201699784313);
\draw [line width=1.6pt,color=ccqqqq] (-1.,-0.56)-- (2.1,2.82);
\draw [line width=1.6pt,color=ccqqqq] (0.4385164656825278,1.1711876386320568) -- (0.6006505476666322,1.022484782374446);
\draw [line width=1.6pt,color=ccqqqq] (0.49934945233336825,1.237515217625554) -- (0.6614835343174726,1.0888123613679432);
\begin{scriptsize}
\draw [fill=qqqqff] (-1.,-0.56) circle (2.5pt);
\draw[color=qqqqff] (-1.5,-0.25) node {$O$};
\draw [fill=qqqqff] (2.1,2.82) circle (2.5pt);
\draw[color=qqqqff] (2.08,3.39) node {$A$};
\draw [fill=qqqqff] (-2.2933658978078877,3.840182343311194) circle (2.5pt);
\draw[color=qqqqff] (-2.16,4.21) node {$B$};
\draw [fill=qqqqff] (3.479331008575662,0.4248825897600912) circle (2.5pt);
\draw[color=qqqqff] (3.62,0.79) node {$C$};
\draw [fill=uuuuuu] (-0.2578750441772161,2.635925212978763) circle (1.5pt);
\draw[color=uuuuuu] (-0.9,2.49) node {$E$};
\draw [fill=uuuuuu] (2.113419968703404,1.2329963212622284) circle (1.5pt);
\draw[color=uuuuuu] (2.04,0.83) node {$D$};
\end{scriptsize}
\end{tikzpicture}
\end{center}

Comme $A$ et $B$ sont symétriques l'un de l'autre par rapport au bord de l'angle, on a $OA=OB$. On a de même $OA = OC$, donc \uline{$OB=OC$, et donc $OBC$ est isocèle.}

Par l'autre caractérisation des triangles isocèles, on en déduit \uline{$\widehat{OBC}=\widehat{OCB}$.}

Or, comme les symétries axiales préservent les angles (non orientés), on a $\widehat{OBC}=\widehat{OBE}=\widehat{OAE}$, et d'autre part, $\widehat{OCB}=\widehat{OCD}=\widehat{OAD}$.

Finalement, \fbox{$\widehat{OAE}=\widehat{OAD}$, et donc $[AO)$ est la bissectrice de $\widehat{EAD}$.}
\end{sol}
\end{exo}



\subsection{Aires et périmètres}



\begin{exo}[Aire d'un quadrilatère orthodiagonal]
%  aire, hauteur,
Un quadrilatère est dit \emph{orthodiagonal} si ses diagonales sont perpendiculaires.

Soit $ABCD$ un quadrilatère orthodiagonal non croisé. Montrer que son aire vaut $\frac12 AC\cdot BD$. 


\begin{hint}   
Décomposer l'aire comme la somme des aires de deux triangles.
\end{hint} 
\end{exo} 

\begin{exo}[Égalité d'aires]
(Complexité : 2)

Soit $ABCD$ un parallélogramme, $P$ un point de $[AB]$ et $Q$ un point de $[BC]$. Montrer que les triangles $CDP$ et $DAQ$ ont la même aire.
\begin{center}
\definecolor{xdxdff}{rgb}{0.49019607843137253,0.49019607843137253,1.}
\definecolor{uuuuuu}{rgb}{0.26666666666666666,0.26666666666666666,0.26666666666666666}
\definecolor{qqqqff}{rgb}{0.,0.,1.}
\begin{tikzpicture}[line cap=round,line join=round,>=triangle 45,x=1.0cm,y=1.0cm]
\clip(-2.52,0.16) rectangle (4.14,3.38);
\draw (-1.8,0.96)-- (2.6,1.);
\draw (2.6,1.)-- (3.4,2.76);
\draw (-1.8,0.96)-- (-1.,2.72);
\draw (-1.,2.72)-- (3.4,2.76);
\draw (-1.,2.72)-- (-0.12013883150152879,0.9752714651681678);
\draw (3.4,2.76)-- (-0.12013883150152879,0.9752714651681678);
\draw (3.0993150684931505,2.0984931506849316)-- (-1.,2.72);
\draw (3.0993150684931505,2.0984931506849316)-- (-1.8,0.96);
\begin{scriptsize}
\draw [fill=qqqqff] (-1.8,0.96) circle (2.5pt);
\draw[color=qqqqff] (-2.32,1.19) node {$A$};
\draw [fill=qqqqff] (2.6,1.) circle (2.5pt);
\draw[color=qqqqff] (2.86,0.79) node {$B$};
\draw [fill=qqqqff] (3.4,2.76) circle (2.5pt);
\draw[color=qqqqff] (3.54,3.13) node {$C$};
\draw [fill=uuuuuu] (-1.,2.72) circle (1.5pt);
\draw[color=uuuuuu] (-1.38,3.07) node {$D$};
\draw [fill=xdxdff] (-0.12013883150152879,0.9752714651681678) circle (2.5pt);
\draw[color=xdxdff] (-0.04,0.51) node {$P$};
\draw [fill=xdxdff] (3.0993150684931505,2.0984931506849316) circle (2.5pt);
\draw[color=xdxdff] (3.36,1.87) node {$Q$};
\end{scriptsize}
\end{tikzpicture}
\end{center}
\begin{sol}
C'est la moitié de l'aire du parallélogramme.
\end{sol}
\end{exo}

\begin{exo}[Théorème du papillon]
Soit $ABCD$ un trapèze de bases $[AB]$ et $[CD]$, et $O$ l'intersection de ses diagonales. Montrer que les triangles $OBC$ et $ODA$ on la même aire.
\begin{center}
\definecolor{uuuuuu}{rgb}{0.26666666666666666,0.26666666666666666,0.26666666666666666}
\definecolor{xdxdff}{rgb}{0.49019607843137253,0.49019607843137253,1.}
\definecolor{qqqqff}{rgb}{0.,0.,1.}
\begin{tikzpicture}[line cap=round,line join=round,>=triangle 45,x=1.0cm,y=1.0cm]
\clip(-1.1,-0.96) rectangle (4.8,2.76);
\draw (-0.44,-0.14)-- (3.82,-0.14);
\draw (-0.44,-0.14)-- (-0.18,2.06);
\draw (-0.18,2.06)-- (2.12,2.06);
\draw (2.12,2.06)-- (3.82,-0.14);
\draw (-0.44,-0.14)-- (2.12,2.06);
\draw (-0.18,2.06)-- (3.82,-0.14);
\begin{scriptsize}
\draw [fill=qqqqff] (-0.44,-0.14) circle (2.5pt);
\draw[color=qqqqff] (-0.8,-0.31) node {$A$};
\draw [fill=qqqqff] (3.82,-0.14) circle (2.5pt);
\draw[color=qqqqff] (4.14,-0.31) node {$B$};
\draw [fill=qqqqff] (2.12,2.06) circle (2.5pt);
\draw[color=qqqqff] (2.26,2.43) node {$C$};
\draw [fill=xdxdff] (-0.18,2.06) circle (2.5pt);
\draw[color=xdxdff] (-0.6,2.45) node {$D$};
\draw [fill=uuuuuu] (1.2224390243902434,1.2886585365853658) circle (1.5pt);
\draw[color=uuuuuu] (1.18,0.93) node {$O$};
\end{scriptsize}
\end{tikzpicture}
\end{center}

\end{exo}
 
\begin{exo}[Triangle dans un pentagone]
% Debart
Soit $ABCD$ un rectangle, $E$, $F$, $G$ et $H$ les milieux de ses côtés, et $I$ le milieu de $[GF]$. Que vaut l'aire du triangle $HEI$, en fonction de celle du rectangle ?
\begin{center}
\definecolor{zzttqq}{rgb}{0.6,0.2,0.}
\definecolor{uuuuuu}{rgb}{0.26666666666666666,0.26666666666666666,0.26666666666666666}
\definecolor{xdxdff}{rgb}{0.49019607843137253,0.49019607843137253,1.}
\definecolor{qqqqff}{rgb}{0.,0.,1.}
\begin{tikzpicture}[line cap=round,line join=round,>=triangle 45,x=1.0cm,y=1.0cm]
\clip(-2.12,-0.96) rectangle (4.54,3.32);
\fill[color=zzttqq,fill=zzttqq,fill opacity=0.10000000149011612] (-1.56,1.31) -- (1.04,0.) -- (2.34,1.965) -- cycle;
\draw (-1.56,0.)-- (3.64,0.);
\draw (-1.56,2.62)-- (3.64,2.62);
\draw (-1.56,0.)-- (-1.56,2.62);
\draw (3.64,0.)-- (3.64,2.62);
\draw (1.04,2.62)-- (3.64,1.31);
\draw [color=zzttqq] (-1.56,1.31)-- (1.04,0.);
\draw [color=zzttqq] (1.04,0.)-- (2.34,1.965);
\draw [color=zzttqq] (2.34,1.965)-- (-1.56,1.31);
\begin{scriptsize}
\draw [fill=qqqqff] (-1.56,0.) circle (2.5pt);
\draw[color=qqqqff] (-1.9,-0.17) node {$A$};
\draw [fill=qqqqff] (3.64,0.) circle (2.5pt);
\draw[color=qqqqff] (3.92,-0.01) node {$B$};
\draw [fill=xdxdff] (3.64,2.62) circle (2.5pt);
\draw[color=xdxdff] (3.78,2.99) node {$C$};
\draw [fill=uuuuuu] (-1.56,2.62) circle (1.5pt);
\draw[color=uuuuuu] (-1.82,2.97) node {$D$};
\draw [fill=uuuuuu] (1.04,0.) circle (1.5pt);
\draw[color=uuuuuu] (1.04,-0.35) node {$E$};
\draw [fill=uuuuuu] (3.64,1.31) circle (1.5pt);
\draw[color=uuuuuu] (3.78,1.59) node {$F$};
\draw [fill=uuuuuu] (1.04,2.62) circle (1.5pt);
\draw[color=uuuuuu] (1.18,2.91) node {$G$};
\draw [fill=uuuuuu] (-1.56,1.31) circle (1.5pt);
\draw[color=uuuuuu] (-1.94,1.55) node {$H$};
\draw [fill=uuuuuu] (2.34,1.965) circle (1.5pt);
\draw[color=uuuuuu] (2.48,2.25) node {$I$};
\end{scriptsize}
\end{tikzpicture}
\end{center}

\begin{sol}
On peut découper en plusieurs petits triangles, ou bien voir que $(GF)//(HE)$. On peut alors calculer l'aire de la façon suivante:
\begin{center}
\definecolor{zzttqq}{rgb}{0.6,0.2,0.}
\definecolor{uuuuuu}{rgb}{0.26666666666666666,0.26666666666666666,0.26666666666666666}
\definecolor{xdxdff}{rgb}{0.49019607843137253,0.49019607843137253,1.}
\definecolor{qqqqff}{rgb}{0.,0.,1.}
\begin{tikzpicture}[line cap=round,line join=round,>=triangle 45,x=1.0cm,y=1.0cm]
\clip(-2.12,-0.96) rectangle (4.54,3.32);
\fill[color=zzttqq,fill=zzttqq,fill opacity=0.10000000149011612] (-1.56,1.31) -- (1.04,0.) -- (2.34,1.965) -- cycle;
\fill[color=zzttqq,fill=zzttqq,fill opacity=0.10000000149011612] (-1.56,1.31) -- (1.04,0.) -- (1.04,2.62) -- cycle;
\draw (-1.56,0.)-- (3.64,0.);
\draw (-1.56,2.62)-- (3.64,2.62);
\draw (-1.56,0.)-- (-1.56,2.62);
\draw (3.64,0.)-- (3.64,2.62);
\draw (1.04,2.62)-- (3.64,1.31);
\draw [color=zzttqq] (-1.56,1.31)-- (1.04,0.);
\draw [color=zzttqq] (1.04,0.)-- (2.34,1.965);
\draw [color=zzttqq] (2.34,1.965)-- (-1.56,1.31);
\draw [color=zzttqq] (-1.56,1.31)-- (1.04,0.);
\draw [color=zzttqq] (1.04,0.)-- (1.04,2.62);
\draw [color=zzttqq] (1.04,2.62)-- (-1.56,1.31);
\begin{scriptsize}
\draw [fill=qqqqff] (-1.56,0.) circle (2.5pt);
\draw[color=qqqqff] (-1.9,-0.17) node {$A$};
\draw [fill=qqqqff] (3.64,0.) circle (2.5pt);
\draw[color=qqqqff] (3.92,-0.01) node {$B$};
\draw [fill=xdxdff] (3.64,2.62) circle (2.5pt);
\draw[color=xdxdff] (3.78,2.99) node {$C$};
\draw [fill=uuuuuu] (-1.56,2.62) circle (1.5pt);
\draw[color=uuuuuu] (-1.82,2.97) node {$D$};
\draw [fill=uuuuuu] (1.04,0.) circle (1.5pt);
\draw[color=uuuuuu] (1.04,-0.35) node {$E$};
\draw [fill=uuuuuu] (3.64,1.31) circle (1.5pt);
\draw[color=uuuuuu] (3.78,1.59) node {$F$};
\draw [fill=uuuuuu] (1.04,2.62) circle (1.5pt);
\draw[color=uuuuuu] (1.18,2.91) node {$G$};
\draw [fill=uuuuuu] (-1.56,1.31) circle (1.5pt);
\draw[color=uuuuuu] (-1.94,1.55) node {$H$};
\draw [fill=uuuuuu] (2.34,1.965) circle (1.5pt);
\draw[color=uuuuuu] (2.48,2.25) node {$I$};
\end{scriptsize}
\end{tikzpicture}
\end{center}
On en déduit que l'aire du triangle vaut un quart de celle du rectangle.
\end{sol}
\end{exo}

\begin{exo}[Théorème du chevron]
% Debart
Soit $ABC$ un triangle, $P$ un point sur $[BC]$, et $D$ un point sur $(AP)$. Montrer que la proportion entre les aires de $ACD$ et $ABD$ est la même que celle entre les longueurs $PC$ et $PB$.

(Par exemple, si $P$ est au tiers de $[BC]$ à partir de $B$, alors $PC = 2PB$ et le résultat dit que l'aire de $ACD$ vaut le double de celle de $ABD$.)
\begin{center}
\definecolor{ttzzqq}{rgb}{0.2,0.6,0.}
\definecolor{zzttqq}{rgb}{0.6,0.2,0.}
\definecolor{xdxdff}{rgb}{0.49019607843137253,0.49019607843137253,1.}
\definecolor{qqqqff}{rgb}{0.,0.,1.}
\begin{tikzpicture}[line cap=round,line join=round,>=triangle 45,x=1.0cm,y=1.0cm]
\clip(-3.22,-0.02) rectangle (2.94,5.86);
\fill[color=zzttqq,fill=zzttqq,fill opacity=0.10000000149011612] (-0.84,5.12) -- (-2.76,1.) -- (-1.1439416043048076,2.3339872617119015) -- cycle;
\fill[color=ttzzqq,fill=ttzzqq,fill opacity=0.10000000149011612] (-1.1439416043048076,2.3339872617119015) -- (2.32,1.) -- (-0.84,5.12) -- cycle;
\draw (-0.84,5.12)-- (-2.76,1.);
\draw (-2.76,1.)-- (2.32,1.);
\draw (2.32,1.)-- (-0.84,5.12);
\draw (-1.289473684210526,1.)-- (-0.84,5.12);
\draw [color=zzttqq] (-0.84,5.12)-- (-2.76,1.);
\draw [color=zzttqq] (-2.76,1.)-- (-1.1439416043048076,2.3339872617119015);
\draw [color=zzttqq] (-1.1439416043048076,2.3339872617119015)-- (-0.84,5.12);
\draw [color=ttzzqq] (-1.1439416043048076,2.3339872617119015)-- (2.32,1.);
\draw [color=ttzzqq] (2.32,1.)-- (-0.84,5.12);
\draw [color=ttzzqq] (-0.84,5.12)-- (-1.1439416043048076,2.3339872617119015);
\begin{scriptsize}
\draw [fill=qqqqff] (-0.84,5.12) circle (2.5pt);
\draw[color=qqqqff] (-0.9,5.59) node {$A$};
\draw [fill=qqqqff] (-2.76,1.) circle (2.5pt);
\draw[color=qqqqff] (-3.1,0.75) node {$B$};
\draw [fill=qqqqff] (2.32,1.) circle (2.5pt);
\draw[color=qqqqff] (2.46,0.59) node {$C$};
\draw [fill=xdxdff] (-1.289473684210526,1.) circle (2.5pt);
\draw[color=xdxdff] (-1.28,0.49) node {$P$};
\draw [fill=xdxdff] (-1.1439416043048076,2.3339872617119015) circle (2.5pt);
\draw[color=xdxdff] (-1.,2.01) node {$D$};
\end{scriptsize}
\end{tikzpicture}
\end{center}
\end{exo}


%%%%%%%%%%%%%%%%%%%%%%%
\subsection{Symétries centrales}

\begin{exo}[Points au tiers des côtés]
% symétrie centrale
Soit $ABCD$ un quadrilatère. On place des points $I$, $J$, $K$ et $L$ au tiers de chacun de ses côtés lorsqu'on les parcourt dans le même sens. 
Montrer que si $ABCD$ est un parallélogramme, alors $IJKL$ aussi.
% et réciproquement?
\begin{sol}
On considère la symétrie centrale  de centre $O$ (le centre de $ABCD$). Comme une symétrie centrale préserve les distances, et de plus envoie une droite sur une droite parallèle, on voit que l'image de $I$ est $K$, et l'image de $J$ est $L$. Ceci montre que $IJKL$ est un parallélogramme.

%Pour la réciproque, on considère la symétrie centrale centrée sur le centre de $IJKL$, et on montre qu'elle envoie $A$ sur $D$ et $B$ sur $D$.
\end{sol}
\end{exo}


\begin{exo}[Trapèze rectangle]
%tags : isocèle, trapèze, symétrie centrale, projection
Soit $\mathcal D$ une droite, $A$ et $B$ deux points hors de cette droite, et $A'$, $B'$ leurs projetés orthogonaux sur $\mathcal D$, supposés distincts. Soit enfin $I$ le milieu de $[AB]$. Montrer que $A'IB'$ est isocèle en $I$.
\begin{center}
\includegraphics{images/img007081-1}
\end{center}
% en utilisant une symétrie centrale, ou bien en utilisant une projection affine.
\begin{hint} %Sans utiliser le "théorème des milieux", on peut compléter le trapèze rectangle en un rectangle grâce à la symétrie de centre $I$.
Les diagonales d'un rectangle sont égales et se coupent en leur milieu. % Y a-t-il plus simple ?
\end{hint}
\begin{sol}
\begin{center}
\includegraphics{images/img007081-2}
\end{center}
\underline{Première solution} : soit $\sigma$ la symétrie centrale de centre $I$ et  $A''$ (respectivement $B''$) l'image de $A'$ (resp. $B'$) par $\sigma$. 

Par construction, $A'B'A''B''$ est un parallélogramme de centre $I$, et par construction également, on a $B=\sigma(A)$.

Montrons que $A'B'A''B''$ est un rectangle. Comme une symétrie centrale envoie une droite sur une droite parallèle, l'image de la droite $(AA')$ lui est parallèle, et doit forcément contenir $\sigma(A)$ c'est-à-dire $B$. C'est donc la droite $(BB')$. Ceci montre que $A''$ est le point d'intersection des droites $(A'I)$ et $(BB')$, et donc que $A'B'A''B''$ est un rectangle.

Comme les diagonales d'un rectangle on même longueur, on a terminé.

\underline{Deuxième solution} : une projection orthogonale sur une droite préserve les milieux : on peut par exemple le prouver en considérant un repère orthonormé et des coordonnées. On voit alors que la  projection orthogonale $I'$ de $I$ sur $(A'B')$ est le milieu de $[A'B']$. Ceci signifie que dans le triangle $A'IB'$, la hauteur issue de $I$ est également la médiane issue de $I$. Le triangle $A'IB'$ est donc isocèle, d'où $A'I = B'I$.
\end{sol}
\end{exo}

\begin{exo}[Distance entre symétriques]
Soit $ABC$ un triangle et $I$, $J$ les milieux de $[AB]$ et $[AC]$.

Pour tout point $M$, on note $M'$ et $M''$ ses symétriques par rapport à $I$ et $J$. Montrer que $M'M''=BC$.
\begin{center}

\definecolor{uuuuuu}{rgb}{0.26666666666666666,0.26666666666666666,0.26666666666666666}
\definecolor{qqqqff}{rgb}{0.,0.,1.}
\begin{tikzpicture}[line cap=round,line join=round,>=triangle 45,x=1.0cm,y=1.0cm]
\clip(-4.3,-1.28) rectangle (4.46,4.28);
\draw (-2.38,-0.78)-- (0.16,3.34);
\draw (0.16,3.34)-- (3.82,0.88);
\draw (3.82,0.88)-- (-2.38,-0.78);
\draw [dash pattern=on 2pt off 2pt] (1.78,0.68)-- (-4.,1.88);
\draw [dash pattern=on 2pt off 2pt] (1.78,0.68)-- (2.2,3.54);
\begin{scriptsize}
\draw [fill=qqqqff] (0.16,3.34) circle (2.5pt);
\draw[color=qqqqff] (0.3,3.71) node {$A$};
\draw [fill=qqqqff] (-2.38,-0.78) circle (2.5pt);
\draw[color=qqqqff] (-2.66,-0.21) node {$B$};
\draw [fill=qqqqff] (3.82,0.88) circle (2.5pt);
\draw[color=qqqqff] (3.96,1.25) node {$C$};
\draw [fill=uuuuuu] (-1.11,1.28) circle (1.5pt);
\draw[color=uuuuuu] (-1.34,1.85) node {$I$};
\draw [fill=uuuuuu] (1.99,2.11) circle (1.5pt);
\draw[color=uuuuuu] (1.9,2.73) node {$J$};
\draw [fill=qqqqff] (1.78,0.68) circle (2.5pt);
\draw[color=qqqqff] (2.14,0.91) node {$M$};
\draw [fill=qqqqff] (-4.,1.88) circle (2.5pt);
\draw[color=qqqqff] (-3.8,2.25) node {$M'$};
\draw [fill=qqqqff] (2.2,3.54) circle (2.5pt);
\draw[color=qqqqff] (2.46,3.91) node {$M''$};
\end{scriptsize}
\end{tikzpicture}
\end{center}
\begin{sol}
Le quadrilatère $AMBM'$ est un parallélogramme car ses diagonales se croisent en leur milieu. Les segments $[AM]$ et $[M'B]$ sont donc parallèles et de même longueur.

De même, $AMCM''$ est un parallélogramme, et donc les segments $[AM]$ et $[M''C]$ sont donc parallèles et de même longueur.

On en déduit que $M'BCM''$ est un parallélogramme, et donc ses côtés opposés $[M'M'']$ et $[BC]$ sont (parallèles et) de même longueur.
\end{sol}
\end{exo}


\begin{exo}[Constructions d'un losange inscrit]
% source : Euclidea
\begin{prerequis}
Médiatrice, losanges
\end{prerequis}
Soit $ABCD$ un rectangle, avec $AB\geq BC$. Expliquer comment construire un point $P$ sur $[AB]$ et un point $Q$ sur $[CD]$ de telle sorte que $APCQ$ soit un losange.
\begin{center}
\definecolor{uuuuuu}{rgb}{0.26666666666666666,0.26666666666666666,0.26666666666666666}
\definecolor{xdxdff}{rgb}{0.49019607843137253,0.49019607843137253,1.}
\definecolor{qqqqff}{rgb}{0.,0.,1.}
\begin{tikzpicture}[line cap=round,line join=round,>=triangle 45,x=1.0cm,y=1.0cm]
\clip(-2.4,0.02) rectangle (3.18,3.38);
\draw (-1.76,0.68)-- (2.34,0.68);
\draw (-1.76,0.68)-- (-1.76,2.8208778783393584);
\draw (-1.76,2.8208778783393584)-- (2.34,2.820877878339359);
\draw (2.34,2.820877878339359)-- (2.34,0.68);
\draw [line width=2.pt] (-1.76,0.68)-- (0.848946108532053,0.68);
\draw [line width=2.pt] (0.848946108532053,0.68)-- (2.34,2.820877878339359);
\draw [line width=2.pt] (2.34,2.820877878339359)-- (-0.2689461085320527,2.820877878339358);
\draw [line width=2.pt] (-0.2689461085320527,2.820877878339358)-- (-1.76,0.68);
\begin{scriptsize}
\draw [fill=qqqqff] (-1.76,0.68) circle (2.5pt);
\draw[color=qqqqff] (-2.24,0.87) node {$A$};
\draw [fill=qqqqff] (2.34,0.68) circle (2.5pt);
\draw[color=qqqqff] (2.48,1.05) node {$B$};
\draw [fill=xdxdff] (2.34,2.820877878339359) circle (2.5pt);
\draw[color=xdxdff] (2.48,3.19) node {$C$};
\draw [fill=uuuuuu] (-1.76,2.8208778783393584) circle (1.5pt);
\draw[color=uuuuuu] (-1.62,3.11) node {$D$};
\draw [fill=uuuuuu] (0.848946108532053,0.68) circle (1.5pt);
\draw[color=uuuuuu] (0.68,1.03) node {$P$};
\draw [fill=uuuuuu] (-0.2689461085320527,2.820877878339358) circle (1.5pt);
\draw[color=uuuuuu] (-0.12,3.11) node {$Q$};
\end{scriptsize}
\end{tikzpicture}
\end{center}
\begin{hint}
Raisonner d'abord par conditions nécessaires en faisant une figure approximative. Si de tels points existent, alors on peut dire un certain nombre de choses sur ce losange.
\end{hint}
\begin{sol}
Procédons par conditions nécessaires, puis suffisantes. (On dit aussi que l'on raisonne par analyse-synthèse.)

Si de tels points existent et que $APCQ$ est un losange, alors ses diagonales $[AC]$ et $[PQ]$ se croisent en leur milieu et sont orthogonales. Donc $(PQ)$ doit être la médiatrice de $[AC]$.

Faisons maintenant la synthèse : traçons la médiatrice de $[AC]$ : elle passe par le centre $O$ du rectangle, et coupe $[AB]$ et $[CD]$ en deux points $P$ et $Q$. De plus, $OP=OQ$, en effet, toute droite passant par le centre du rectangle coupe le rectangle en deux points qui sont à même distance du centre, par symétrie centrale.
\end{sol}
\end{exo}


\begin{exo}[Parallélogramme cinq fois plus grand]
(Complexité : 4+2)
% source : http://debart.pagesperso-orange.fr/college/parallelogramme_translation_classique.html

Soit $ABCD$ un parallélogramme et $A'$ (respectivement $B'$, $C'$ et $D'$) le symétrique de $A$ (resp. $B$, $C$ et $D$) par rapport à $B$ (resp. $C$, $D$ et $A$).

Montrer que $A'B'C'D'$ est un parallélogramme, d'aire cinq fois plus grande que $ABCD$.
\begin{center}
\definecolor{uuuuuu}{rgb}{0.26666666666666666,0.26666666666666666,0.26666666666666666}
\definecolor{qqqqff}{rgb}{0.,0.,1.}
\begin{tikzpicture}[line cap=round,line join=round,>=triangle 45,x=0.5cm,y=0.5cm]
\clip(-5.671071951385791,-2.8689342666231306) rectangle (9.262854391689489,7.027400156332925);
\draw (-0.36,3.3)-- (-2.52,-2.14);
\draw (-1.44,0.58)-- (8.44,0.78);
\draw (3.5,0.68)-- (5.66,6.12);
\draw (4.58,3.4)-- (-5.3,3.2);
\draw (-5.3,3.2)-- (-2.52,-2.14);
\draw (-2.52,-2.14)-- (8.44,0.78);
\draw (8.44,0.78)-- (5.66,6.12);
\draw (5.66,6.12)-- (-5.3,3.2);
\begin{scriptsize}
\draw [fill=qqqqff] (-1.44,0.58) circle (2.5pt);
\draw[color=qqqqff] (-1.9152341884566793,1.170081740336344) node {$A$};
\draw [fill=qqqqff] (3.5,0.68) circle (2.5pt);
\draw[color=qqqqff] (3.301207148944865,0.21621818149720606) node {$B$};
\draw [fill=qqqqff] (4.58,3.4) circle (2.5pt);
\draw[color=qqqqff] (5.0598930855545285,3.5547406374341883) node {$C$};
\draw [fill=uuuuuu] (-0.36,3.3) circle (1.5pt);
\draw[color=uuuuuu] (-0.3652059053430775,3.882631235785142) node {$D$};
\draw [fill=qqqqff] (8.44,0.78) circle (2.5pt);
\draw[color=qqqqff] (8.726306139842471,1.3191229214049591) node {$A'$};
\draw [fill=qqqqff] (5.66,6.12) circle (2.5pt);
\draw[color=qqqqff] (5.954140171966222,6.68460543987511) node {$B'$};
\draw [fill=qqqqff] (-5.3,3.2) circle (2.5pt);
\draw[color=qqqqff] (-5.015290754683883,3.7633982909302497) node {$C'$};
\draw [fill=uuuuuu] (-2.52,-2.14) circle (1.5pt);
\draw[color=uuuuuu] (-3.196988345646773,-1.7809336448222388) node {$D'$};
\end{scriptsize}
\end{tikzpicture}
\end{center}
\begin{sol}
Soit $O$ le centre de $ABCD$.

Comme $BA'=DC'$ et $(BA')//(DC')$, on en déduit que $BA'DC'$ est un parallélogramme.

\marginpar{1} 

Son centre est le milieu de la diagonale $[BD]$ c'est-à-dire $O$, et donc $O$ est le milieu de $[A'C']$. (En d'autres termes, $A'$ et $C'$ sont symétriques para rapport à $O$.)

\marginpar{1} 

On montre de même que $O$ est le milieu de $[B'D']$.

\marginpar{1} 

Donc $A'B'C'D'$ est un parallélogramme, soit parce que ses diagonales se croisent en leur milieu, soit, en reformulant avec des symétries, parce que la symétrie centrale de centre $O$ envoie $A'$ sur $C'$ et $B'$ sur $D'$.

\marginpar{1} 

Montrons que son aire est cinq fois celle de $ABCD$.

Le triangle $BA'B'$ a la même aire que $BAB'$, car ils ont des bases $[AB]$ et $[BA']$ de même longueur et la même hauteur.

\marginpar{1} 

Ensuite, le triangle $BAB'$ a la même aire que $ABCD$, car sa base $[BB']$ est deux fois plus grande que la base $[BC]$ du parallélogramme, et les hauteurs relativement à ces bases sont égales.

\marginpar{1} 

On peut faire un peu plus rapide avec la théorème des milieux, mais cette preuve-ci évite de faire appel à ce théorème.
\end{sol}
\end{exo}







\subsection{Inégalité triangulaire}

\begin{exo}[Chemin moins long]
% un grand classique
Soit $ABC$ un triangle, et soit $O$ un point à l'intérieur du triangle. Montrer que 
\[OA+OB \leq CA+CB.\]
\begin{center}
\definecolor{qqqqff}{rgb}{0.,0.,1.}
\begin{tikzpicture}[line cap=round,line join=round,>=triangle 45,x=1.0cm,y=1.0cm]
\clip(-2.8,-0.86) rectangle (3.42,3.38);
\draw (-1.24,2.92)-- (-1.98,-0.36);
\draw (-1.98,-0.36)-- (2.56,0.72);
\draw (2.56,0.72)-- (-1.24,2.92);
\draw [dash pattern=on 2pt off 2pt] (0.62,0.92)-- (-1.24,2.92);
\draw [dash pattern=on 2pt off 2pt] (0.62,0.92)-- (-1.98,-0.36);
\begin{scriptsize}
\draw [fill=qqqqff] (-1.24,2.92) circle (2.5pt);
\draw[color=qqqqff] (-1.74,3.01) node {$A$};
\draw [fill=qqqqff] (-1.98,-0.36) circle (2.5pt);
\draw[color=qqqqff] (-2.5,-0.27) node {$B$};
\draw [fill=qqqqff] (2.56,0.72) circle (2.5pt);
\draw[color=qqqqff] (2.9,0.81) node {$C$};
\draw [fill=qqqqff] (0.62,0.92) circle (2.5pt);
\draw[color=qqqqff] (0.76,1.29) node {$O$};
\end{scriptsize}
\end{tikzpicture}
\end{center}
\begin{hint}
Prolonger le segment $[AO]$. 
\end{hint}
\begin{sol}
La droite $(AO)$ coupe $[BC]$ en un point $A'$. 
\begin{center}
\definecolor{uuuuuu}{rgb}{0.26666666666666666,0.26666666666666666,0.26666666666666666}
\definecolor{qqqqff}{rgb}{0.,0.,1.}
\begin{tikzpicture}[line cap=round,line join=round,>=triangle 45,x=1.0cm,y=1.0cm]
\clip(-2.8,-0.86) rectangle (3.42,3.38);
\draw (-1.24,2.92)-- (-1.98,-0.36);
\draw (-1.98,-0.36)-- (2.56,0.72);
\draw (2.56,0.72)-- (-1.24,2.92);
\draw [dash pattern=on 2pt off 2pt] (0.62,0.92)-- (-1.98,-0.36);
\draw [dash pattern=on 2pt off 2pt,domain=-2.8:3.42] plot(\x,{(--2.9512-2.*\x)/1.86});
\begin{scriptsize}
\draw [fill=qqqqff] (-1.24,2.92) circle (2.5pt);
\draw[color=qqqqff] (-1.74,3.01) node {$A$};
\draw [fill=qqqqff] (-1.98,-0.36) circle (2.5pt);
\draw[color=qqqqff] (-2.5,-0.27) node {$B$};
\draw [fill=qqqqff] (2.56,0.72) circle (2.5pt);
\draw[color=qqqqff] (2.9,0.81) node {$C$};
\draw [fill=qqqqff] (0.62,0.92) circle (2.5pt);
\draw[color=qqqqff] (0.76,1.29) node {$O$};
\draw [fill=uuuuuu] (1.123747204386408,0.378336339369454) circle (1.5pt);
\draw[color=uuuuuu] (1.32,0.67) node {$A'$};
\end{scriptsize}
\end{tikzpicture}
\end{center}
On applique l'inégalité triangulaire sur $AA'C$ et $OA'B$:
\[ AO+OA' \leq AC+CA',\]
\[ OB \leq OA'+A'B\]
En sommant, on obtient le résultat.
\end{sol}
\end{exo}

\begin{exo}[Conditions d'intersection de cercles]
Soient $\mathcal C$ et $\mathcal C'$ deux cercles de rayon $R$ et $R'$ (avec $R\leq R'$), dont les centres sont à une distance $d$ l'un de l'autre.

Montrer que
\begin{enumerate}
\item les cercles sont extérieurs si et seulement si $d > R+R'$;
\item les cercles sont tangents extérieurement si et seulement si $d=R+R'$.
\item les cercles sont sécants si et seulement si $R'-R<d<R+R'$;
\item les cercles sont tangents intérieurement si et seulement si $d=R'-R$.
\item les cercles sont intérieurs l'un à l'autre si et seulement si $d<R'-R$.
\end{enumerate}
\end{exo}


\begin{exo}[Prendre de l'eau à la rivière \faLightbulbO]
\label{riviere}

Soit $\mathcal D$ une droite, et $A$ et $B$ deux points distincts, situés du même côté de $\mathcal D$. Quel est le plus court chemin allant de $A$ à $B$, et qui touche la droite $\mathcal D$ ?

Que dire des angles d'incidence et de réflexion du chemin sur la droite $\mathcal D$ ?

\begin{sol}
Considérons la symétrie axiale d'axe $\mathcal D$, et soit $B'$ l'image de $B$ par cette symétrie axiale.  Les points $A$ et  $B'$ sont situés de part et d'autre de la droite $\mathcal D$.

Si un chemin va de $A$ à $B$ en touchant la droite, alors à partir du moment où on touche la droite, il est équivalent d'aller vers $B$ ou d'aller vers $B'$, puisque la droite $\mathcal D$ est la médiatrice de $[BB']$.


Considérons alors le segment $[AB']$ : c'est le plus court chemin entre $A$ et $B'$. Il coupe $\mathcal D$ en un point $C$. Alors le plus court chemin de $A$ à $B$ qui touche la droite $\mathcal D$ consiste à aller de $A$ à $C$ en ligne droite, puis de $C$ à $B$ en ligne droite.

En utilisant qu'une symétrie axiale conserve les angles (non orientés), ainsi que des angles opposés par le sommet ou bien alternes-internes, on en déduit que les angles d'incidence et de réflexion sur la droite $\mathcal D$ sont égaux.

\end{sol}
\end{exo}

\begin{exo}[Double contrainte]
\label{rivieres}
On donne deux droites $\mathcal D$ et $\mathcal D'$, et deux points $A$ et $B$ comme sur l'illustration. Parmi tous les chemins allant de $A$ à $B$ en passant par la droite $\mathcal D$ puis par la droite $\mathcal D'$, déterminer le plus chemin le plus court.
\begin{center}
\definecolor{qqqqff}{rgb}{0.,0.,1.}
\begin{tikzpicture}[line cap=round,line join=round,>=triangle 45,x=1.0cm,y=1.0cm]
\clip(-3.06,-0.68) rectangle (4.94,4.92);
\draw [domain=-3.06:4.94] plot(\x,{(--23.5544-0.76*\x)/5.46});
\draw [domain=-3.06:4.94] plot(\x,{(-3.8484--1.92*\x)/6.58});
\draw [dash pattern=on 2pt off 2pt] (-2.1,2.66)-- (-0.47988996090716435,4.380790544009056);
\draw [dash pattern=on 2pt off 2pt] (-0.47988996090716435,4.380790544009056)-- (3.0011383543355796,0.2908488815082543);
\draw [dash pattern=on 2pt off 2pt] (3.0011383543355796,0.2908488815082543)-- (4.02,1.46);
\begin{scriptsize}
\draw [fill=qqqqff] (-2.1,2.66) circle (2.5pt);
\draw[color=qqqqff] (-2.24,2.09) node {$A$};
\draw [fill=qqqqff] (4.02,1.46) circle (2.5pt);
\draw[color=qqqqff] (3.94,2.11) node {$B$};
\end{scriptsize}
\end{tikzpicture}
\end{center}
\begin{hint}
Considérer les symétries axiales d'axe $\mathcal D$ et $\mathcal D'$.
\end{hint}
\end{exo}

\begin{exo}[Périmètre minimal à base fixée]
%[Application directe de l'exercice \ref{riviere}]

\begin{prerequis} Aire = base $\cdot$ hauteur, inégalité triangulaire, symétries axiales. 
\end{prerequis}
Soient $A$ et $B$ deux points, et $\mathcal A$ une mesure d'aire. Parmi tous les triangles $ABC$ d'aire $\mathcal A$, déterminer celui ou ceux dont le périmètre est minimal.
\begin{sol}
Il y a deux triangles vérifiant ces conditions : ce sont les deux triangles isocèles de base $AB$ et d'aire $\mathcal A$. Pour le démontrer, considérons les deux droites parallèles à $(AB)$ et à une distance $h =\frac{\mathcal A}{ AB}$ de la droite $(AB)$. Un triangle $ABC$ a une aire égale à $\mathcal A$ si et seulement si le point $C$ appartient à un de ces deux droites (suivant s'il est direct ou indirect).

Pour simplifier, supposons que l'on cherche uniquement les triangles $ABC$ directs, et appelons $\mathcal D$ la droite formée des points $C$ tels que $ABC$ soit direct et d'aire $\mathcal A$.

Il s'agit donc de trouver un tel point $C$ pour que le périmètre de $ABC$ soit minimal. Comme la distance $AB$ est fixe, il s'agit donc de minimiser $AC+CB$. On peut appliquer le résultat de l'exercice \ref{riviere}, d'après lequel le point $C$ qui convient est l'intersection de $\mathcal D$ et du segment joignant $A$ à l'image de $B$ par symétrie axiale d'axe $\mathcal D$. Ceci produit un triangle isocèle (utilisation d'angles opposés, ou alors alternes-internes).
\end{sol}
\end{exo}

\begin{exo}[Aire maximale à base et périmètre fixés]
%[Le problème dual du précédent]
Soient $A$ et $B$ deux points. Parmi tous les triangles de périmètre $p$ fixé, trouver celui ou ceux d'aire maximale.
\begin{sol}


Ce sont les deux triangles isocèles de base $[AB]$. Le problème est dual du précédent.

Autre solution, de plus haut niveau : fixer le périmètre revient à fixer la longueur $AC+CB$. Le point $C$ décrit donc une ellipse.

D'autre part,  par la formule $\mathcal A = AB \cdot h$ avec $h$ la longueur de la hauteur issue de $C$, le  ou les triangles d'aire maximale sont ceux pour lesquels $C$ est le plus éloigné de la droite $(AB)$.

On voit alors que l'aire maximale est atteinte lorsque $C$ est sur a médiatrice de $[AB]$.


\end{sol}
\end{exo}



\subsection{Sorti du programme : triangles rectangles et demi-cercles}

\begin{exo}[Triangles rectangles et demi-cercles]
\begin{enumerate}
\item Soit $ABC$ un triangle rectangle en $A$. Montrer que $[BC]$ est un diamètre du cercle circonscrit.
\item Montrer ensuite la réciproque : si un triangle est inscrit dans un demi-cercle au sens où $B$ appartient au cercle de diamètre $[AB]$, alors il est rectangle (en $B$).
\end{enumerate}

\begin{sol}

Notons $O$ le milieu de $[BC]$, et considérons le symétrique $A'$ de $A$ par rapport à $O$. Alors $ABA'C$ est un rectangle : en effet c'est un parallélogramme car ses côtés sont deux à deux de même longueur (ou aussi : parallèles deux à deux, puisqu'une symétrie centrale envoie une droite sur une droite parallèle). Or, un parallélogramme qui possède un angle droit est un rectangle.

On en déduit que les médiatrices de ses côtés se croisent au centre de ce rectangle, qui est le milieu de ses diagonales, donc le milieu de $[BC]$, et donc $O$. Ceci montre que $O$ est le centre du cercle circonscrit à $ABC$.

Preuve de la réciproque:
\end{sol}
\end{exo}


\begin{exo}[Triangle de l'écolier, 1]
%tags : triangle rectangle, cercle circonscrit
On donne deux points $A$ et $B$. Construire un triangle $ABC$ rectangle en $C$ tel que $AB = 2AC$.

\begin{hint}   Tracer le milieu $I$ de $AB$ puis procéder par analyse-synthèse.
\end{hint}      
\begin{sol} 
On relie les points $A$ et $B$. On construit le milieu $I$ de $[AB]$. On a donc $AI = \frac{1}{2}AB$. On trace le cercle $\mathcal{C}$ de centre $A$ et de rayon $AI$. Ceci fait, On trace le cercle $\mathcal{C}'$ centr\'e en $I$ de diam\`etre $AB$. Nommons $C$ l'un des deux points d'intersection de $\mathcal{C}$ et $\mathcal{C}'$. Il s'agit de l'une des deux solutions possibles.

En effet, par construction $AC = \frac{1}{2} AB$, puisque $C$ appartient \`a $\mathcal{C}$. En outre $ACB$ est rectangle en $C$. En effet, $ACB$ est inscrit dans le cercle $\mathcal{C}'$ et $AB$ est un diam\`etre de $\mathcal{C}'$.
\end{sol}  
\end{exo}  

\begin{exo}[Triangle de l'écolier, 2]
Soit $ADC$ un triangle équilatéral et $B$ le symétrique de $A$ par rapport à $D$. Montrer que $ABC$ est rectangle en $C$ et déterminer également $\widehat A$ et $\widehat B$.


\begin{sol} 
Comme par hypoth\`ese $DA = DC = DB$, les points $A$, $B$ et $C$ sont situ\'es sur le cercle de centre $A$ et de diam\`etre $AB$. Donc, $ACB$ est rectangle en $C$. 

Puisque $ADC$ est \'equilat\'eral, $\widehat{DAC} = \widehat{BAC}= 60^{\circ}$. Comme $ACB$ est rectangle en $C$, donc $\widehat{ACB}=90^{\circ}$.

De la relation $180^{\circ} = \widehat{BAC} + \widehat{ACB} + \widehat{CBA}$ on tire $\widehat{CBA} = 30^{\circ}$.    
\end{sol}  
\end{exo} 



\begin{exo}[Droites parallèles]
 % exo7
% collège
% triangle rectangle, cercle circonscrit
Soit $ABC$ un triangle et $\mathcal C$ son cercle circonscrit, de centre $O$.
$A’$ est le point diamétralement opposé à $A$ sur le cercle $\mathcal C$.
La hauteur $(AH)$ issue de $A$ du triangle $ABC$ recoupe le cercle $\mathcal C$ au point $D$.

Montrer que la droite $(DA’)$ est parallèle à $(BC)$.

\begin{hint}   
Le triangle rectangle $ADA’$ est inscrit dans un demi-cercle. 
% donc il y a un autre angle droit
\end{hint}      
\begin{sol} 

%%%%%%%%%%%%
%% FIGURE %%
%%%%%%%%%%%%

D'après les hypothèses, la droite $(BC)$ est perpendiculaire \`a la droite $(AH)$ et $ADA'$ est rectangle en $D$. Donc, les droites $(AH)$ et $(A'D)$ sont perpendiculaires. Or, deux droites perpendiculaires à une même droite sont parallèles.  
\end{sol}  
\end{exo}  


\begin{exo}[Angles inscrits égaux]\label{angles_inscrits_egaux}

Soit $[AB]$ un segment et $M$, $N$ deux points appartenant au cercle  $\mathcal C$  de diamètre $[AB]$. 

Montrer (sans utiliser le théorème de l'angle au centre) que $\widehat{MAN}$ et $\widehat{MBN}$ sont égaux si $M$ et $N$ sont du même côté de $[AB]$, et supplémentaires sinon.

\begin{center}
\definecolor{uuuuuu}{rgb}{0.26666666666666666,0.26666666666666666,0.26666666666666666}
\definecolor{xdxdff}{rgb}{0.49019607843137253,0.49019607843137253,1.}
\definecolor{qqqqff}{rgb}{0.,0.,1.}
\begin{tikzpicture}[line cap=round,line join=round,>=triangle 45,x=1.0cm,y=1.0cm]
\clip(-3.64,0.26) rectangle (2.76,5.32);
\draw(0.38,2.72) circle (2.0003999600079982cm);
\draw [domain=-3.64:2.76] plot(\x,{(--6.853230602674042--0.2985798317801791*\x)/1.5141288186885857});
\draw [domain=-3.64:2.76] plot(\x,{(-1.177621588999542--1.9006083732491723*\x)/-1.5995181006112178});
\draw [domain=-3.64:2.76] plot(\x,{(-9.654619241994855-2.0193916267508274*\x)/-2.3995181006112176});
\draw [domain=-3.64:2.76] plot(\x,{(-0.47030949883892875--3.621420168219821*\x)/-0.7141288186885858});
\begin{scriptsize}
\draw [fill=qqqqff] (-0.02,0.76) circle (2.5pt);
\draw[color=qqqqff] (0.12,1.13) node {$A$};
\draw [fill=qqqqff] (0.78,4.68) circle (2.5pt);
\draw[color=qqqqff] (0.96,4.29) node {$B$};
\draw [fill=xdxdff] (-0.7341288186885858,4.381420168219821) circle (2.5pt);
\draw[color=xdxdff] (-0.62,4.97) node {$M$};
\draw [fill=xdxdff] (-1.6195181006112178,2.6606083732491723) circle (2.5pt);
\draw[color=xdxdff] (-2.16,2.77) node {$N$};
%\draw [fill=uuuuuu] (-2.735570440533963,3.986744302208616) circle (1.5pt);
%\draw[color=uuuuuu] (-2.6,4.27) node {$P$};
%\draw [fill=uuuuuu] (-0.5691134172688166,3.5446102158279733) circle (1.5pt);
%\draw[color=uuuuuu] (-0.36,3.27) node {$Q$};
\end{scriptsize}
\end{tikzpicture}
\end{center}
\begin{hint}   
Pour le premier cas, utiliser des angles complémentaires.
\end{hint}      
\begin{sol}
Les angles en $M$ et $N$ sont droits.

Dans le premier cas, les angles sont égaux car ils ont le même complémentaire.

Dans le second cas, ils sont supplémentaires car leur comme plus deux angles droits vaut $2\pi$ (dans un quadrilatère convexe, la somme des angles vaut $2\pi$).
\end{sol}  
\end{exo}  



\subsection{Orthocentre}

L'orthocentre a l'air d'être au programme de 5ème mais sans démonstration que les hauteurs sont concourantes. La notion d'orthocentre apparaît dans les exercices de 4ème et de 3ème du Sésamath donc ça devrait être bon.

\begin{exo}[Les hauteurs sont concourantes]
Montrer que les hauteurs d'un triangle sont concourantes, en utilisant le triangle formé par les trois droites passant par les sommets et parallèles aux côtés. (Et trois parallélogrammes.)
\begin{hint}
\end{hint}
\end{exo}


\begin{exo}[Construction d'un projeté à la règle seule]
\begin{prerequis}
Triangles rectangles et demi-cercles
\end{prerequis}
 % exo7
% orthocentre, triangle rectangle inscrit 
% cercle circonscrit, hauteurs
On donne  un cercle $\mathcal C$, un diamètre $[AB]$ et un troisième point $M$ du cercle. L'objectif est de construire le projeté orthogonal de $M$ sur $(AB)$ à la règle seule.
\begin{enumerate}
\item Montrer qu'il suffit de construire une droite orthogonale à $(AB)$ coupant le cercle en deux points.
\item Construire une telle droite.
\end{enumerate}

\begin{hint}
\begin{enumerate}
\item Penser à un trapèze.
\item On peut obtenir une telle droite comme hauteur d'un triangle $ABC$ adéquat.
\end{enumerate}
\end{hint}      
\begin{sol} 
Prendre un deuxième point $N$ sur le cercle de telle sorte que $(AM)$ et $(BN)$ se coupent en un point $C$. On peut alors construire l'orthocentre de $ABC$. La troisième hauteur fournit une droite orthogonale à $(AB)$, coupant le cercle en deux points $P$ et $Q$. On peut alors compléter $MPQ$ en un trapèze (isocèle) $MPQR$, en utilisant les diagonales d'un tel trapèze. La droite $(MR)$ est orthogonale à $(AB)$.
\end{sol}  
\end{exo}  



\begin{exo}[Droites concourantes]
\begin{prerequis}
Triangles rectangles et demi-cercles
\end{prerequis}
 % exo7
% collège
% orthocentre, triangle rectangle inscrit 
% cercle circonscrit, hauteurs
Soit $ABC$ un triangle. Le cercle $\mathcal C$ (resp. $\mathcal C'$) de diamètre $[BC]$  (resp. $[CA]$) coupe la droite $(CA)$ (resp. la droite $(BC)$) en $P$ (resp. $Q$). Les cercles $\mathcal C$  et $\mathcal C'$ se recoupent en un second point $R$. Montrer que $R$ est sur $[AB]$ et que $(CR)$, $(BP)$ et $(AQ)$ sont concourantes.
\begin{center}
\definecolor{uuuuuu}{rgb}{0.26666666666666666,0.26666666666666666,0.26666666666666666}
\definecolor{qqqqff}{rgb}{0.,0.,1.}
\begin{tikzpicture}[line cap=round,line join=round,>=triangle 45,x=1.0cm,y=1.0cm]
\clip(-2.58,-0.48) rectangle (3.9,4.84);
\draw(-0.19,2.55) circle (1.9912307751739875cm);
\draw(1.9,2.28) circle (1.613071604114337cm);
\draw (-1.74,1.3)-- (2.44,0.76);
\draw (2.44,0.76)-- (1.36,3.8);
\draw (1.36,3.8)-- (-1.74,1.3);
\begin{scriptsize}
\draw [fill=qqqqff] (-1.74,1.3) circle (2.5pt);
\draw[color=qqqqff] (-2.14,1.13) node {$A$};
\draw [fill=qqqqff] (2.44,0.76) circle (2.5pt);
\draw[color=qqqqff] (2.8,0.43) node {$B$};
\draw [fill=qqqqff] (1.36,3.8) circle (2.5pt);
\draw[color=qqqqff] (1.5,4.17) node {$C$};
\draw [fill=uuuuuu] (1.8012144504227523,2.5580630284396606) circle (1.5pt);
\draw[color=uuuuuu] (1.94,2.85) node {$Q$};
\draw [fill=uuuuuu] (0.5289029003783106,3.129760403530896) circle (1.5pt);
\draw[color=uuuuuu] (0.34,3.49) node {$P$};
\draw [fill=uuuuuu] (0.9914478721008778,0.9471335284845757) circle (1.5pt);
\draw[color=uuuuuu] (1.14,0.45) node {$R$};
\end{scriptsize}
\end{tikzpicture}
\end{center}
\begin{hint}   
Montrer que $(AB) \bot (RC)$.
\end{hint}      
\begin{sol} 
Les trois droites sont les hauteurs de $ABC$.
%%%%%%%%%%%%
%% FIGURE %%
%%%%%%%%%%%%

\end{sol}  
\end{exo}  


\begin{exo}[Angle à déterminer]
\begin{prerequis}
Triangles rectangles et demi-cercles
\end{prerequis}
 % exo7
% collège
% orthocentre, triangle rectangle inscrit 
% cercle circonscrit, hauteurs

Soit $[AB]$ un segment et $M$, $N$ deux points appartenant au cercle  $\mathcal C$  de diamètre $[AB]$. On suppose que les droites $(MB)$ et $(AN)$ (respectivement $(NB)$ et $(AM)$ ) s'intersectent en $P$ (respectivement en $Q$). Déterminer l'angle formé par  les droites $(AB)$ et $(PQ)$.
\begin{center}
\definecolor{uuuuuu}{rgb}{0.26666666666666666,0.26666666666666666,0.26666666666666666}
\definecolor{xdxdff}{rgb}{0.49019607843137253,0.49019607843137253,1.}
\definecolor{qqqqff}{rgb}{0.,0.,1.}
\begin{tikzpicture}[line cap=round,line join=round,>=triangle 45,x=1.0cm,y=1.0cm]
\clip(-3.64,0.26) rectangle (2.76,5.32);
\draw(0.38,2.72) circle (2.0003999600079982cm);
\draw [domain=-3.64:2.76] plot(\x,{(--6.853230602674042--0.2985798317801791*\x)/1.5141288186885857});
\draw [domain=-3.64:2.76] plot(\x,{(-1.177621588999542--1.9006083732491723*\x)/-1.5995181006112178});
\draw [domain=-3.64:2.76] plot(\x,{(-9.654619241994855-2.0193916267508274*\x)/-2.3995181006112176});
\draw [domain=-3.64:2.76] plot(\x,{(-0.47030949883892875--3.621420168219821*\x)/-0.7141288186885858});
\begin{scriptsize}
\draw [fill=qqqqff] (-0.02,0.76) circle (2.5pt);
\draw[color=qqqqff] (0.12,1.13) node {$A$};
\draw [fill=qqqqff] (0.78,4.68) circle (2.5pt);
\draw[color=qqqqff] (0.96,4.29) node {$B$};
\draw [fill=xdxdff] (-0.7341288186885858,4.381420168219821) circle (2.5pt);
\draw[color=xdxdff] (-0.62,4.97) node {$M$};
\draw [fill=xdxdff] (-1.6195181006112178,2.6606083732491723) circle (2.5pt);
\draw[color=xdxdff] (-2.16,2.77) node {$N$};
\draw [fill=uuuuuu] (-2.735570440533963,3.986744302208616) circle (1.5pt);
\draw[color=uuuuuu] (-2.6,4.27) node {$P$};
\draw [fill=uuuuuu] (-0.5691134172688166,3.5446102158279733) circle (1.5pt);
\draw[color=uuuuuu] (-0.36,3.27) node {$Q$};
\end{scriptsize}
\end{tikzpicture}
\end{center}
\begin{hint}   
Déterminer les angles $\widehat{AMB}$ et $\widehat{ANB}$.
\end{hint}      
\begin{sol}  Le point $A$ est l'orthocentre de $PQB$. L'angle est donc droit.
\end{sol}  
\end{exo}  



%%%%%%%%%%%%%%%%%%%%%%%%%
\subsection{Bissectrices sans cercle inscrit}

%Exercices sur les losanges, cerf-volants.

\begin{exo}[Bissectrices d'un parallélogramme]
% pas mis sur exo7
% Debart, http://debart.pagesperso-orange.fr/college/parallelogramme_translation_classique.html
Montrer que les quatre points d'intersection des bissectrices intérieures d'un parallélogramme sont les quatre sommets d'un rectangle. Réciproquement, montrer que si les bissectrices d'un quadrilatère non croisé sont perpendiculaires, alors ce quadrilatère est un parallélogramme.
% y a-t-il besoin de non croisé ?
\begin{sol}
Dans un parallélogramme, les angles opposés sont égaux, et deux angles consécutifs sont supplémentaires. On en déduit que deux bissectrices consécutives sont perpendiculaires. La réciproque est vraie aussi : si les bissectrices consécutives sont perpendiculaires, alors deux angles consécutifs sont supplémentaires et donc on a un parallélogramme.
\end{sol}
\end{exo}


\begin{exo}[Construction d'un point équidistant $\heartsuit$ ]
% bissectrice, losange inscrit dans un triangle.
% Debart, triangle_college_classique.html
Soit $ABC$ un triangle. Construire un point $M$ sur le côté $[AB]$ de telle sorte que la parallèle à $(BC)$ passant par $M$ coupe $[AC]$ en un point $N$ tel que $BM=MN$.
\begin{center}
\definecolor{uuuuuu}{rgb}{0.26666666666666666,0.26666666666666666,0.26666666666666666}
\definecolor{qqqqff}{rgb}{0.,0.,1.}
\begin{tikzpicture}[line cap=round,line join=round,>=triangle 45,x=1.0cm,y=1.0cm]
\clip(-3.14,-1.02) rectangle (5.22,5.52);
\draw (-1.02,5.02)-- (-2.46,0.04);
\draw (-2.46,0.04)-- (4.26,0.04);
\draw (4.26,0.04)-- (-1.02,5.02);
\draw [domain=-3.14:5.22] plot(\x,{(--137.06427964905203-0.*\x)/48.07102112277704});
\draw [line width=2.pt] (-2.46,0.04)-- (-1.6470977226389687,2.8512870425402332);
\draw [line width=2.pt] (-2.1827151819262993,1.4309444865774053) -- (-1.9521602440689736,1.3642779985222755);
\draw [line width=2.pt] (-2.154937478569995,1.5270090440179573) -- (-1.9243825407126696,1.4603425559628276);
\draw [line width=2.pt] (-1.6470977226389687,2.8512870425402332)-- (1.2793583163428859,2.8512870425402332);
\draw [line width=2.pt] (-0.2338697031480416,2.9712870425402333) -- (-0.2338697031480416,2.7312870425402336);
\draw [line width=2.pt] (-0.13386970314804159,2.9712870425402333) -- (-0.13386970314804159,2.7312870425402336);
\begin{scriptsize}
\draw [fill=qqqqff] (-1.02,5.02) circle (2.5pt);
\draw[color=qqqqff] (-1.58,4.89) node {$A$};
\draw [fill=qqqqff] (-2.46,0.04) circle (2.5pt);
\draw[color=qqqqff] (-2.9,0.33) node {$B$};
\draw [fill=qqqqff] (4.26,0.04) circle (2.5pt);
\draw[color=qqqqff] (4.24,0.65) node {$C$};
\draw [fill=uuuuuu] (1.2793583163428859,2.8512870425402332) circle (1.5pt);
\draw[color=uuuuuu] (1.42,3.15) node {$N$};
\draw [fill=uuuuuu] (-1.6470977226389687,2.8512870425402332) circle (1.5pt);
\draw[color=uuuuuu] (-2.1,3.27) node {$M$};
\end{scriptsize}
\end{tikzpicture}
\end{center}
\begin{hint}
Reformulation de l'énoncé : le triangle $BMN$ doit être isocèle.
\end{hint}
\begin{sol}
Méthodologie : tracer la figure avec le point $M$ construit (même approximativement), ce qui permet de réfléchir. Ensuite, tracer ce qui peut être tracé, par exemple ici la droite $(BN)$. On voit que cette droite doit être la bissectrice de l'angle $\widehat{CBA}$, bissectrice qui est constructible  priori. Une fois construite, elle coupe $[AC]$ en un point $N$. On construit alors $M$ en menant la parallèle.

Noter qu'en appelant $M'$ le symétrique de $M$ par rapport à la bissectrice $(BN)$, alors $M'$ est sur $[BC]$ et en plus $BMNM'$ est un losange, ce qui montre que la construction est correcte.
\end{sol}
\end{exo}


\begin{exo}[Triangles bisocèles]
% Debart, triangle_college_classique.html
% ouverture vers le nombre d'or et les triangles d'or et d'argent
Un triangle bisocèle est, par définition,  un triangle isocèle tel qu'une des trois bissectrices le partage en deux triangles  isocèles.

Montrer qu'il n'existe que deux types possibles de triangles bisocèles. (Par \og type\fg{}, on entend la donnée des trois angles\footnote{Autrement dit, le type est la classe de similitude, mais on n'a pas besoin de parler de similitude ou de triangles semblables pour cet exercice.}.)
\end{exo}




\subsection{Sorti du programme : centre de gravité}

\begin{exo}[Les médianes sont concourantes]
% parallélogrammes, symétries centrales
Montrer que dans un triangle, deux médianes quelconques se croisent en un point, qui sur chacune des médianes est situé aux deux-tiers à partir du sommet. En déduire que les médianes d'un triangle sont concourantes. Leur point de concours $G$ est  appelé \emph{centre de gravité} du triangle.
\begin{hint}
Considérer deux médianes ainsi que leur point d'intersection $G$, puis considérer le symétrique de $G$ par rapport au pied des deux médianes.
\end{hint}
\begin{sol}
Notons $ABC$ un triangle quelconque, $I$ (resp. $J$) le pied de la médiane issue de $A$ (resp. $B$). Soit $M$ (resp. $N$) le symétrique de $G$ par rapport à  $I$ (resp. $J$).
\begin{center}
\definecolor{uuuuuu}{rgb}{0.26666666666666666,0.26666666666666666,0.26666666666666666}
\definecolor{qqqqff}{rgb}{0.,0.,1.}
\begin{tikzpicture}[line cap=round,line join=round,>=triangle 45,x=1.0cm,y=1.0cm]
\clip(-1.22,-4.32) rectangle (10.38,5.06);
\draw [line width=1.6pt] (-0.68,4.28)-- (4.08,4.28);
\draw [line width=1.6pt] (4.08,4.28)-- (3.06,0.4);
\draw [line width=1.6pt] (3.06,0.4)-- (-0.68,4.28);
\draw (-0.68,4.28)-- (3.57,2.34);
\draw (4.08,4.28)-- (1.19,2.34);
\draw [dash pattern=on 2pt off 2pt] (-0.68,4.28)-- (0.22666666666666646,1.6933333333333336);
\draw [dash pattern=on 2pt off 2pt] (0.22666666666666646,1.6933333333333336)-- (3.06,0.4);
\draw [dash pattern=on 2pt off 2pt] (4.08,4.28)-- (4.986666666666667,1.6933333333333336);
\draw [dash pattern=on 2pt off 2pt] (4.986666666666667,1.6933333333333336)-- (3.06,0.4);
\draw [dash pattern=on 2pt off 2pt] (0.22666666666666646,1.6933333333333336)-- (1.19,2.34);
\draw [dash pattern=on 2pt off 2pt] (3.57,2.34)-- (4.986666666666667,1.6933333333333336);
\draw [dash pattern=on 2pt off 2pt] (2.1533333333333338,2.9866666666666672)-- (3.06,0.4);
\begin{scriptsize}
\draw [fill=qqqqff] (-0.68,4.28) circle (2.5pt);
\draw[color=qqqqff] (-0.86,4.75) node {$A$};
\draw [fill=qqqqff] (4.08,4.28) circle (2.5pt);
\draw[color=qqqqff] (4.22,4.65) node {$B$};
\draw [fill=qqqqff] (3.06,0.4) circle (2.5pt);
\draw[color=qqqqff] (3.38,0.27) node {$C$};
\draw [fill=uuuuuu] (3.57,2.34) circle (1.5pt);
\draw[color=uuuuuu] (3.94,2.69) node {$I$};
\draw [fill=uuuuuu] (1.19,2.34) circle (1.5pt);
\draw[color=uuuuuu] (1.14,1.95) node {$J$};
\draw [fill=uuuuuu] (2.1533333333333338,2.9866666666666672) circle (1.5pt);
\draw[color=uuuuuu] (1.84,3.45) node {$G$};
\draw [fill=uuuuuu] (4.986666666666667,1.6933333333333336) circle (1.5pt);
\draw[color=uuuuuu] (5.12,1.99) node {$M$};
\draw [fill=uuuuuu] (0.22666666666666646,1.6933333333333336) circle (1.5pt);
\draw[color=uuuuuu] (-0.46,1.87) node {$N$};
\end{scriptsize}
\end{tikzpicture}
\end{center}

Alors, $AGCN$ et $BGCM$ sont des parallélogrammes car leurs diagonales se croisent en leur milieu.

On en déduit que $ABMN$ est un parallélogramme. Donc ses diagonales se coupent en leur milieu, ce qui montre que $AG = 2GI$ et $BG = 2GJ$. (Note : on aurait aussi peu utiliser que $GMCN$ est un parallélogramme.)

On a donc montré qu'étant donné un triangle, deux médianes quelconques se coupent toujours aux deux-tiers à partir des sommets. Ceci montre que les trois médianes sont forcément concourantes, puisqu'on peut échanger le rôle des deux dernières médianes.

(Autre façon de conclure, en utilisant le théorème des milieux : montrons que si $K$ est le milieu de $[AB]$, alors $(CK)$ contient le point $G$. Par le théorème des milieux appliqué au triangle $ABM$, le segment $[GK]$ est parallèle à $(BM)$, donc par ce qui précède, à $(GC)$. On en déduit que $K$, $G$ et $C$ sont alignés.)
\end{sol}
\end{exo}

\begin{exo}[Alignement]
(Complexité : 3/4)

Soit $ABCD$ un parallélogramme, et $M$ le milieu de $[BC]$. Soit $K$ le point de $[DM]$ tel que $KD=2KM$. Montrer que $A$, $K$ et $C$ sont alignés.
\begin{center}
\definecolor{uuuuuu}{rgb}{0.26666666666666666,0.26666666666666666,0.26666666666666666}
\definecolor{qqqqff}{rgb}{0.,0.,1.}
\begin{tikzpicture}[line cap=round,line join=round,>=triangle 45,x=1.0cm,y=1.0cm]
\clip(-2.5444651162790692,0.027720930232561652) rectangle (4.839348837209292,4.7551627906976766);
\draw (-1.44,0.58)-- (-0.9447441860465123,3.213860465116279);
\draw (-0.9447441860465123,3.213860465116279)-- (3.9839069767441835,3.8321860465116298);
\draw (3.9839069767441835,3.8321860465116298)-- (3.4886511627906955,1.198325581395351);
\draw (3.4886511627906955,1.198325581395351)-- (-1.44,0.58);
\draw [dash pattern=on 3pt off 3pt] (-0.9447441860465123,3.213860465116279)-- (1.0243255813953478,0.8891627906976756);
\begin{scriptsize}
\draw [fill=qqqqff] (-1.44,0.58) circle (2.5pt);
\draw[color=qqqqff] (-1.9816744186046515,0.8944186046511661) node {$A$};
\draw [fill=qqqqff] (3.4886511627906955,1.198325581395351) circle (2.5pt);
\draw[color=qqqqff] (3.89386046511627,1.0970232558139565) node {$B$};
\draw [fill=qqqqff] (3.9839069767441835,3.8321860465116298) circle (2.5pt);
\draw[color=qqqqff] (4.254046511627898,4.338697674418607) node {$C$};
\draw [fill=uuuuuu] (-0.9447441860465123,3.213860465116279) circle (1.5pt);
\draw[color=uuuuuu] (-1.3963720930232568,3.775906976744188) node {$D$};
\draw [fill=uuuuuu] (1.0243255813953478,0.8891627906976756) circle (1.5pt);
\draw[color=uuuuuu] (1.192465116279065,1.322139534883724) node {$M$};
\draw [fill=uuuuuu] (0.36796899224806134,1.6640620155038768) circle (1.5pt);
\draw[color=uuuuuu] (0.40455813953487996,2.2451162790697703) node {$K$};
\end{scriptsize}
\end{tikzpicture}
\end{center}
\begin{sol}
\marginpar{1} Le point $K$ est le centre de gravité de $ABD$. 

\marginpar{1} On en déduit que $(AK)$ est une autre médiane de ce triangle, et donc qu'elle coupe $[BD]$ en son milieu $O$. 

\marginpar{1}  Et donc, $(AK)$ contient le centre du parallélogramme $ABCD$, c'est donc une diagonale et donc elle contient $C$.
\begin{center}
\definecolor{uuuuuu}{rgb}{0.26666666666666666,0.26666666666666666,0.26666666666666666}
\definecolor{qqqqff}{rgb}{0.,0.,1.}
\begin{tikzpicture}[line cap=round,line join=round,>=triangle 45,x=1.0cm,y=1.0cm]
\clip(-2.5444651162790692,0.027720930232561652) rectangle (4.839348837209292,4.7551627906976766);
\draw (-1.44,0.58)-- (-0.9447441860465123,3.213860465116279);
\draw (-0.9447441860465123,3.213860465116279)-- (3.9839069767441835,3.8321860465116298);
\draw (3.9839069767441835,3.8321860465116298)-- (3.4886511627906955,1.198325581395351);
\draw (3.4886511627906955,1.198325581395351)-- (-1.44,0.58);
\draw [dash pattern=on 3pt off 3pt] (-0.9447441860465123,3.213860465116279)-- (1.0243255813953478,0.8891627906976756);
\draw [dash pattern=on 3pt off 3pt] (-0.9447441860465123,3.213860465116279)-- (3.4886511627906955,1.198325581395351);
\draw [dash pattern=on 3pt off 3pt] (-1.44,0.58)-- (3.9839069767441835,3.8321860465116298);
\begin{scriptsize}
\draw [fill=qqqqff] (-1.44,0.58) circle (2.5pt);
\draw[color=qqqqff] (-1.9816744186046515,0.8944186046511661) node {$A$};
\draw [fill=qqqqff] (3.4886511627906955,1.198325581395351) circle (2.5pt);
\draw[color=qqqqff] (3.89386046511627,1.0970232558139565) node {$B$};
\draw [fill=qqqqff] (3.9839069767441835,3.8321860465116298) circle (2.5pt);
\draw[color=qqqqff] (4.254046511627898,4.338697674418607) node {$C$};
\draw [fill=uuuuuu] (-0.9447441860465123,3.213860465116279) circle (1.5pt);
\draw[color=uuuuuu] (-1.3963720930232568,3.775906976744188) node {$D$};
\draw [fill=uuuuuu] (1.0243255813953478,0.8891627906976756) circle (1.5pt);
\draw[color=uuuuuu] (1.192465116279065,1.322139534883724) node {$M$};
\draw [fill=uuuuuu] (0.36796899224806134,1.6640620155038768) circle (1.5pt);
\draw[color=uuuuuu] (0.40455813953487996,2.2451162790697703) node {$K$};
\draw [fill=uuuuuu] (1.2719534883720915,2.206093023255815) circle (1.5pt);
\draw[color=uuuuuu] (1.2825116279069717,2.8079069767441887) node {$O$};
\end{scriptsize}
\end{tikzpicture}
\end{center}
\end{sol}
\end{exo}



\begin{exo}[Subdivision en trois \faLightbulbO]% exo7
On donne un segment $[AB]$. Le diviser en trois parts égales.
\begin{hint}
Il n'y a pas besoin du théorème de Thalès  : les médianes d'un triangle se croisent en un point situé aux deux-tiers des sommets.
\end{hint}
\end{exo} 

\begin{exo}[Triangles de même aire]
Démontrer qu'une médiane divise un triangle en deux triangles de même aire, et que les trois médianes partagent le triangle en six petits triangles de même aire.
\begin{sol}
Les deux triangles coupés par la médiane ont la même hauteur et des bases de même longueur (égales à la moitié de la base du triangle d'origine).
\end{sol}
\end{exo}



\begin{exo}[Triangle avec $AB=2BC$ $\heartsuit$\faLightbulbO]
(Complexité : 4, initiative)
\begin{prerequis}
Centre de gravité
\end{prerequis}
% joli
% Source : http://irem-fpb.univ-lyon1.fr/feuillesprobleme/feuille10/enonces/truc5.html
% Tags :  médiane, bissectrice
Soit $ABC$ un triangle avec $AB = 2 BC$ et $M$ un point de $[AC]$ tel que $AM = 2 MC$. Comparer les angles $\widehat{ABM}$ et $\widehat{MBC}$.
\begin{hint}
Où se trouve le point $M$ sur le segment $[AC]$ ?
\end{hint}
\begin{sol}

\begin{center}
\definecolor{qqwuqq}{rgb}{0.,0.39215686274509803,0.}
\definecolor{xdxdff}{rgb}{0.49019607843137253,0.49019607843137253,1.}
\definecolor{uuuuuu}{rgb}{0.26666666666666666,0.26666666666666666,0.26666666666666666}
\definecolor{qqqqff}{rgb}{0.,0.,1.}
\begin{tikzpicture}[line cap=round,line join=round,>=triangle 45,x=1.0cm,y=1.0cm]
\clip(-3.6,-1.06) rectangle (4.3,4.2);
\draw [shift={(1.1,3.42)},color=qqwuqq,fill=qqwuqq,fill opacity=0.1] (0,0) -- (-157.7619672844355:0.6) arc (-157.7619672844355:-107.95975501222517:0.6) -- cycle;
\draw [shift={(1.1,3.42)},color=qqwuqq,fill=qqwuqq,fill opacity=0.1] (0,0) -- (-107.95975501222517:0.6) arc (-107.95975501222517:-58.15754274001482:0.6) -- cycle;
\draw (-2.96,1.76)-- (1.1,3.42);
\draw (2.257061141239912,1.556935450545904)-- (1.1,3.42);
\draw (2.257061141239912,1.556935450545904)-- (-2.96,1.76);
\draw [dash pattern=on 5pt off 5pt] (-0.93,2.59)-- (3.414122282479824,-0.30612909890819173);
\draw [dash pattern=on 5pt off 5pt] (2.257061141239912,1.556935450545904)-- (3.414122282479824,-0.30612909890819173);
\draw [dash pattern=on 5pt off 5pt] (3.414122282479824,-0.30612909890819173)-- (-2.96,1.76);
\draw (1.1,3.42)-- (0.5180407608266075,1.6246236336972695);
\begin{scriptsize}
\draw [fill=qqqqff] (-2.96,1.76) circle (2.5pt);
\draw[color=qqqqff] (-2.82,2.12) node {$A$};
\draw [fill=qqqqff] (1.1,3.42) circle (2.5pt);
\draw[color=qqqqff] (1.24,3.78) node {$B$};
%\draw [fill=uuuuuu] (-0.93,2.59) circle (1.5pt);
%\draw[color=uuuuuu] (-1.32,2.92) node {$I$};
\draw [fill=xdxdff] (2.257061141239912,1.556935450545904) circle (2.5pt);
\draw[color=xdxdff] (2.4,1.92) node {$C$};
\draw [fill=qqqqff] (3.414122282479824,-0.30612909890819173) circle (2.5pt);
\draw[color=qqqqff] (3.56,0.06) node {$D$};
\draw [fill=uuuuuu] (0.5180407608266075,1.6246236336972695) circle (1.5pt);
\draw[color=uuuuuu] (0.08,1.3) node {$M$};
\end{scriptsize}
\end{tikzpicture}
\end{center}
\marginpar{1} Soit $D$ le symétrique de $B$ par rapport à $C$.

\marginpar{1} Alors $AB=BD$ donc $ABD$ est isocèle en $B$.

\marginpar{1} De plus, $M$ est son centre de gravité puisqu'il est aux deux tiers d'une des médianes.

\marginpar{1} La droite $(BM)$ est donc la médiane issue de $B$, et donc également la bissectrice de $ABD$ issue de $B$. On en déduit que
\[\widehat{ABM} = \widehat{MBC}.\]
\end{sol}
\end{exo}



%%%%%%%%%%%%%%%%%%%%%%%%%%
\section{Classe de 4ème}
%%%%%%%%%%%%%%%%%%%%%%%%%%


\subsection{Calcul littéral}

\begin{exo}[Somme d'entiers consécutifs]
[Éliminatoires de la coupe Animath d'automne  2017]
% http://www.animath.fr/IMG/pdf/coupe_animath_automne_17_sujet.pdf
On peut écrire $225$ comme la somme de trois entiers consécutifs : $225 = 74+75+76$.
\begin{enumerate}
\item Peut-on l'écrire comme la somme de cinq nombres consécutifs ?
\item Peut-on l'écrire comme la somme de quatre nombres consécutifs ?
\end{enumerate}
\begin{sol}
\begin{enumerate}
\item C'est le cas si et seulement s'il existe un entier $n$ tel que $225 = n+(n+1)+...+(n+4) = 5n + (0+1+2+3+4) = 5n+10$. Or, cette équation est équivalente à $215=5n$ donc à $n = 215/5$. Ici, comme $215$ est bien divisible par $5$, il y a bien une solution entière, c'est $n=215/5=43$. On a donc $225=43+44+45+46+47$.
\item Le même raisonnement aboutit à $219=4n$, qui n'a pas de solution entière car $219$ n'est pas pair, et donc a fortiori n'est pas divisible par quatre.
\end{enumerate}
Pour un corrigé différent, voir \url{http://www.animath.fr/IMG/pdf/coupe_animath_automne_17_corrige.pdf}

Ouverture pour le lycée : on peut se demander pour quels $k$ est-ce que $225$ est la somme de $k$ entiers consécutifs.
\end{sol}
\end{exo}

\begin{exo}[Foot à trois]
[Coupe Animath d'automne 2017]
%http://www.animath.fr/IMG/pdf/coupe_animath_automne_17_sujet.pdf
Le foot à trois  personnes se joue en phases successives. Un joueur est gardien pendant que les deux autres, appelés \og joueurs de champ\fg, tentent de marquer un but. Dès qu'un joueur marque, la phase se termine et il devient gardien pour la phase suivante. Amandine, Bobby et Charles jouent à ce jeu. La partie terminée, ils se souviennent qu'Amandine était $12$ fois joueuse de champ, Bobby $21$ joueur de champ, et Charles $8$ fois gardien.
\begin{enumerate}
\item Combien y a-t-il eu de phases au total ?
\item Qui a marqué le sixième but ?
\end{enumerate}
\begin{sol}
On trouve, en résolvant un système linéaire ou bien en étant un peu astucieux, qu'il y a eu 25 phases. Amandine a donc été $13$ fois gardienne, et comme le gardien change à chaque phase, elle a forcément été gardienne lors des phases $1$, $3$, $5$, ... et $25$. Ceci montre que c'est elle qui a marqué le sixième but, puisqu'elle était la gardienne durant la septième phase.

Pour un corrigé plus détaillé, voir \url{http://www.animath.fr/IMG/pdf/coupe_animath_automne_17_corrige.pdf}.
\end{sol}
\end{exo}


\subsection{Translations}

\begin{exo}[Corde de longueur fixée]
% translations
Soit $\mathcal C$ un cercle et $D$ une droite. Construire une droite parallèle à $D$ coupant le cercle $\mathcal C$ en deux points situés à une distance $a$ donnée (inférieure au diamètre).

\begin{hint}
Considérer la translation $\tau$ de distance $a$ suivant la direction de la droite.
\end{hint}      
\begin{sol} 
Appliquer la translation au cercle. (Si on n'a pas donné le centre du cercle, commencer par construire le centre.)

Les points d'intersection des deux cercles fournissent les (ou la) solutions du problème.
\end{sol}  
\end{exo}  


\begin{exo}[Alignement] % exo7
% tags : translation

Soit $ABCD$ un rectangle et $M$ un point du plan.

On note $C’$ le projeté orthogonal de $C$ sur $(AM)$,
$D’$  le projeté orthogonal de $D$ sur $(BM)$ et 
$M’$  le projeté orthogonal de $M$ sur $(AB)$. 
Enfin, on note $I$ le point d'intersection des droites $(CC’)$ et $(DD’)$.

Montrer que les points $M$, $M’$ et $I$ sont alignés.

\begin{hint}   
Considérer la translation de vecteur $\overrightarrow{CB}$ et l'image de $I$ par cette translation.
\end{hint}

\end{exo}  

\begin{exo}[Pappus affine] % exo7
% Source :  Audin par exemple 
% translation, parallélogrammes
Soient $D$ et $D'$ deux droites  parallèles. Soient $A, B, C$ trois points sur $D$, et $A'$, $B'$ et $C'$ trois points sur $D'$. Si $(AB') // (BC')$ et $(BA') // (CB')$, alors $(AA') // (CC')$.
\begin{hint}
Des translations  commutent
\end{hint}
\begin{sol}     
Soit $O$ le pt d'intersection. On note $\phi$ la translation qui envoie $A$ sur $B$, et $\psi$ celle qui envoie $B$ sur $C$. Alors $\phi\psi = \psi\phi$. L'image de $A$ est $C$ et l'image de $A'$ est $C'$, d'où le  parallélisme demandé.
\'Ecrire la solution de façon élémentaire avec des parallélogrammes.
\end{sol}
\end{exo}  

%%%%%%%%%%%%%%%%%
\subsection{Rotations}


\begin{exo}[Carré d'aire moitié]
%[Cloître]
% Source : http://debart.pagesperso-orange.fr/college/carre_college.html
% Tags :  carré
On se donne un carré, et on cherche à construire un carré de même centre, aux côtés parallèles, et d'aire deux fois plus petite, comme ci-dessous:

\begin{center}
%
\begin{tikzpicture}[line cap=round,line join=round,>=triangle 45,x=1.0cm,y=1.0cm]
\clip(-1.76,-0.54) rectangle (2.74,4.08);
%\draw(0.47,1.75) circle (1.7800280896660026cm);
\draw [dash pattern=on 5pt off 5pt] (-0.79572113832392,0.49842099729981115)-- (-0.7815790027001891,3.0157211383239195);
\draw [dash pattern=on 5pt off 5pt] (-0.7815790027001891,3.0157211383239195)-- (1.7357211383239195,3.0015790027001885);
\draw [dash pattern=on 5pt off 5pt] (1.7357211383239195,3.0015790027001885)-- (1.7215790027001878,0.4842788616760799);
\draw [dash pattern=on 5pt off 5pt] (1.7215790027001878,0.4842788616760799)-- (-0.79572113832392,0.49842099729981115);
\draw (-1.32,-0.02)-- (-1.3,3.54);
\draw (-1.3,3.54)-- (2.26,3.52);
\draw (2.26,3.52)-- (2.24,-0.04);
\draw (2.24,-0.04)-- (-1.32,-0.02);
\end{tikzpicture}
%
\end{center}



\begin{enumerate}
\item (Question intermédiaire) Soit $\mathcal C$ un cercle. Montrer qu'un carré circonscrit au cercle a une aire deux fois plus grande qu'un carré inscrit dans le cercle.
\item En déduire une solution au problème initial.
\end{enumerate}
\begin{hint}
Faire tourner le carré circonscrit par rapport au carré inscrit.
\end{hint}
\begin{sol}


\begin{center}
\begin{tikzpicture}[line cap=round,line join=round,>=triangle 45,x=1.0cm,y=1.0cm]
\clip(-3.94,-0.88) rectangle (5.24,4.22);
\draw(0.47,1.75) circle (1.7800280896660026cm);
\draw [dash pattern=on 5pt off 5pt] (-0.79572113832392,0.49842099729981115)-- (-0.7815790027001891,3.0157211383239195);
\draw [dash pattern=on 5pt off 5pt] (-0.7815790027001891,3.0157211383239195)-- (1.7357211383239195,3.0015790027001885);
\draw [dash pattern=on 5pt off 5pt] (1.7357211383239195,3.0015790027001885)-- (1.7215790027001878,0.4842788616760799);
\draw [dash pattern=on 5pt off 5pt] (1.7215790027001878,0.4842788616760799)-- (-0.79572113832392,0.49842099729981115);
\draw (-1.32,-0.02)-- (-1.3,3.54);
\draw (-1.3,3.54)-- (2.26,3.52);
\draw (2.26,3.52)-- (2.24,-0.04);
\draw (2.24,-0.04)-- (-1.32,-0.02);
\draw (-1.31,1.76)-- (0.48,3.53);
\draw (0.48,3.53)-- (2.25,1.74);
\draw (2.25,1.74)-- (0.46,-0.03);
\draw (-1.31,1.76)-- (0.46,-0.03);
\end{tikzpicture}
\end{center}
\end{sol}
\end{exo}



\begin{exo}[Aire de l'intersection de deux carrés]
 % exo7


% source : Debart rotations au collège
% tags : rotations, collège
Soit $ABCD$ un carré de centre $O$, et $OPQR$ un second carré de même taille. Calculer l'aire de l'intersection de ces deux carrés en fonction de l'aire de $ABCD$.

\begin{center}
\definecolor{uuuuuu}{rgb}{0.26666666666666666,0.26666666666666666,0.26666666666666666}
\definecolor{zzttqq}{rgb}{0.6,0.2,0.}
\definecolor{qqqqff}{rgb}{0.,0.,1.}
\begin{tikzpicture}[line cap=round,line join=round,>=triangle 45,x=1.0cm,y=1.0cm]
\clip(-0.5,-3.58) rectangle (8,5.8);
\fill[color=zzttqq,fill=zzttqq,fill opacity=0.1] (0.,0.) -- (3.,0.) -- (3.,3.) -- (0.,3.) -- cycle;
\fill[color=zzttqq,fill=zzttqq,fill opacity=0.1] (1.5,1.5) -- (4.38,0.66) -- (5.22,3.54) -- (2.34,4.38) -- cycle;
\draw [color=zzttqq] (0.,0.)-- (3.,0.);
\draw [color=zzttqq] (3.,0.)-- (3.,3.);
\draw [color=zzttqq] (3.,3.)-- (0.,3.);
\draw [color=zzttqq] (0.,3.)-- (0.,0.);
\draw [color=zzttqq] (1.5,1.5)-- (4.38,0.66);
\draw [color=zzttqq] (4.38,0.66)-- (5.22,3.54);
\draw [color=zzttqq] (5.22,3.54)-- (2.34,4.38);
\draw [color=zzttqq] (2.34,4.38)-- (1.5,1.5);
\begin{scriptsize}
\draw [fill=qqqqff] (0.,0.) circle (2.5pt);
\draw[color=qqqqff] (0.14,0.36) node {$A$};
\draw [fill=qqqqff] (3.,0.) circle (2.5pt);
\draw[color=qqqqff] (3.14,0.36) node {$B$};
\draw [fill=uuuuuu] (3.,3.) circle (2.5pt);
\draw[color=uuuuuu] (3.14,3.36) node {$C$};
\draw [fill=uuuuuu] (0.,3.) circle (2.5pt);
\draw[color=uuuuuu] (0.14,3.36) node {$D$};
\draw [fill=uuuuuu] (1.5,1.5) circle (1.5pt);
\draw[color=uuuuuu] (1.08,1.64) node {$O$};
\draw [fill=qqqqff] (4.38,0.66) circle (2.5pt);
\draw[color=qqqqff] (4.74,0.82) node {$P$};
\draw [fill=uuuuuu] (5.22,3.54) circle (1.5pt);
\draw[color=uuuuuu] (5.36,3.82) node {$Q$};
\draw [fill=uuuuuu] (2.34,4.38) circle (1.5pt);
\draw[color=uuuuuu] (2.48,4.66) node {$R$};
\end{scriptsize}
\end{tikzpicture}
\end{center}

\begin{hint}   
L'aire de l'intersection vaut le quart de l'aire du carré $ABCD$.
\end{hint}      
\begin{sol} 
Tracer une figure en prolongeant les segments $[OP]$ et $[OR]$, et considérer une rotation de centre $O$ et d'angle $\pi/2$.
\end{sol}  
\end{exo}  

\begin{exo}[Alignement] % exo7


% source : Debart rotations aux collège
% rotations, collège
Soit $ABCD$ un carré de centre $O$, et $M$ un point sur $(AB)$. À partir de $M$, on construit le triangle isocèle $OMN$, rectangle en $O$. Montrer que les points $B$, $C$ et $N$ sont alignés.

\begin{hint}   
Considérer la rotation de centre $O$ et d'angle $\pi/2$.
\end{hint}

\end{exo}  


\begin{exo}[Pseudo-carré] % exo7
% tags : rotation, (ou nombres complexes, plus tard)

Soit $BOA$ un triangle indirect quelconque, $OAC$ et $OEB$  deux triangles rectangles isocèles en $O$ directs.

Montrer que $ACEB$ est un pseudo-carré, c'est-à-dire que les droites $(AE)$ et $(BC)$ sont perpendiculaires et que $BC = AE$.

\begin{hint}   
Considérer la rotation de centre $O$ et d'angle $\pi/2$.
\end{hint}
\end{exo}  



\begin{exo}[Octogone régulier]
%rotation, octogone, analyse-synthèse
Construire  un octogone régulier inscrit dans un carré donné (c'est-à-dire un octogone dont quatre des huit cotés s'appuient sur les cotés du carré).

\begin{hint}   
Procéder par analyse-synthèse et considérer des rotations.
\end{hint}
\begin{sol}
On construit l'image du carré par une rotation d'angle $\pi/4$. Les points d'intersection des deux carrés forment un octogone régulier qui répond à la question.
\end{sol}
\end{exo}



\begin{exo}[Preuve d'Euclide du th. de Pythagore]
% rotations, aire = base x hauteur
% on peut utiliser le cas d'égalité des triangles à la place des rotations mais ici les rotations se voient bien (angle pi/2)
Soit $ABC$ un triangle rectangle en $A$. On considère trois carrés appuyés sur les côtés de $ABC$ comme sur la figure ci-dessous:
\begin{center}
\definecolor{qqwuqq}{rgb}{0.,0.39215686274509803,0.}
\definecolor{uuuuuu}{rgb}{0.26666666666666666,0.26666666666666666,0.26666666666666666}
\definecolor{zzttqq}{rgb}{0.6,0.2,0.}
\definecolor{xdxdff}{rgb}{0.49019607843137253,0.49019607843137253,1.}
\definecolor{qqqqff}{rgb}{0.,0.,1.}
\begin{tikzpicture}[line cap=round,line join=round,>=triangle 45,x=1.0cm,y=1.0cm]
\clip(-6.176698268643814,-5.426018961186285) rectangle (6.017753828701081,2.4386705561407784);
\fill[color=zzttqq,fill=zzttqq,fill opacity=0.1] (0.2447370553439779,-0.3641709207789719) -- (-0.5105558292674985,-3.7252242573000416) -- (2.8504975072535705,-4.480517141911518) -- (3.605790391865048,-1.1194638053904489) -- cycle;
\fill[color=zzttqq,fill=zzttqq,fill opacity=0.1] (-1.6168516867892904,0.054163628015021004) -- (0.2447370553439779,-0.3641709207789719) -- (0.6630716041379707,1.4974178213542957) -- (-1.198517137995297,1.9157523701482893) -- cycle;
\fill[color=zzttqq,fill=zzttqq,fill opacity=0.1] (-0.5105558292674985,-3.7252242573000416) -- (-1.6168516867892904,0.054163628015021004) -- (-5.396239572104353,-1.052132229506771) -- (-4.289943714582561,-4.831520114821833) -- cycle;
\draw[color=qqwuqq,fill=qqwuqq,fill opacity=0.1] (-0.18817755183130794,-0.2668867393912673) -- (-0.2854617332190126,-0.6998013465665531) -- (0.14745287395627327,-0.7970855279542577) -- (0.2447370553439779,-0.3641709207789719) -- cycle; 
\draw[color=qqwuqq,fill=qqwuqq,fill opacity=0.1] (-1.2324904678037198,-1.258912006086213) -- (-0.8066486774534186,-1.1342603368424438) -- (-0.9313003466971879,-0.7084185464921426) -- (-1.357142137047489,-0.8330702157359118) -- cycle; 
\draw [line width=1.6pt] (0.2447370553439779,-0.3641709207789719)-- (-1.6168516867892904,0.054163628015021004);
\draw [line width=1.6pt] (-1.6168516867892904,0.054163628015021004)-- (-0.5105558292674985,-3.7252242573000416);
\draw [line width=1.6pt] (-0.5105558292674985,-3.7252242573000416)-- (0.2447370553439779,-0.3641709207789719);
\draw [color=zzttqq] (0.2447370553439779,-0.3641709207789719)-- (-0.5105558292674985,-3.7252242573000416);
\draw [color=zzttqq] (-0.5105558292674985,-3.7252242573000416)-- (2.8504975072535705,-4.480517141911518);
\draw [color=zzttqq] (2.8504975072535705,-4.480517141911518)-- (3.605790391865048,-1.1194638053904489);
\draw [color=zzttqq] (3.605790391865048,-1.1194638053904489)-- (0.2447370553439779,-0.3641709207789719);
\draw [color=zzttqq] (-1.6168516867892904,0.054163628015021004)-- (0.2447370553439779,-0.3641709207789719);
\draw [color=zzttqq] (0.2447370553439779,-0.3641709207789719)-- (0.6630716041379707,1.4974178213542957);
\draw [color=zzttqq] (0.6630716041379707,1.4974178213542957)-- (-1.198517137995297,1.9157523701482893);
\draw [color=zzttqq] (-1.198517137995297,1.9157523701482893)-- (-1.6168516867892904,0.054163628015021004);
\draw [color=zzttqq] (-0.5105558292674985,-3.7252242573000416)-- (-1.6168516867892904,0.054163628015021004);
\draw [color=zzttqq] (-1.6168516867892904,0.054163628015021004)-- (-5.396239572104353,-1.052132229506771);
\draw [color=zzttqq] (-5.396239572104353,-1.052132229506771)-- (-4.289943714582561,-4.831520114821833);
\draw [color=zzttqq] (-4.289943714582561,-4.831520114821833)-- (-0.5105558292674985,-3.7252242573000416);
\draw [dash pattern=on 3pt off 3pt,domain=-6.176698268643814:6.017753828701081] plot(\x,{(-1.6470947566852019--1.1062958575217918*\x)/3.7793878853150624});
\begin{scriptsize}
\draw [fill=qqqqff] (0.2447370553439779,-0.3641709207789719) circle (2.5pt);
\draw[color=qqqqff] (0.09831996326608039,0.13783053777381865) node {$A$};
\draw [fill=qqqqff] (-1.6168516867892904,0.054163628015021004) circle (2.5pt);
\draw[color=qqqqff] (-2.139769872781782,0.32608108473111536) node {$B$};
\draw [fill=xdxdff] (-0.5105558292674985,-3.7252242573000416) circle (2.5pt);
\draw[color=xdxdff] (-0.5500985873646087,-4.003681495286709) node {$C$};
\draw [fill=uuuuuu] (2.8504975072535705,-4.480517141911518) circle (2.5pt);
\draw[color=uuuuuu] (3.131245442022529,-4.254682224563105) node {$D$};
\draw [fill=uuuuuu] (3.605790391865048,-1.1194638053904489) circle (2.5pt);
\draw[color=uuuuuu] (3.7587472652135188,-0.7406720146935659) node {$E$};
\draw [fill=uuuuuu] (0.6630716041379707,1.4974178213542957) circle (2.5pt);
\draw[color=uuuuuu] (0.8094886962158684,1.8739189152688882) node {$F$};
\draw [fill=uuuuuu] (-1.198517137995297,1.9157523701482893) circle (2.5pt);
\draw[color=uuuuuu] (-1.3240175026334955,2.2713367366231814) node {$G$};
\draw [fill=uuuuuu] (-5.396239572104353,-1.052132229506771) circle (2.5pt);
\draw[color=uuuuuu] (-5.9047808119277185,-0.7825054695729652) node {$H$};
\draw [fill=uuuuuu] (-4.289943714582561,-4.831520114821833) circle (2.5pt);
\draw[color=uuuuuu] (-4.712527347864839,-4.589349863598299) node {$I$};
\draw [fill=uuuuuu] (-1.357142137047489,-0.8330702157359118) circle (1.5pt);
\draw[color=uuuuuu] (-1.4495178672716935,-1.2217567458066576) node {$J$};
\draw [fill=uuuuuu] (-5.136530022362552,-1.9393660732577038) circle (1.5pt);
\draw[color=uuuuuu] (-5.277278988736729,-2.288509845231339) node {$K$};
\end{scriptsize}
\end{tikzpicture}
\end{center}
\begin{enumerate}
\item Montrer que $BGF$ et $BJH$ ont même aire.
\item De même, montrer que $CDE$ et $CIK$ ont même aire.
\item En déduire que $AB^2+AC^2=BC^2$.
\end{enumerate}
\begin{hint}
Utiliser des rotations.
\end{hint}
\begin{sol}
Les triangles $BGF$ et $BGC$ ont même aire. Ensuite, les triangles $BGC$ et $BAH$ ont même aire car ils se déduisent l'un de l'autre par rotation de centre $B$ et d'angle $\pi/2$. (On peut aussi utiliser le critère d'égalité des triangles et regarder les angles. Mais il est clair que la rotation fixe $B$, et envoie $G$ sur $A$, et $C$ sur $H$.)

Ensuite, on remarque que les triangles $BAH$ et $BJH$ ont même aire.

On fait de même pour le second triangle.

Ceci permet de conclure, en multipliant les aires par deux,  que le carré $BAFG$ a la même aire que le rectangle $BHKJ$, et que le carré $ACDE$ a même aire que le rectangle $CIKJ$, d'où le résultat.

C'est la preuve que l'on trouve dans Euclide, prop. 37 du libre I.
\end{sol}
\end{exo}



%----------
\begin{exo}[Triangle inscrit dans un carré]
% source : Debart "rotation au collège"
% tags: collège, rotation
Sur les côtés $[AB]$ et $[BC]$ d'un carré direct $ABCD$, on place des points $M$ et $N$ vérifiant $AM = BN$. Soit $H$ le point d'intersection des droites $(AN)$ et $(CM)$. Montrer que $H$ est l'orthocentre du triangle $DMN$.
\begin{center}
\definecolor{xdxdff}{rgb}{0.49019607843137253,0.49019607843137253,1.}
\definecolor{uuuuuu}{rgb}{0.26666666666666666,0.26666666666666666,0.26666666666666666}
\definecolor{qqqqff}{rgb}{0.,0.,1.}
\begin{tikzpicture}[line cap=round,line join=round,>=triangle 45,x=1.0cm,y=1.0cm]
\clip(-2.5,-0.38) rectangle (2.82,5.02);
\draw (-1.62,0.72)-- (-1.6,4.32);
\draw (-1.6,4.32)-- (2.,4.3);
\draw (2.,4.3)-- (1.98,0.7);
\draw (1.98,0.7)-- (-1.62,0.72);
\draw (-1.62,0.72)-- (1.987293205543007,2.012776997741286);
\draw (-0.3072230022587146,0.7127067944569928)-- (2.,4.3);
\begin{scriptsize}
\draw [fill=qqqqff] (-1.62,0.72) circle (2.5pt);
\draw[color=qqqqff] (-2.02,0.49) node {$A$};
\draw [fill=qqqqff] (1.98,0.7) circle (2.5pt);
\draw[color=qqqqff] (2.16,0.37) node {$B$};
\draw [fill=uuuuuu] (2.,4.3) circle (2.5pt);
\draw[color=uuuuuu] (2.14,4.67) node {$C$};
\draw [fill=uuuuuu] (-1.6,4.32) circle (2.5pt);
\draw[color=uuuuuu] (-2.02,4.69) node {$D$};
\draw [fill=xdxdff] (-0.3072230022587146,0.7127067944569928) circle (2.5pt);
\draw[color=xdxdff] (-0.36,0.37) node {$M$};
\draw [fill=uuuuuu] (1.987293205543007,2.012776997741286) circle (1.5pt);
\draw[color=uuuuuu] (2.12,2.31) node {$N$};
\draw [fill=uuuuuu] (0.09210166251354221,1.333580743503478) circle (1.5pt);
\draw[color=uuuuuu] (-0.22,1.83) node {$H$};
\end{scriptsize}
\end{tikzpicture}
\end{center} 
\begin{sol}
Pour montrer que $H$ est l'orthocentre du triangle $DMN$, il suffit de montrer que  $(AN)$ et $(CM)$ sont des hauteurs de ce triangle : leur point d'intersection $H$ sera alors l'orthocentre.\\

Soit $\rho$ la rotation de centre $O$ (le centre du carré) et d'angle $\pi/2$. Par définition d'un carré direct, on a $\rho(A)=B$, $\rho(B)=C$, $\rho(C)=D$ et $\rho(D)=A$.

On a de plus \underline{$\rho(M)=N$}. En effet, comme $M \in [AB]$, on a $\rho(M) \in [\rho(A)\rho(B)] = [BC]$, et d'autre part, comme $\rho$ est une isométrie, on a $AM = \rho(A)\rho(M) =  B\rho(M)$. Or il n'y a qu'un point sur $[BC]$ à distance $AM$ de $B$, et d'après l'énoncé c'est $N$.

La rotation $\rho$ envoie donc le triangle $DAM$ sur $ABN$. Comme c'est une rotation d'angle $\pi/2$, on en déduit que $(DM)\bot (AN)$ et donc que $(AN)$ est une hauteur de $DMN$. On procède de même pour la deuxième hauteur.

\emph{Remarque: on peut rédiger la solution sans rotations, juste en utilisant des angles complémentaires, mais c'est plus laborieux et moins éclairant, donc (fortement) déconseillé.}
\end{sol}
\end{exo}




\begin{exo}[Angle et distance] % exo7
% tags : rotation
Soit $BOA$ un triangle indirect isocèle en $O$, $OAC$ et $OEB$  deux triangles équilatéraux directs.

Montrer que $BC = AE$ et vérifier que l'angle des droites $(BC)$ et $(AE)$ est de $\pi/3$.
\begin{hint}   
Considérer la rotation de centre $O$ et d'angle $\pi/3$.
\end{hint}
\end{exo}  

\begin{exo}[Triangle rectangle isocèle]
% exo7
% pas génial
Soit $ABCD$ un carré, $E$ le symétrique de $C$ par rapport à $D$, $I$  le milieu de $[BC]$ et $J$ le milieu de $[DE]$.

Montrer que le triangle $AIJ$ est rectangle isocèle en $A$.

\begin{hint}   
Considérer  une rotation de centre $A$.
\end{hint}
\end{exo}  


\begin{exo}[Triangle équilatéral sur trois droites \faCogs][Difficile] % exo7
% tags : rotation, dur, utile
On considère trois droites parallèles $D_1$, $D_2$ et $D_3$. Construire un triangle équilatéral dont les sommets appartiennent respectivement à $D_1$, $D_2$ et $D_3$. 
\begin{hint}   
Si $ABC$ est un tel triangle, considérer les rotations d'angles $\pm \pi/3$ et centrées sur les sommets. Déterminer les images des différents points et droites par ces rotations.
\end{hint}
\end{exo}  



\begin{exo}[Deux triangles isocèles rectangles] % exo7
% rotations
Soient $AOB$ et $COD$ deux triangles directs, isocèles rectangles en $O$. Soient $I$, $J$, $K$ et $L$ les milieux des segments $[AB]$, $[BC]$, $[CD]$ et $[DA]$.
\begin{center}
\includegraphics{images/img007167-1}
\end{center}
\begin{enumerate}
\item Montrer que $(AC)\bot (BD)$ et que $AC=BD$.
\item Montrer que $IJKL$ est un carré. 
\end{enumerate}


\begin{sol}
\begin{enumerate}
\item Soit $\rho$ la rotation de centre $O$ et d'angle $\pi/2$. D'après l'énoncé, on a $\rho(B)=A$ et $\rho(D)=C$. Donc $[AC]$ est l'image de $[AD]$ par $\rho$, d'où on déduit que $(AC)\bot (BD)$ et que $AC=BD$.

\item Le quadrilatère $IJKL$ est toujours un parallélogramme, même sans hypothèses sur $ABCD$ (c'est le théorème de Varignon). En effet, par le théorème de Thalès (ou simplement le théorème des milieux) : 
\[ \overrightarrow{IL} = \frac12 \overrightarrow{BD} = \overrightarrow{JK}.\]
Ceci signifie que les côtés $[IL]$ et $[JK]$ ont même longueur et sont parallèles, donc $IJKL$ est un parallélogramme.

Pour voir que c'est un carré, remarquons qu'on montre de même que  
\[ \overrightarrow{IJ} = \frac12 \overrightarrow{AC} = \overrightarrow{LK},\]
et d'après la première question, $(AC)\bot (BD)$ et que $AC=BD$, donc $IJKL$ est un parallélogramme ayant un angle droit et des cotés consécutifs de même longueur. C'est donc un carré.
\end{enumerate}
\end{sol}
\end{exo}


\begin{exo}[Construction du centre d'une rotation]
Soit $\rho$ une rotation du plan. On donne deux points $A$ et $B$, ainsi que leurs images $\rho(A)$ et $\rho(B)$. Construire le centre de la rotation, en distinguant les cas.
\end{exo}

\begin{exo}[Carré partagé en deux \faLightbulbO]
% initiative
% fait une bonne "preuve sans mots"
% compléter en un gros carré, symétrie centrale
Soit $ABCD$ un carré et $E$ un point à l'extérieur du carré tel que $AEB$ soit rectangle en $E$. Soit $\mathcal D$ la bissectrice de l'angle $\widehat{AEB}$. Montrer qu'elle coupe le carré $BCD$ en deux parties de même aire.
\begin{hint}
Il suffit de montrer que cette bissectrice passe par le centre du carré.
\end{hint}
\begin{sol}
Considérer une rotation de centre $O$ et d'angle $\pi/4$, et les images itérées de $E$ par cette rotation. Ce sont les sommets d'un carré.
\end{sol}
\end{exo}


\subsection{Rotations et translations}


\begin{exo}[Carré invisible \faCogs][difficile]
% tags : rotation, translation, difficile
On considère un carré $ABCD$ et on place quatre points $E$, $F$, $G$, et $H$ sur les côtés de ce carré, en-dehors des sommets. Puis, on efface le carré. En considérant une rotation et une translation, reconstruire le carré.

\begin{hint}   
Il suffit de construire une des droites d'appui du carré. Pour cela, il suffit de construire un deuxième point sur cette droite.
\end{hint}
\begin{sol}
Voici une solution dans une configuration \og générique\fg.

Commençons par analyser le problème. On suppose que $ABCD$ est direct et que $E\in [AB]$, $F \in [BC]$ etc.
Considérons la rotation de centre $O$ (le centre du carré), et d'angle $\pi/4$. L'image du segment $[EG]$ est un segment $[E'G']$, que l'on suppose distinct de $[FH]$. En fait, on suppose pour simplifier $E'\neq F$. Considérons alors la translation de vecteur $\overrightarrow{E'F}$. Elle envoie $G'$ sur un point $G''$ appartenant à la droite $(DA)$. Si on suppose que ce point est différent de $H$, alors on a $(DA) = (G''H)$. 

Voici comment obtenir le point $G''$. On construit la perpendiculaire à $(EG)$ passant par $F$, et sur cette droite, on place le point $G''$ tel que $FG''=EG$ et $(\overrightarrow{EG},\overrightarrow{FG''}) = \pi/2$. Ceci permet de tracer la droite $(HG'')$ c'est-à-dire $(AD)$.

\begin{center}
\includegraphics{images/img007104-1}
\end{center}

On projette ensuite les points $E$ et $G$ sur cette droite, ce qui donne $A$ et $D$. On peut ensuite terminer la construction du carré.

Il existe d'autres solutions qui utilisent le théorème de l'angle au centre.
\end{sol} 
\end{exo}  


\subsection{Pythagore}

\begin{exo}[Aire maximale à deux côtés fixés]
% utile pour l'inégalité isopérimétrique pour des polygones : c'est utilisé pour ontrer que les points doivent être sur un cercle.
Soient $a$ et $b$ deux longueurs fixées. Déterminer les triangles $ABC$ avec $BC=a$ et $CA=b$, et qui sont d'aire maximale.
\begin{hint}
Ce sont les deux triangles rectangles en $C$.
\end{hint}
\begin{sol}
On peut supposer, sans perte de généralité, que $B$ et $C$ sont fixes et vérifient $BC=a$. On cherche alors le point $A$ tel que $CA=b$, tel que l'aire de $ABC$ soit maximale. 

A priori, le point $A$ doit appartenir au cercle de centre $C$ et de rayon $b$.

Si $A$ est sur la perpendiculaire à $(BC)$ passant par $C$, alors l'aire est $\mathcal A = BC\cdot CA = ab$.

Si $A'$ est un autre point du cercle, en notant $H$ son projeté orthogonal sur $(BC)$, l'aire est $BC\cdot AH$ (base fois hauteur), or par Pythagore dans le triangle rectangle $AHC$ rectangle en $H$, on a $A'H \leq A'C=AC$.

Note : on peut aussi calculer l'aire avec un sinus et utiliser que $\sin\theta$ est maximal pour $\theta=\pi/2$, mais ceci compliqué au niveau collège et de toute façon la preuve est celle donnée plus haut avec Pythagore.
\end{sol}
\end{exo}


\begin{exo}[Test d'entrée à l'OFM 2014]
\begin{prerequis}
Pythagore, triangles isocèles
\end{prerequis}
Soit $ABCD$ un carré. On suppose qu'il existe un point $E$ sur $[AD]$ et un point $F$ sur $[BC]$ vérifiant: $BE=EF=FD=1$. Quelle est l'aire du carré ?
\begin{sol}
(Note : l'aire est forcément inférieure à $1$, puisque par Pythagore $AB\leq EB=1$.)

De $EB=EF$ on tire que $EFB$ est isocèle en $E$, et on conclut de la même façon que $DFE$ est isocèle en $F$.

Soit $E'$ (respectivement $F'$) le pied de la hauteur de $BEF$ (resp. $DFE$) issue de $E$ (resp. $F$).

Alors comme ces deux hauteurs sont aussi des médianes, on a $BE'=E'F$ et d'autre part $EF'=F'D$.
D'autre part, comme $ABE'E$, $EE'FF'$ et $F'FCD$ sont des rectangles (ils ont trois angles droits), on a finalement:
\[ AE=EF'=F'D  =  BE'=E'F=FC \]
et ces quantités  valent donc le tiers du côté du carré.

On applique alors Pythagore dans un des triangles rectangles, et on obtient $AE=\frac{1}{\sqrt 10}$. On en déduit que le carré a une aire de $9/10$.

Voir la correction d'Animath, ainsi que des autres exercices de 2014, ici : \url{http://www.animath.fr/IMG/pdf/2014-10-test-OFM-corrige.pdf}


\end{sol}
\end{exo}

%%%%%%%%%%%%%%%%%%%%%%%%%%%%%%%%%%%
\section{Sorti du programme de 4ème en 2016}
%%%%%%%%%%%%%%%%%%%%%%%%%%%%%%%%%%%

%Conseil : ancien manuel de Sesamath, avec les exos qui vont bien


\subsection{Distance d'un point à une droite, projetés orthogonaux}

Ces notions sont relativement intuitives et n'utilisent que le théorème de Pythagore.

\begin{exo}[Projeté orthogonal et distance]
Soit $A$ un point et $\mathcal D$ une droite. Montrer que le point de $\mathcal D$ le plus proche de $A$ est le projeté orthogonal de $A$ sur $\mathcal D$, c'est-à-dire $A$ lui-même si $A\in \mathcal D$ et sinon, le  point $H$ vérifiant $(AH)\bot \mathcal D$.
\begin{hint} Utiliser le théorème de Pythagore.
\end{hint}
\begin{sol}
Pour tout autre point $H'$ de $\mathcal D$, on écrit Pythagore dans le triangle $AHH'$, ce qui donne $AH'^2=AH^2+HH'^2$, d'où on déduit que $AH' > AH$.
\end{sol}
\end{exo}

\begin{exo}[Une projection orthogonale diminue les distances]
Soit $\mathcal D$ une droite, et $A$, $B$ deux points du plan n'appartenant pas à $\mathcal D$.

Notons $A'$ (respectivement $B'$) le point tel que $(AA')\bot \mathcal D$ (respectivement $(BB')\bot \mathcal D$).

Montrer que $A'B' \leq BB'$.
\begin{hint}
Utiliser le théorème de Pythagore.
\end{hint}
\begin{sol}
Si $A'=B'$, alors le résultat est vrai.

Sinon, cela signifie que $(AA')$ et $(BB')$ sont deux droites différentes. Notons $H$ le projeté orthogonal de $A$ sur $(BB')$, c'est-à-dire le point tel que $(AH)\bot (BB')$.

Notons que $AHB'A'$ a trois angles droits donc est un rectangle, et donc en particulier $AH=A'B'$.

Si $H=B$, cela signifie que $AB=A'B'$ et on a fini.

Sinon, le triangle $ABH$ est rectangle en $H$ et d'après le théorème de Pythagore, on a $AB^2=BH^2+AH^2$. Comme $AH=A'B'$, ceci donne donc $AB^2=A'B'^2+BH^2$ donc $AB\geq A'B'$, ce qu'il fallait démontrer.

Note : avec des vecteurs et du produit scalaire, on n'a pas besoin de séparer les différents cas, les points peuvent être sur la droite, être confondus etc : une projection orthogonale diminue toujours les distances (au sens large).
\end{sol}
\end{exo}


\begin{exo}[Distances aux côtés \faLightbulbO]
\label{sommeDistances} % pris dans Oudompheng 
Soit $ABC$ un triangle équilatéral. Pour tout point $M$ à l'intérieur du triangle, on note $d = dist(M,[AB]) + dist(M,[BC]) + dist(M,[AC])$ la somme des distances de $M$ aux trois côtés. Montrer que $d$ ne dépend en fait pas du point $M$. 
\begin{hint} Méthodologie : essayer avec plusieurs points $M$. Que remarque-t-on ?
 

% autre solution, voir vieux TD, rotations pour se ramener à un seul côté ?
\end{hint}

\begin{sol}
\'Ecrire chacune des distances à l'aide d'aires de triangles. 

Ou bien dessiner les trois petits triangles, chaque distance est une hauteur d'un petit triangle équilatéral, faire tourner ces hauteurs.
\end{sol}
\end{exo} 




\begin{exo}[Minimisation]
%exo7
Soient $C$ et $C'$ deux cercles sécants de centres $O$ et $O'$, et $A$ un de leurs points d'intersection. Une droite $D$ passant par $A$ recoupe les deux cercles en $M$ et $M'$. Déterminer la position de la droite qui maximise la distance $MM'$ et calculer le maximum.

\begin{hint}   
Considérer $I$ et $J$ les milieux de $MA$ et $AM'$ ainsi qu'une projection orthogonale. Une telle projection réduit les distances.
\end{hint}      
\begin{sol} 
On a et $MM' = 2IJ$ et d'autre part $I$ et $J$ sont les projetés de $O$ et $O'$ sur la droite $D$, donc $IJ \leq OO'$ avec égalité ssi $(OO') // (IJ)$, dans ce cas le maximum est donc $2OO'$.
\end{sol}  
\end{exo}  

\subsection{Tangente à un cercle}

Les tangentes à un cercle semblent avoir disparu des derniers programmes. Pour le cours et de nombreux exercices de base, voir par exemple le Sésamath 4ème pré-2016.

\begin{exo}[Construction de cercles : DDP]
\label{DPPparallele}
Soit $\mathcal D$ une droite, et soient $A$ et $B$ deux points distincts hors de la droite tels que $\mathcal D // (AB)$.

Construire un cercle $\mathcal C$ passant par $A$ et $B$, et tangent à la droite $\mathcal D$.
\end{exo}



\begin{exo}[Construction de la tangente]
On donne un cercle $\mathcal C$ et un point $P$ sur le cercle. Tracer la tangente à $\mathcal C$ au point $P$.
\end{exo}

\begin{exo}[Construction de la tangente, bis]
\begin{prerequis} Triangle rectangle inscrit dans un demi-cercle.\end{prerequis}
On donne un cercle $\mathcal C$ et un point $P$ hors du cercle. Tracer la tangente à $\mathcal C$ passant par $P$.
\end{exo}


\begin{exo}[Construction de tangentes communes]
On donne deux cercles distincts de même rayon. Combien y a-t-il de tangentes communes aux deux cercles ? Tracer ces tangentes.
\end{exo}


\begin{exo}[Construction de cercles : DDP]
On donne deux droites parallèles et un point $A$ entre les deux droites. Dénombrer et tracer les cercles passant par $A$ et tangents aux deux droites. %translations sous-jacentes.

\begin{hint}
Commencer par construire un cercle tangent aux deux droites.
\end{hint}
\begin{sol}
Après avoir suivi l'indication, translater ce cercle de manière à ce qu'il contienne le point $A$.
\end{sol}
\end{exo}



\begin{exo}[Cercles tangents]
% exo7
%  triangle rectangle, cercle
On donne deux cercles $\mathcal C$ et $\mathcal C'$ de rayons distincts, de centres $O$ et $O'$, tangents extérieurement en  un point $A$. On admet qu'il existe trois tangentes communes  à $\mathcal C$ et $\mathcal C'$ : la tangente commune en $A$, qui est directement constructible, et deux autres droites. L'objectif de l'exercice est de tracer ces deux dernières tangentes. 
\begin{enumerate}

\item Considérons donc une droite tangente à $\mathcal C$ en $B$ et à $\mathcal C'$ en $C$, avec $B\neq C$. La tangente commune en $A$ aux deux cercles coupe $(BC)$ en $I$. Montrer que $I$ est le milieu de $[BC]$ et que $ABC$ est rectangle en $A$.
\item Finir l'exercice (c'est-à-dire construire $B$ et $C$) de l'une des deux façons suivantes:
\begin{enumerate}
\item Soit $D$ tel que $ABDC$ soit un rectangle. Quels sont les points d'intersection entre $(DB)$, $(DC)$ et $(OO')$ ? En déduire une construction du point $D$.
\item Montrer que $OIO'$ est rectangle en $I$ et en déduire une construction du point $I$.
\end{enumerate}
\end{enumerate}

\begin{hint}   
\begin{enumerate}
\item Penser au triangle de l'écolier.
\end{enumerate}
\end{hint}      
\begin{sol} 
\begin{enumerate}
\item Par définition, $(IA)$ et $(IB)$ sont tangentes au cercle $\mathcal C$, donc $IA=IB$. On a de même $IA=IC$ et donc $I$ est le milieu de $[BC]$. Le triangle $ABC$ est donc un triangle d'écolier et il est rectangle en $A$.
\end{enumerate}
\end{sol}  
\end{exo}  


\begin{exo}[Distances]
 % exo7
% triangle isocèle,  tangente à un cercle, angles opposés
% C'est un exercice sur les angles en fait
Soit $A$  un point quelconque du diamètre d'un cercle $\mathcal C$ et $B$ l'extrémité d'un rayon perpendiculaire à ce diamètre. On mène une droite $(BA)$ qui coupe le cercle en $P$, puis la tangente au point $P$ qui coupe en $C$ le diamètre prolongé. Démontrer que $CA = CP$.

\begin{hint}   
Utiliser la caractérisation des triangles isocèles à l'aide d'angles.
% Utiliser des angles opposés par le sommet. 
\end{hint}      
\begin{sol}  
%%%%%%%%%%%%
%% FIGURE %%
%%%%%%%%%%%%

Pour montrer $CA=CP$, on va montrer que le triangle $CAP$ est donc isocèle en $C$.

On a les égalités d'angles:
\begin{align*}
\widehat{CAP} &= \widehat{OAB} \text{ car les angles sont opposés par le sommet}\\
&= \pi/2 - \widehat{ABO} \\
&= \pi/2 -\widehat{OPB}  \text{ car $POB$ est isocèle en $O$}\\
&= \widehat{APC}
\end{align*}

Le triangle $CAP$ est donc isocèle en $C$, et donc $CA=CP$.
\end{sol}  
\end{exo}  


\begin{exo}[Cercle tangent à un carré] % exo7
% tangente à un cercle, pythagore, trinôme
On considère un carré $ABCD$, et un cercle $\mathcal C$ passant par $A$ et $B$ et tangent à $[CD]$.
\begin{enumerate}
\item Montrer que le point de tangence est le milieu de $[CD]$.
\item Montrer que si le rayon vaut $r=10$ alors $AB=16$, et réciproquement.
%\item Montrer que $AB = \frac{8}{5} r$.
\end{enumerate}

\begin{hint}
Pythagore.
\end{hint}  
\end{exo}  


%------------------------------
\begin{exo}[Angle inscrit dans le cas limite \faLightbulbO]
% n'utilise que des triangles isocèles et rectangles
Soit $\mathcal C$ un cercle de centre $O$, $[AB]$ une corde et $\mathcal T$ la tangente de $A$. Montrer que l'angle entre $\mathcal T$ et $(AB)$ vaut la moitié de $\widehat{AOB}$.

\begin{center}
\includegraphics{images/img007118-1}
\end{center}

\begin{hint}
Triangles isocèles et rectangles
\end{hint}
\begin{sol} 
Traçons la figure, où on a placé $I$ le milieu de $[AB]$, de telle sorte que $\frac12(OA,OB)=(OA,OI)$.

\begin{center}
\includegraphics{images/img007118-2}
\end{center}


Les angles $(AO,\mathcal T)$ et $(AI,IO)$ sont droits.
On a d'une part :
\[ 0=(\mathcal T,\mathcal T) =  (\mathcal T,AI) +(AI,AO)+ \pi/2, \]
et d'autre part, dans le triangle $AIO$:
\[ 0=(AI,AO)+(IO,IA)+(OA,OI)=(AI,AO)+\pi/2+(OA,OI).\]
Finalement, on a donc:
\[ (\mathcal T,AB) = (\mathcal T, AI) = -(AI,AO)-\pi/2 = (OA,OI)=\frac{1}{2}(OA,OB),\]
ce qu'il fallait démontrer.$\qed$
\end{sol}  
\end{exo}  


\subsection{Bissectrices et cercle inscrit}

Les bissectrices sont toujours au programme de collège mais pas le fait qu'elles soient concourantes, ni le cercle inscrit .


\begin{exo}[Les bissectrices sont concourantes]
Montrer que les bissectrices (intérieures) d'un triangle sont concourantes et que leur intersection est le centre d'un cercle tangent à chaque côté, que l'on appelle cercle inscrit.
\begin{hint}
La bissectrice de l'angle $\widehat A$ est la demi-droite composée des points à égale distance des demi-droites $[AB)$ et $[AC)$.
\end{hint}
\begin{sol}
Considérons deux bissectrices du triangle (celles des angles $\widehat A$ et $\widehat B$ par exemple), ainsi que leur point d'intersection $I$. Alors $I$ est à égale distance des trois côtés, donc il est sur la troisième bissectrice, et c'est aussi le centre d'un cercle qui est tangent aux trois côtés.


\end{sol}
\end{exo}

\begin{exo}[Cercles exinscrits]
Soit $ABC$ un triangle. Montrer que les bissectrices extérieures de $\widehat B$ et $\widehat C$ et la bissectrice intérieure en $A$ sont concourantes, et que leur intersection est le centre l'un cercle tangent à $[BC]$ et aux demi-droites $[AB)$ et $[AC)$. Ce cercle est appelé le cercle exinscrit en $\widehat{A}$.
\end{exo}

\begin{exo}[Application des cercles inscrits et exinscrits]
% Problème DDD
On donne trois droites. Combien y a-t-il de cercles tangents aux trois droites ?
\begin{sol}
En général (si les trois droites sont distinctes), quatre : les droites forment un triangle, et les cercles qui conviennent sont le cercle inscrit et les trois cercles exinscrits.
\end{sol}
\end{exo}

\begin{exo}[Tangentes communes concourantes]
Trois cercles sont tangents extérieurement. Montrer que les trois tangentes communes aux points de contact sont concourantes.
\begin{hint}
Centre du cercle inscrit.
\end{hint}
\end{exo}


\begin{exo}[Distance au centre du cercle inscrit] % exo7
% cercle inscrit, somme des angles d'un triangle
Soit $ABC$ un triangle et $I$ le centre de son cercle inscrit, dont on note $r$ le rayon. Montrer qu'un des sommets du triangle est à distance $\geq 2r$ de $I$, et qu'un autre est à distance $\leq 2r$.

\begin{hint}   
\'Ecrire les distances aux sommets en fonction des angles du triangle.
\end{hint}      
\begin{sol} 
Deuxième indication: un des angles du triangle a une mesure $\geq \pi/3$, et un autre a une mesure $\leq \pi/3$.
\end{sol}  

\end{exo}  



\begin{exo}[Quadrilatères tangentiels]%exo7
%  cercle inscrit, bissectrice, triangles isocèles
Un quadrilatère convexe est dit \emph{tangentiel} ou \emph{circonscriptible} s'il possède un cercle inscrit, c'est-à-dire si ses quatre côtés sont tangents à un même cercle.

\begin{enumerate}
\item Montrer qu'un quadrilatère est tangentiel ssi ses bissectrices intérieures sont concourantes.
\item Montrer le théorème de Pitot (1725) : dans un quadrilatère tangentiel, la somme des longueurs de deux côtés opposés est égale à la somme des deux autres. Réciproque ?
\item Montrer qu'un cerf-volant isocèle (ou rhomboïde) est tangentiel.
\end{enumerate}

\begin{hint}   
Décomposer les longueurs suivant les points de tangence du cercle inscrit.
\end{hint}
     
      
\end{exo}  





\begin{exo}[Théorème des trois tangentes]%exo7
% triangle isocèle, tangentes
% facile, intersection de deux tangentes => équidistance
Soit $ABC$ un triangle. Le cercle exinscrit dans l'angle en $A$ touche les côtés $[BC]$, $[CA]$ et $[AB]$ en $P$, $Q$ et $R$. Montrer que la somme $AR+AQ$ est égale au périmètre du triangle.
\end{exo} 


\begin{exo}[Aire, périmètre et cercle inscrit]%exo7
Soit $ABC$ un triangle dont on note $a$, $b$ et $c$ les longueurs des côtés.
\begin{enumerate}
\item Exprimer l'aire $S$ du triangle en fonction du périmètre $a+b+c=2p$ et du rayon $r$ du cercle inscrit.
\item Exprimer également $S$ en fonction de $a$ et du rayon $r_A$ du cercle exinscrit en $A$.
\item En déduire $\frac{1}{r} = \frac{1}{r_A}+\frac{1}{r_B}+\frac{1}{r_C}$.
\end{enumerate}

\begin{hint}   
Partitionner le triangle en plusieurs triangles pour calculer l'aire.
\end{hint}   
\end{exo}  



\begin{exo}[Cercles inscrits et exinscrits] % exo7
% Euclide, I prop. 17
% 
 On donne un cercle  $\mathcal C$ (de centre $O$), un point $M$ à l'extérieur du cercle, les deux tangentes $\mathcal D$ et $\mathcal D'$ à $\mathcal C$ passant par $M$. On notera $A$ et $B$ les points de tangence. 

Le cercle $\mathcal C$ coupe $(MO)$ en deux points $P$ et $Q$. D'autre part, soit $H$ l'intersection de la corde $[AB]$ avec $(OM)$. Montrer que les cercles de centres $P$ et $Q$ et passant par $H$ sont tangents à $\mathcal D$ et $\mathcal D'$. 


\begin{sol} 
Pour le cercle de centre $P$, il suffit de montrer que $P$ est le centre du cercle inscrit du triangle $MAB$. Pour cela, en notant $C$ le projeté orthogonal de $P$ sur $(MA)$, il suffit de montrer que $PC=PH$, ou de montrer que $AC=AH$. Or, on a $AC=AH = \cos(\widehat{AMO}) / OA$.

D'autres solutions sont possibles, par exemple avec des homothéties.
\end{sol}  
\end{exo}  




%%%%%%%%%%%%%%%%%%%%%%%%%%%%%%%%%%%
\section{Classe de 3ème}
%%%%%%%%%%%%%%%%%%%%%%%%%%%%%%%%%%%

%fractions irréductibles

%Racine carrée ?

%Thalès et réciproque, thm des milieux, angle inscrit, cosinus, sinus. Parallélogrammes, angles alternes/internes, médiatrices, somme des angles, inégalité triangulaire, hauteurs.

%Homothéties. Cas d'égalité des triangles.



%Identités remarquables.

%Equations et inéquations de degré un et calcul littéral.

%Un peu d'arithmétique.





\subsection{Théorème des milieux (ou Thalès)}

Le théorème des milieux et sa réciproque ne semblent plus être au programme, (mais Thalès oui). On a néanmoins séparé les exercices qui ne nécessitent que les milieux et pas Thalès en toute généralité.

\begin{exo}[Trapèze isocèle]%exo7
% tags : symétrie centrale, réflexion orthogonale, théorème des milieux
Soit $ABC$ un triangle,  $P$ le pied de la hauteur issue de $A$ et $I$, $J$, $K$  les milieux des côtés $[AB]$, $[BC]$ et $[CA]$. Montrer que le quadrilatère $IPJK$ est un trapèze isocèle.
\begin{center}
\definecolor{qqwuqq}{rgb}{0.,0.39215686274509803,0.}
\definecolor{uuuuuu}{rgb}{0.26666666666666666,0.26666666666666666,0.26666666666666666}
\definecolor{qqqqff}{rgb}{0.,0.,1.}
\begin{tikzpicture}[line cap=round,line join=round,>=triangle 45,x=1.0cm,y=1.0cm]
\clip(-3.72,-2.28) rectangle (6.96,6.56);
\draw[color=qqwuqq,fill=qqwuqq,fill opacity=0.1] (-0.503984356339779,-1.1145289582773472) -- (-0.4838950747109654,-0.6907407791552316) -- (-0.907683253833081,-0.670651497526418) -- (-0.9277725354618945,-1.0944396766485336) -- cycle; 
\draw (-0.6,5.82)-- (-2.92,-1.);
\draw (-2.92,-1.)-- (5.94,-1.42);
\draw (5.94,-1.42)-- (-0.6,5.82);
\draw (-0.6,5.82)-- (-0.9277725354618945,-1.0944396766485336);
\draw (-1.76,2.41)-- (-0.9277725354618945,-1.0944396766485336);
\draw (1.51,-1.21)-- (2.67,2.2);
\draw (2.67,2.2)-- (-1.76,2.41);
\begin{scriptsize}
\draw [fill=qqqqff] (-0.6,5.82) circle (2.5pt);
\draw[color=qqqqff] (-0.46,6.19) node {$A$};
\draw [fill=qqqqff] (-2.92,-1.) circle (2.5pt);
\draw[color=qqqqff] (-3.22,-0.65) node {$B$};
\draw [fill=qqqqff] (5.94,-1.42) circle (2.5pt);
\draw[color=qqqqff] (6.08,-1.05) node {$C$};
\draw [fill=uuuuuu] (-0.9277725354618945,-1.0944396766485336) circle (1.5pt);
\draw[color=uuuuuu] (-1.08,-1.43) node {$P$};
\draw [fill=uuuuuu] (-1.76,2.41) circle (1.5pt);
\draw[color=uuuuuu] (-1.88,2.73) node {$I$};
\draw [fill=uuuuuu] (2.67,2.2) circle (1.5pt);
\draw[color=uuuuuu] (2.8,2.49) node {$K$};
\draw [fill=uuuuuu] (1.51,-1.21) circle (1.5pt);
\draw[color=uuuuuu] (1.66,-1.45) node {$J$};
\end{scriptsize}
\end{tikzpicture}
\end{center}

\begin{sol}

Par le théorème des milieux, on a $(IK) // (PJ)$ donc $IPJK$ est un trapèze. On va montrer qu'il est isocèle en trouvant un axe de symétrie.

Soit $Q$ le milieu de $[IK]$. Le point $Q$ est aussi le milieu de $[AJ]$ (on le voit en considérant l'homothétie de centre $A$ et de rapport deux, qui envoie $Q$ sur $J$).

En appliquant le théorème des milieux à $APJ$, on voit que la perpendiculaire à $(IK)$ en $Q$ coupe $[PJ]$ (orthogonalement) en son milieu.

C'est donc un axe de symétrie de $IPJK$.
\end{sol}
\end{exo} 

\begin{exo}[Partage en trois](Complexité : 4)

Soit $ABCD$ un parallélogramme, $M$ le milieu de $[AB]$ et $N$ le milieu de $[CD]$. Montrer que les droites $(DM)$ et $(BN)$ coupent la diagonale $[AC]$ en deux points $K$ et $L$ le divisant en trois segments égaux.

\begin{sol}
\marginpar{1} On a $MB=DN$ donc $MBND$ est un parallélogramme.

\marginpar{1} On en déduit que $(DM)//(BN)$.

\marginpar{1} Comme $M$ est le milieu de $[AB]$, on a par le théorème des milieux que $AK=KL$.

\marginpar{1} De la même façon, le théorème de Thalès appliqué dans $DCK$ entraîne que $KL=LC$, d'où le résultat.

\underline{Variante, demande plus d'initiative} :\\
Soit $Q$ le symétrique de $M$ par rapport à $B$. On a $MB=BQ$, donc $BQ=NC$ et $(BQ)//(NC)$. Donc $BQCM$ est un parallélogramme et donc $(NB)//(CQ)$.

Comme $AM=MB=BQ$ et que $(MK)//(BL)//(QC)$, le théorème de Thalès donne $AK=KL=LC$.

\underline{Autre preuve, avec centre de gravité} :\\
Dans le triangle $ABD$, le point $K$ est l'intersection des deux médianes $(DM)$ et $(AO)$. C'est donc le centre de gravité de $ABD$.

On en déduit que $KA = 2KO$.

Par symétrie centrale de centre $O$, on a $AK=CL$ et $KO=LO$, et finalement $AK = KO+OL = KL = LC$.
\end{sol}
\end{exo}


\begin{exo}[Un autre partage en trois, plus difficile]
% assez difficile
% symétrie centrale, homothéthie, Thalès
Soit $ABCD$ un parallélogramme, $K$ le milieu de $[AD]$, $L$ le milieu de $[BC]$.
Les diagonales du parallélogramme $ABLK$ se coupent en G.
Montrer que les droites $(CG)$ et $(DG)$ coupent $[AB]$ deux points $I$ et $J$ qui le partagent en trois parties égales.
\end{exo}


\begin{exo}[Théorème de Varignon] \label{Varignon}
Soit $ABCD$ un quadrilatère convexe, et $I$, $J$, $K$, $L$ les milieux de ses côtés. Montrer que $IJKL$ est un parallélogramme. Montrer que l'aire de $ABCD$ est le double de celle de $IJKL$.
\begin{hint} Méthodologie : de quels théorèmes dispose-t-on ? Lesquels concernent le parallélisme ? Pour l'aire, considérer l'aire du complémentaire de $IJKL$ par exemple, ou bien utiliser les diagonales de $ABCD$.
\end{hint}
% couper le quadrilatère en deux triangles et montrer que les côtés sont parallèles à l'aide de Thalès.
% pour une deuxième preuve de l'aire, voir 
% http://serge.mehl.free.fr/anx/th_varignon.html
\begin{sol}
\begin{enumerate}
\item 
Dans le triangle $ABC$, en notant $I$ est le milieu de $[AB]$ et $J$ le milieu de $[BC]$, le théorème de Thalès dit que $(IJ)$ est parall\`ele \`a $(AC)$ et $IJ = \frac{1}{2} AC$. On raisonne pareillement avec le triangle $ACD$, ce qui donne $(KL)$ parall\`ele \`a $AC$ et $KL = \frac{1}{2} AC$. Or, un quadrilat\`ere qui a deux c\^ot\'es parall\`eles et de m\^eme longueur est un parall\'elogramme.

\item La preuve la plus élémentaire utilise uniquement qu'une médiane d'un triangle donné le partage en deux triangles de même aire.

Soit $O$ le point d'intersection des diagonales du quadrilat\`ere $ABCD$. On consid\`ere le triangle $AOB$. Soit $O_1$ le point d'intersection de la diagonale $[AC]$ avec $[IL]$ et soit $O_2$ le point d'intersection de $[IJ]$ avec la diagonale $[BD]$.

% Par construction, le quadrilat\`ere $IO_1OO_2$ est un parall\'elogramme. 
 
Par le théorème de Thalès,  $O_1$ est le milieu de $[AO]$ et $O_2$ le milieu de $[BO]$. Les triangles $IO_1A$ et $IO_1B$ ont même aire, de même que les triangles $I0_2O$ et $IO_2B$.

La somme des aires des triangles $AIO_1$ et $IO_2B$ est donc exactement \'egale \`a l'aire du parall\'elogramme $IO_1OO_2$. 
 
On applique le m\^eme raisonnement aux triangles $BCO$, $CDO$ et $ADO$, ce qui signifie que, dans le quadrilat\`ere $ABCD$, la partie compl\'ementaire de $IJKL$ a une aire qui est exactement \'egale \`a celle de $IJKL$, ce qui permet de conclure. \end{enumerate}

\end{sol}
\end{exo}

\subsection{Théorème de Thalès}

%Attention : pas de longueur algébrique : quelle est la réciproque au programme ? Peut-on utilisé les triangles opposés par le sommet ?

La section comporte peu d'exercices, beaucoup ayant été placés dans la section sur les homothéties.

\begin{exo}[Le tourniquet dans le triangle]
% source : exogeo.pdf (pdf interactif), chap; geom affine.
% transformations affines, involutions
Par un point $D$ du côté $[AB]$ d'un triangle $ABC$, on trace la parallèle à $[BC]$ qui coupe $[AC]$ en $E$.
Par $E$ on trace la parallèle à $[AB]$ qui coupe $[CB]$ en $F$.
Par $F$ on trace la parallèle à $[AC]$ qui coupe $[BC]$ en $G$.
On construit de même $H$, $I$ et $J$. Montrer que $J=D$.
\begin{hint} Considérer l'application du segment $[AB]$ dans lui-même qui a un point $D$ sur le segment associe $G$ comme construit dans l'énoncé. Que dire si $D$ est une des extrémités du segment ? Que peut-on dire de cette application ? % affine, et involutive car échange deux points; en fait c'est le symétrique sur le segment
\end{hint}

\begin{sol}
Soit $\phi$ l'application du segment $[AB]$ dans lui-même qui a un point $D$ sur le segment associe le point $G$ comme construit dans l'énoncé. On veut montrer qu'appliquer deux fois de suite la fonction $\phi$ à un point revient à ne rien faire. 

Pour comprendre l'application $\phi$, calculons les images de quelques points.
SI $D=A$, on voit on effectuant les trois projections que $\phi(A)=B$. On voit de la même manière que $\phi(B)=A$. L'application $\phi$ échange donc les deux extrémités du segment. D'autre part, on voit en utilisant le théorème des milieux trois fois de suite que l'image du milieu de $[AB]$ par $\phi$ est toujours le milieu de $[AB]$. Ceci porte à croire que l'application $\phi$ est la symétrie du segment par rapport à son milieu, autrement dit que si $D$ est un point de $[AB]$, alors $\phi(D)$ (autrement dit $G$ dans les notations de l'énoncé) est le point qui est à la même distance de $B$ que $D$ de $A$. 

Autrement dit, on veut montrer:
\[ AD=BG\text{ ou bien, de façon équivalente: } BD =AG.\]
On prouve cette égalité en appliquant trois fois le théorème de Thalès (une fois pour chaque projection).
\end{sol}
\end{exo}


\begin{exo}[Subdivision en sept]
[Subdivision]
On donne un segment $[AB]$. Le diviser en sept parts égales.
\end{exo} 




\begin{exo}[Nombres constructibles]
% source : wikipedia, Carrega
On donne deux points $O$ et $I$, avec $OI=1$. Un réel $r$ est constructible si on peut construire à la règle et au compas un point $M$ tel que $\overrightarrow{OM}=r\overrightarrow{OI}$. Le but de l'exercice est de montrer que l'ensemble des nombres constructibles est un sous-corps de $\R$ stable par racine carrée.
\begin{enumerate}
\item  Construire sur la droite $(OI)$ des points $A$, $B$ et  $C$  tels que $OA = \frac{1}{\sqrt{2}}$,  $OB =\sqrt{2}$ et $OC =\sqrt{3}$.
\item (Construction du produit et de l'inverse de deux nombres constructibles.) On donne deux points $A$ et $B$ alignés avec $O$. Construire sur la droite $(AB)$ des points $C$ et $D$ tel que $OC = OA\times OB$ et $OD = \frac{OA}{OB}$. 
\item (Construction de la racine carrée.) Soit $A$ un point sur la demi-droite $[OI)$. Soit $I'$ le symétrique de $I$ par rapport à $O$, soit $\mathcal C$ le cercle de diamètre $I'A$, et soit $F$ l'une des intersections du cercle $\mathcal C$ avec la perpendiculaire à $(OA)$ passant par $O$. Montrer que $OF = \sqrt{OA}$.
\end{enumerate}

\begin{hint}   
\begin{enumerate}
\item Utiliser des triangles particuliers.
\item Utiliser le théorème de Thalès.
\end{enumerate}
\end{hint}
     
      
\end{exo}  






 
%- - - - - - - - - - - -
\subsection{Calcul littéral, identités remarquables}



\begin{exo}[Mise en équation]
% source : Arnold : 77 problems for children 5 to 15
% Masha -> Maria, et Misha -> Mikhaïl
% mise en équation
Il manque sept kopecks à Masha pour s'acheter un livre, mais il n'en manque qu'un à Misha. Même en mettant leur argent en commun, il leur manque encore de l'argent. Combien coûte le livre ?
\begin{sol}
Le livre coûte sept kopecks.
\end{sol}
\end{exo}

\begin{exo}[Mise en équation, bis]
% source : Arnold : 77 problems for children 5 to 15
% mise en équation
Une bouteille avec son bouchon en liège coûte dix kopecks. La bouteille seule est neuf fois plus chère que son bouchon. Combien coûte la bouteille sans son bouchon ?
\end{exo}

\begin{exo}[Mise en équation, ter]
% source : Arnold : 77 problems for children 5 to 15
% mise en équation
Une brique pèse $500$ grammes de plus qu'une moitié de brique. Combien pèse une brique ?
\end{exo}



\begin{exo}[Chiffres  d'un nombre]
[Éliminatoires de la coupe Animath d'automne 2017]
% http://www.animath.fr/IMG/pdf/coupe_animath_automne_17_sujet.pdf
Pour tout entier $n$ strictement positif, on définit le nombre $a_n$ comme étant  le dernier chiffre de la somme des chiffres du nombre
\[ 20052005...2005 \text{(2005 écrit $n$ fois.)}\]
Par exemple $a_1=7$, $a_2=4$ etc.
\begin{enumerate}
\item Quels sont les entiers strictement positifs $n$ tels que $a_n=0$ ?
\item Calculer $a_1+a_2+...+a_{2005}$.
\end{enumerate}
\begin{hint}
Écrire de façon plus simple le nombre dont $a_n$ est le dernier chiffre.
\end{hint}
\begin{sol}
Le nombre dont $a_n$ est le dernier chiffre est simplement $7n$. En considérant la table de multiplication par sept, on voit que les premières valeurs sont $7$, $4$, $1$, $8$, $5$, $2$, $9$, $6$, $3$, $0$, $7$ et à partir de ce moment la suite \og boucle \fg. On en déduit que les $n$ pour lesquels $a_n=0$ sont les multiples de $10$, et que la somme demandée vaut $45\cdot 200+7+4+1+8+5 = 9025$. Pour un corrigé plus détaillé, voir la correction d'Animath : \url{www.animath.fr/IMG/pdf/coupe_animath_automne_17_corrige.pdf}. 
\end{sol}
\end{exo}


\begin{exo}[Deux vieilles dames]
% source : Arnold : 77 problems for children 5 to 15
Deux vieilles dames habitent dans deux villages différents, le village $A$ et le village $B$.
Au lever du soleil, elles partent chacune en direction de l'autre village. Elles se croisent à midi, mais continuent chacune leur chemin, chacune à sa vitesse. La première dame arrive au village $B$ à quatre heures de l'après-midi, l'autre arrive au village $A$ à neuf heures du soir.

À quelle heure sont-elles parties ? À quel endroit se sont-elles croisées?
\begin{sol}
Elles sont parties à six heures, et se sont croisées au $2/5$ème du trajet pour la plus lente, au $3/5$ème pour la plus rapide.

Cet exercice tombe tout de suite si on trace le graphe des positions des deux dames au cours du temps et qu'on applique... le théorème de Thalès. On obtient immédiatement, en notant $t$ le temps écoulé entre le départ et midi :
\[ \frac{t+4}{t+9} = \frac{4}{t} = \frac{t}{9},\]
d'où $t=6$.
\end{sol}
\end{exo}

\begin{exo}[Mise en équation, quater]
% source : Arnold : 77 problems for children 5 to 15
% mise en équation
Vasya a deux sœurs de plus qu'il n'a de frères. Combien de filles de plus que de garçons les parents de Vasya ont-ils ?
\begin{sol} Une de plus. On peut noter $f$ le nombre de filles et $g$ le nombre de garçons. Alors, Vasya a $f$ sœurs et $g-1$ frères, et la première phrase se traduit donc par $f = (g-1)+2$.
\end{sol}
\end{exo}


%\begin{exo}
% source : Arnold : 77 problems for children 5 to 15
%loup, chèvre et chou.
%\end{exo}

%\begin{exo}[Changement de perspective]
% source : Arnold : 77 problems for children 5 to 15
%Deux trains partent
%Mouche entre deux trains
%\end{exo}



\begin{exo}[Fourmi]
% source : Arnold : 77 problems for children 5 to 15
Une fourmi se trouve à un coin d'une pièce cubique et souhaite se rendre au coin opposé, en suivant les murs. Quel est le chemin le plus court (peut-il y avoir plusieurs chemins de longueur minimale) et quelle longueur fait-il ?

\begin{sol}
On déplie le patron du cube et on trace une ligne droite. Si le cube est de côté $1$, on trouve $\sqrt 5$.
\end{sol}
\end{exo}


\begin{exo}[Die Hard 3]
% source : Arnold : 77 problems for children 5 to 15
On dispose de deux récipients, l'un de cinq litres, l'autre de trois litres. Comment faire pour obtenir exactement un litre dans un des deux récipients?
\end{exo}


\begin{exo}[Un système]
% source : Arnold : 77 problems for children 5 to 15
% mise en équation
Dans une famille, il y a cinq têtes et quatorze jambes. Combien de chiens et combien d'humains compte cette famille ?
\end{exo}

\begin{exo}[Sections possibles d'un cube]
% source : Arnold : 77 problems for children 5 to 15
On découpe un cube suivant un plan. Sur le plan de découpe, la trace du cube forme un polygone. Quels sont les polygones qui peuvent apparaître ? Combien de côtés ? Peuvent-ils être réguliers ?
\end{exo}

\begin{exo}[Tétraèdres]
% source : Arnold : 77 problems for children 5 to 15
Dans un cube, il y a deux tétraèdres réguliers inscrits maximaux. Quelle est l'intersection de ces tétraèdres ?
\end{exo}


\begin{exo}[Somme des impairs]
Combien font $1+3$ ? Et $1+3+5$ ? Et enfin $1+3+5+7$ ? Que remarque-t-on ? Peut-on le prouver ?
\begin{hint}
Les sommes semblent toujours donner des carrés.
\end{hint}
\begin{sol}
La somme demandée vaut $n^2$.
Au lycée, on peut prouver le résultat demandé par récurrence, ou bien encore, si on connaît la formule $1+2+...+n = n(n+1)/2$, on peut écrire $1+3+5+...+(2n+1) = 1+2+3+4+...+2(n+1) -(2+4+6+(2(n+1)) = 1+2+...+2(n+1) -2(1+2+...+(n+1))$ et trouver le résultat.

Au collège, on peut expliquer la \og preuve sans mots\fg avec le dessin du carré quadrillé.
\end{sol}
\end{exo}

\begin{exo}[Somme des entiers]
Trouver un moyen de calculer $1+2+3+...+100$ relativement rapidement, sans faire toutes les additions.

\begin{sol}
Méthode de Gau\ss : on écrit deux fois la somme en commençant par la fin la deuxième fois. On obtient $(101\times 100)/2 = 5050$.
\end{sol}
\end{exo}

%%%%%%%%%%%%%%%%%%
\subsection{Homothéties}
 
\begin{exo}[Condition sur quatre points] % exo7
\begin{enumerate}
\item À quelle condition sur quatre points $P_1, ... P_4$ existe-t-il une homothétie $h$ telle que $h(P_1)=P_2$ et $h(P_3)=P_4$ ?
\item Soit $\phi$ une homothétie. On donne deux points $A$ et $B$, ainsi que leurs images $\phi(A)$ et $\phi(B)$. Le centre de l'homothétie n'est pas donné. Le construire, y compris si les quatre points donnés sont alignés.
% dans le deuxième cas, tracer un triangle ABC, et tracer son image par l'homthétie en tracant les parallèles. On trouve un point C', et CC' intersecte la droite (AB) en O.
\end{enumerate}
\end{exo}  


\begin{exo}[Deux cercles sont homothétiques]
%tags : homothéties
% source : http://debart.pagesperso-orange.fr/geoplan/construc_probleme_classique.html#ch13
Soient $\mathcal C$ et $\mathcal C'$ deux cercles de rayons distincts. Montrer qu'il existe des homothéties transformant l'un en l'autre. Suivant la position et la taille des cercles, combien y a-t-il de telles homothéties ? Tracer leurs centres.

\begin{hint}   
Si $\phi$ est une homothétie envoyant $\mathcal C$ sur $\mathcal C'$, alors $O'=\phi(O)$. Il suffit d'avoir un deuxième couple $(M, \phi(M))$ pour pouvoir tracer le centre de l'homothétie $\phi$.
\end{hint} 
\begin{sol}  
Tracer des rayons de $\mathcal C$ et $\mathcal C'$ parallèles entre eux.
\end{sol}   
\end{exo}  




\begin{exo}[Construction d'un carré inscrit dans un triangle \faLightbulbO]
% homothétie, Thalès
Soit $ABC$ un triangle. Construire un carré dont un sommet appartient à $[AB]$, un à $[AC]$ et deux sommets adjacents appartiennent à $[BC]$.

\begin{hint}   
En faisant  une figure avec le carré déjà construit, on voit alors deux segments parallèles, ce qui invite à utiliser  une homothétie.
\end{hint}      
\begin{sol} 

Analyse. Traçons comme suggéré une figure avec le carré déjà construit : on trace un carré puis on trace un triangle adéquat autour. On constate qu'un des côtés du carré, notons-le $[IJ]$, est parallèle à $[BC]$. Il y a une homothétie $h$ de centre $A$ qui envoie $[IJ]$ sur $[BC]$. Alors, l'image du carré $IJKL$ par $h$ est un carré dont un des côtés est $[BC]$. Notons $BCDE$ ce carré et traçons-le. On constate que $h(K)=D$ et $h(L)=E$, c'est-à-dire $K=h^{-1}(D)$ et $L = h^{-1}(E)$. Il ne reste plus qu'à faire la synthèse.


\end{sol}  
\end{exo}  


\begin{exo}[Cercle d'Euler]
% tags : homothéties, cercle circonscrit, cercle des neuf points
% cercle de Feuerbach, cercle de Terquem, cercle médian
Soit $ABC$ un triangle. On note $G$, $\Omega$ et $H$ le centre de gravité, le centre du cercle circonscrit $\mathcal C$ et l'orthocentre. Soit $\mathcal C'$ le cercle passant par les milieux $I_A$, $I_B$ et $I_C$ des côtés de $ABC$.
\begin{enumerate}
\item Montrer que le centre de $\mathcal C'$ appartient à la droite $(G\Omega)$. Calculer son rayon.
\item Montrer que $\mathcal C'$ coupe les segments reliant les sommets à l'orthocentre $H$ en leur milieu.
\end{enumerate}
% il y a beaucoup d'autres choses à montrer, mais c'est plus pratique avec les angles inscrits.
\begin{center}
\includegraphics{images/img007086-1}
\end{center}


\begin{hint}   
\begin{enumerate}
\item Considérer une homothétie de centre $G$.
\item Considérer une homothétie de centre $H$. On rappelle que $\overrightarrow{G\Omega}=-\frac12\overrightarrow{GH}$.
\end{enumerate}
\end{hint}      
\begin{sol} 
\begin{enumerate}
\item Le triangle des milieux est l'image de $ABC$ par l'homothétie de centre $G$ et de rapport $-1/2$. On en déduit que $\mathcal C$ est l'image du cercle circonscrit par cette homothétie, et donc que son centre est l'image de $\Omega$ par cette homothétie : il est donc sur la droite $(G\Omega)$.

\item Montrons que $\mathcal C'$ est l'image de $\mathcal C$ par l'homothétie de centre $H$ et de rapport $1/2$. Considérons la composition de l'homothétie de centre $G$ et de rapport $-1/2$ avec l'homothétie de rapport $-1$ et de centre $J$. C'est une homothétie de rapport $1/2$ qui envoie $\mathcal C$ sur $\mathcal C'$. Comme elle envoie de plus $\Omega$ sur $J$, son centre est le point $M$ tel que $\overrightarrow{MJ}=\frac12\overrightarrow{M\Omega}$. Or on sait déjà, par exemple en considérant l'homothétie de centre $G$ et de rapport $-\frac12$, que $\overrightarrow{G\Omega}=-\frac12\overrightarrow{GH}$. On en déduit que $M=H$.
\end{enumerate}
\end{sol}  
\end{exo} 




 

\begin{exo}[Théorème du trapèze]
% homothéties
Soient $[AB]$ et $[CD]$ deux segments parallèles de longueurs différentes. Montrer qu'il existe des homothéties transformant l'un en l'autre. Combien y a-t-il de telles homothéties ? Tracer leurs centres. Montrer que la droite reliant ces centres coupe les segments en leur moitié.

Application : construire à la règle seule le symétrique de $A$ par rapport à $B$. Expliquer comment construire à la règle seule n'importe quel barycentre à coefficients rationnels de $A$ et de $B$.
\end{exo}  



\begin{exo}[Application du th. du trapèze]
Soit $\mathcal D$ une droite, $A$ et $B$ deux points distincts  n'appartenant pas à cette droite, et $A'$ le symétrique de $A$ par rapport à $\mathcal D$. On suppose que $A'B$ n'est pas parallèle à $\mathcal D$. Construire à la règle seule le symétrique $B'$ de $B$ par rapport à $\mathcal D$.

\begin{hint}   
Considérer le quadrilatère $ABB'A'$ et ses diagonales : elles se croisent sur la droite.
\end{hint} 
     
\end{exo}  

\begin{exo}[Application du th. du trapèze, bis]
Soient deux droites parallèles distinctes $\mathcal D$ et $\mathcal D'$, et $P$ un point non situé sur ces droites. Construire à la règle seule la droite passant par $P$ et parallèle aux deux autres.
\end{exo}  


\begin{exo}[Projections affines]
% source : Leitchnam
% se fait facilement en coisissant des coordonnées affines, en fait.

Soit $ABCD$ un parallélogramme, et $M$ un point sur la diagonale $(BD)$. Soit $I$ le symétrique de $C$ par rapport à $M$. Soit $E$ la projection de $I$ sur $(AB)$ parallèlement à $(AD)$, et $F$ la projection de $I$ sur $(AD)$ parallèlement à $(AB)$. Montrer que $E$, $M$ et $F$ sont alignés.

\begin{hint}    Utiliser une homothétie et une symétrie centrale.
\end{hint}
      
\end{exo}  








\begin{exo}[Pappus affine]
% Source :  Audin par exemple 
% homothéties
Soient $D$ et $D'$ deux droites non parallèles. Soient $A, B, C$ trois points sur $D$, et $A'$, $B'$ et $C'$ trois points sur $D'$. Si $(AB') // (BC')$ et $(BA') // (CB')$, alors $(AA') // (CC')$.
\begin{hint}    Des homothéties de même centre commutent
\end{hint}
\begin{sol}     
Soit $O$ le pt d'intersection. On note $\phi$ l'homothétie qui envoie $A$ sur $B$, et $\psi$ celle qui envoie $B$ sur $C$. Alors $\phi\psi = \psi\phi$. L'image de $A$ est $C$ et l'image de $A'$ est $C'$, d'où le  parallélisme demandé.
\end{sol}
\end{exo}  


\begin{exo}[Desargues affine]
% tags : homothéties.
\begin{enumerate}
\item Soient $ABC$ et $A'B'C'$ deux triangles (non aplatis) sans sommet commun. Montrer qu'ils se déduisent l'un de l'autre par homothétie ou translation ssi leurs côtés sont parallèles.
\item (Application) On donne deux droites se coupant en un point $O$ hors de la feuille, ainsi qu'un point $M$ hors de ces droites. Tracer la droite $(OM)$.
\end{enumerate}
\begin{hint}   
Homothéties et translations
\end{hint}
\end{exo}  








\subsection{Triangles semblables et trigonométrie}

La notion de triangle semblable est au programme et figure dans le Sésamath 2016. La définition est que les angles sont égaux, et il est admis que dans ce cas, les longueurs des côtés sont proportionnelles. La réciproque est énoncée mais sans démonstration non plus. 

(La notion - plus difficile - de similitude elle, est hors-programme du collège mais l'a quasiment toujours été : elle était auparavant vue en terminale S (spécialité maths) mais plus depuis quelques années.)





\begin{exo}[Triangles rectangles semblables]
Soit $ABC$ un triangle rectangle en $A$, et $H$ le pied de la hauteur issue de $A$. Montrer que $ABH$ et $ACH$ sont semblables à $ABC$.
\begin{sol}
Ces triangles ont les mêmes angles, donc sont semblables.
\end{sol}
\end{exo}

\begin{exo}[$HA^2=HB\cdot HC$] % exo7
% puissance, triangles rectangles, semblables
Soit $ABC$ un triangle rectangle en $A$, et $H$ le pied de la hauteur issue de $A$. Montrer $AH^2 = HB\cdot HC$.

\begin{hint}
Utiliser des triangles semblables, ou bien utiliser la puissance d'un point par rapport à un cercle.
\end{hint}
\begin{sol} 
Les triangles $ABC$, $ABH$ et $ACH$ sont semblables car ils ont à chaque fois deux (donc trois) angles identiques. Les rapports de longueurs de côtés homologues sont donc égaux, ce qui donne
\[ \frac{AB}{AC} = \frac{HB}{HA} = \frac{HA}{HC}\]
d'où on tire $HA^2 = HB\cdot HC$, ce qu'il fallait démontrer.\\

Remarque (autre preuve, plus sophistiquée) : Soit $A'$ le symétrique de $A$ par rapport à $[BC]$. Alors, $BACA'$ est inscriptible, et la puissance de $H$ par rapport à son cercle circonscrit est 
\[ p_{\mathcal C}(H) = HB\cdot HC=  HA\cdot HA' = HA^2.\]
\end{sol}
\end{exo}

%\begin{exo}[Preuves du théorème de Pythagore avec triangles semblables]
%\end{exo}

\begin{exo}[Quadrilatère croisé]
Soit $ABCD$ un quadrilatère croisé, tel que $(AB)//(CD)$. Soit $O$ le point d'intersection de $(BC)$ et $(AD)$. Montrer que $ABO$ et $OCD$ sont semblables.
\begin{sol}
Les triangles ont les mêmes angles (alternes-internes).

Ou alors, on applique directement Thalès \og inversé\fg.
\end{sol}
\end{exo}

\begin{exo}[Un quadrilatère]
Soit $ABCD$ un quadrilatère tel que $\widehat{BAC} = \widehat{BDC}$. Montrer que $\widehat{ABD}=\widehat{ACD}$.
\begin{sol}
Les angles $\widehat{AIB}$ et $\widehat{DIC}$ sont opposés par le sommet donc leurs mesures sont égales.

On en déduit que les triangles $AIB$ et $DIC$ sont semblables car ils ont deux paires d'angles égaux. 

On en déduit que leurs troisièmes angles sont égaux.
\end{sol}
\end{exo}

\subsection{Arithmétique}
La division euclidienne est au programme, de même que la décomposition unique en facteurs premiers (admise).

Le pgcd est sorti du programme à la dernière réforme mais on peut le  redéfinir et  redémontrer ses propriétés grâce à la décomposition en facteurs premiers, qui est un résultat beaucoup plus sophistiqué.


\begin{exo}[Muguet]
% ancien Sesamath 2012, CS3, page 9, pgcd
Pour le premier mai, on dispose de $182$ brins de muguet et de $78$ roses.
On veut faire le plus grand nombre de bouquets identiques en utilisant toutes les fleurs. Combien de bouquets identiques pourra-t-on faire et quelle sera la composition des bouquets ?
\end{exo}

\begin{exo}[Poteaux pour enclos]
% Sesamath 2012, pgcd
On souhaite clôturer un terrain rectangulaire de dimensions $78$ mètres sur $102$. Afin de poser un grillage, on doit planter des poteaux régulièrement espacés et pour simplifier le travail, on veut que la distance entre chaque poteau soit un nombre entier de mètres. De plus, il faut un poteau à chaque coin.

On veut utiliser le moins de poteaux possibles. Combien de poteaux suffisent ?
\end{exo}

\begin{exo}[\'Ecriture en base dix]
Si $a$ est un chiffre, démontrer que le nombre $\overline{a00a}$ est divisible par $143$.
\begin{hint}
Que valent ces nombres ?
\end{hint}
\begin{sol}
Ces nombres sont les neuf premiers multiples de $1001$, donc il suffit de vérifier pour celui-ci, puisque $143=11\times 13$.
\end{sol}
\end{exo}

\begin{exo}[Impairs consécutifs]
Montrer que la somme de deux nombres impairs consécutifs est divisible par quatre.
\end{exo}

\begin{exo}[Même reste]
%ppcm, chronomath
Quel est le plus grand nombre de 3 chiffres qui, divisé par 6, 9 et 12 donne toujours le même reste 5 ? 
\begin{sol}
$n-5$ doit être multiple de $36$.
$977$.
\end{sol}
\end{exo}


%%%%%%%%%%%%%%%%%%%%%%%%%%%%%%%%%%%
\section{Abordable en 3ème après compléments}
%%%%%%%%%%%%%%%%%%%%%%%%%%%%%%%%%%%


\subsection{Autour de l'inégalité arithmético-géométrique}

L'inégalité arithmético-géométrique (à deux variables) peut se prouver facilement en utilisant simplement l'inégalité remarquable $(x-y)^2=x^2-2xy+y^2$. On présente le résultat sous forme d'exercice, puis quelques exercices d'application.

\begin{exo}[Inégalité arithmético-géométrique]
Soient $a$ et $b$ deux nombres positifs. Montrer
\[ \frac{a+b}{2} \geq \sqrt{ab}.\]
(\og La moyenne arithmétique est supérieure ou égale à la moyenne géométrique.\fg)
\begin{sol}
Comme les deux quantités sont positives, il suffit de vérifier que les carrés des deux quantités vérifient la même inégalité. 

Ou alors, comme $a$ et $b$ sont positifs, on peut écrire $a=x^2$ et $b=y^2$ et on reconnait une identité remarquable.
\end{sol}
\end{exo}



\begin{exo}[Aire maximale à périmètre fixé] Soit $p$ un nombre réel positif. Déterminer le ou les rectangles de périmètre égal à $p$ dont l'aire est maximale.
\begin{sol}
Soient $a$ et $b$ les mesures des côtés du rectangle. L'aire du rectangle vaut donc: 
\[ A=ab.\]
 
D'autre part, en  calculant le périmètre en fonction de $a$ et $b$ on obtient la contrainte :
\[ 2(a+b)=p.\]

Il s'agit donc de déterminer $a$ et $b$ tels que $a+b=p/2$, de telle façon à maximiser la quantité $ab$. Or, l'inégalité arithmético-géométrique donne
\[ \sqrt ab \leq \frac{a+b}{2} \]
avec égalité si et seulement si $a=b$, autrement dit, en élevant au carré et en écrivant le résultat en fonction de $p$ et de $A$:
\[ A \leq \frac{p^2}{16},\]
avec égalité ssi $a=b$.

Ceci montre que l'aire maximale est atteinte lorsque les deux côtés du rectangle sont égaux, c'est-à-dire lorsque le rectangle est un carré. Dans ce cas, le périmètre vaut $4a=4b$ et l'aire vaut $A=p^2/16 = a^2$.
\end{sol}
\end{exo}

\begin{comment}
\begin{exo}[Une inégalité]
 
%[application directe d'AM>GM]
Soient $a$ et $b$ deux réels.
Montrer que $ab \leq \frac{a^2}{4}+b^2$.
Généraliser.
 

\begin{sol} 
On applique l'inégalité arithmético-géométrique à $\frac{a^2}{4}$ et à $b^2$, ce qui donne
\[
\frac{a^2}{4}+ b^2
\geq
2\sqrt{\frac{a^2}{4}\cdot b^2} = |ab| \geq ab.
\]

On peut généraliser en remplaçant $4$ par un réel strictement positif. Une formulation est la suivante. Pour $a$ et $b$ réels et $\lambda>0$, on a 
\[
2|ab|  \leq \frac{a^2}{\lambda}+ \lambda b^2.
\]
\end{sol}  
\end{exo}
\end{comment}


\begin{exo}[Une inégalité]
 
%[application directe de'AM>GM]
On considère deux réels positifs dont le produit vaut $100$. Leur somme a-t-elle une valeur minimale ou maximale et si oui la(les)quelle(s) et dans quel(s) cas?

\begin{sol} 
Notons $a$ et $b$ les nombres de l'énoncé. On a $ab=100$.

On voit assez vite que la somme de $a$ et $b$ peut être aussi grande que l'on veut, par exemple l'on désire avoir une somme supérieure à un million, il suffit de choisir $a=1000000$, puis  $b=\frac{1}{10000}$. Plus généralement, pour avoir une somme supérieure à un nombre arbitraire $M>0$ il suffit de prendre $a=M$ et $b=100/M$.

Essayons donc de voir si la somme a une valeur minimale.

L'inégalité arithmético-géométrique fournit :
\[ \frac{a+b}{2}\geq \sqrt{ab}=10\]
avec égalité ssi $a=b$, donc la somme est supérieure à $20$, avec égalité ssi $a=b=10$.
\end{sol}  
\end{exo}



\begin{comment}
\begin{exo}[Une inégalité]
%[séparer les termes]
Soient $a, b \in \R^*$. Montrer que $(1+a^2)(1+b^2) \geq 4ab$.
\begin{sol} 
On applique l'inégalité arithmético-géométrique à chacun des deux facteurs ce qui donne:
\[
(1+a^2)(1+b^2) \geq 
(2\sqrt{a^2})(2\sqrt{b^2})
= 4|ab| \geq  4ab.
\]

Remarque : on aurait également pu développer le membre de gauche et minorer par une seule utilisation de l'inégalité arithmético-géométrique à quatre variables:
\[
(1+a^2)(1+b^2) 
=1+a^2+b^2+a^2b^2
\geq  
4\sqrt[4]{a^2b^2a^2b^2} =4\sqrt[4]{a^4b^4}
= 4|ab| \geq 4ab.
\]
\end{sol}  
\end{exo}



\begin{exo}[Une inégalité]
 
%[application directe d'AM>GM après avoir développé et séparé les termes]
% question ouverte
Soient $a$ et $b$ deux réels positifs tels que $a+b=8$. Déterminer la valeur minimale de la quantité 
$\left(1+\frac1a\right)\left(1+\frac1b\right)$ 
et préciser pour quelles valeurs de $a$ et de $b$ elle est atteinte.
 

\begin{sol} 
On peut essayer d'appliquer l'inégalité arithmético-géométrique à chaque facteur. Ceci donne
\[ 
\left(1+\frac1a\right)\left(1+\frac1b\right)
\geq 
\left(\frac{2}{\sqrt a}\right)\left(\frac{2}{\sqrt b}\right)
= \frac{4}{\sqrt{ab}},
\]
ce qui minore la quantité par $1$ après une deuxième utilisation de l'inégalité arithmético-géométrique sur le dénominateur et utilisation de $a+b=8$, mais cette dernière minoration est évidente vu la forme initiale de l'expression : les deux facteurs sont supérieurs à $1$.

(Remarque : lors de la première utilisation de l'inégalité arithmético-géométrique, il y avait égalité ssi $a=1$ et $b=1$ ce qui est impossible vu l'énoncé. L'inégalité est donc toujours stricte, ce qui indique que la minoration n'est sans doute pas très précise.)

Commençons donc plutôt par développer la quantité à minorer. On a 
\[ \left(1+\frac1a\right)\left(1+\frac1b\right)
=
\frac{1+a+b+ab}{ab} = \frac{9+ab}{ab} = 1+\frac{9}{ab}.\]
Il s'agit donc de majorer le produit $ab$. L'inégalité arithmético-géométrique donne $\sqrt{ab} \leq \frac{a+b}{2}=4$, donc $ab\leq 16$. On en déduit que $\frac{1}{ab} \geq \frac{1}{16}$ et donc que 
\[
\left(1+\frac1a\right)\left(1+\frac1b\right) \geq 1+\frac{9}{16} = \frac{25}{16},
\]
avec égalité ssi $a=b=4$.

Remarque : il est préférable d'utiliser les contraintes (ici $a+b=8$) le plus tôt possible dans les majorations ou minorations successives, pour gagner en précision.
\end{sol}  
\end{exo}




\begin{exo}[Inégalité de Cauchy-Schwarz]
% application directe d'AM>GM après avoir développé
Soient $a$, $b$, $c$ et $d$ quatre réels. Montrer que
\[ (ac+bd)^2 \leq (a^2+b^2)(c^2+d^2),\]
et préciser le cas d'égalité.
 

\begin{sol} 
En développant, l'inégalité est équivalente à 
\[
a^2d^2+b^2c^2 \geq 2abcd.
\]
En appliquant l'inégalité arithmético-géométrique à $a^2d^2$ et $b^2c^2$, on obtient:
\[
a^2d^2+b^2c^2 \geq 
2\sqrt{a^2d^2b^2c^2}
=2|abcd|
\geq 2abcd,
\]
la première inégalité étant une égalité ssi 
\[ a^2d^2=b^2c^2,\]
c'est-à-dire ssi $|ad|=|bc|$, et la seconde inégalité étant une égalité ssi $|abcd|=abcd$. Finalement, on a égalité ssi
\[ 
|ad|=|bc| \text{ et }  |abcd|=abcd \\
\Leftrightarrow ad-bc=0,
\]
c'est-à-dire ssi les vecteurs de coordonnées $(a,b)$ et $(c,d)$ sont colinéaires

\underline{Deuxième solution.}\\
En fait, on a l'identité remarquable
\[ (a^2+b^2)(c^2+d^2)=  (ac+bd)^2 + (ad-bc)^2,\]
donc on en déduit l'inégalité avec égalité ssi $(ad-bc)^2=0$, c'est-à-dire ssi $ad-bc=0$. Cette identité remarquable est utile à connaître, elle sert parfois en arithmétique, où elle permet de montrer que si deux nombres sont des sommes de deux carrés, alors leur produit aussi.\\
%preuve sans mots de ce fait ? Mais ce n'est pas homogène... 
% Ensuite, voir thm des deux carrés de Fermat .


\underline{Remarque 1} : majorer chacun des deux facteurs dans le membre de gauche donne juste 
\[ (a^2+b^2)(c^2+d^2) \geq 4|ab|\cdot|cd|,\]
ce qui ne permet pas de conclure puisque par ailleurs on a également 
\[ (ac+bd)^2 \geq 4|acbd|.\]

\underline{Remarque 2} : comme souvent, le cas d'égalité est presque plus important que l'inégalité elle-même.

\end{sol}  
\end{exo}
\end{comment}


\subsection{Exercices utilisant les anciens programmes de 5ème et 4ème}



\begin{exo}[Trois cercles tangents] % exo7


% collège
% triangles isocèles, médiatrices, cercle circonscrit
Trois cercles sont tangents extérieurement deux à deux. Montrer que les tangentes communes sont concourantes.

\begin{hint}   
Utiliser des triangles isocèles.
\end{hint}      
\begin{sol}  
L'exercice se résout assez simplement en utilisant trois triangles isocèles, mais on peut remarquer que les trois tangentes sont les trois axes radicaux, qui s'intersectent tous trois  au centre radical des trois cercles. 
\end{sol}  
\end{exo}  




\begin{exo}[Tangentes communes à deux cercles]%exo7
% tags triangle rectangle, cercle circonscrit
On donne deux cercles $\mathcal C$ et $\mathcal C'$ de rayons $r < r'$, de centres $O$ et $O'$, disjoints et extérieurs l'un à l'autre. On admet qu'il existe quatre tangentes communes  à $\mathcal C$ et $\mathcal C'$. L'objectif est de les construire.
\begin{enumerate}
\item (Analyse) Soit $\mathcal D$ une tangente commune. On note $A$ et $A'$ les points de contact de $\mathcal D$ avec les deux cercles. Que peut-on dire de la parallèle à $(AA')$ passant par $O$ et de son intersection avec $(O'A')$ ? 
\item (Synthèse) En déduire une construction du point d'intersection de ces deux droites, puis ces deux droites et enfin de $\mathcal D$. Tracer les quatre tangentes communes de cette façon.
\end{enumerate}


\begin{hint}   
\begin{enumerate}
\item À quelle distance de $O'$ se situe ce point d'intersection ?
\item Utiliser  le cercle de centre $O'$ et de rayon $r'-r$ pour une tangente \og extérieure\fg{} ou bien $r'+r$ pour une  tangente \og intérieure\fg, et un autre cercle.
\end{enumerate}
\end{hint}      

\begin{sol} 
\begin{enumerate}
\item Les droites sont perpendiculaires.
\item Pour les tangentes communes extérieures, le cercle de centre $O'$ et de rayon $r'-r$ intersecte le cercle de diamètre $[OO']$ en deux points $C$ et $D$. Les droites $(O'C)$ et $(O'D)$ coupent $\mathcal C'$ en deux points $A'$ et $B'$. Les tangentes extérieures sont les parallèles à $(OC)$ et $(OD)$ passant par $A'$ et $B'$. Pour les tangentes intérieures, utiliser le cercle de centre $O'$ et de rayon $r'+r$. 
\end{enumerate}
\end{sol}  
\end{exo}  





\begin{exo}[Construction de cercles, CDP]
\begin{enumerate}
\item On donne une droite $\mathcal D$ et un point $O$ hors de la droite. Tracer le cercle de centre $O$ et tangent à $\mathcal D$.

\item On donne deux droites $\mathcal D$ et $\mathcal D'$ sécantes et un point $O$ d'une bissectrice (et hors des droites). Construire le cercle de centre $O$ et tangent aux deux droites. 
\end{enumerate}


\begin{hint}   
1. Commencer par trouver le point de la droite qui va appartenir au cercle.\\

2. Idem.
\end{hint}  
    
\end{exo}  

\begin{exo}[Construction de cercles DPP]
\begin{enumerate}
\item On donne une droite $\mathcal D$, un point $H$ sur $\mathcal D$ et un point $A$ en-dehors. Tracer le cercle passant par $A$ et tangent à la droite en $H$.% cas particulier de PPP ou PPD



\item On donne trois droites dont deux parallèles. Dénombrer et construire les cercles tangents aux trois droites. 


\end{enumerate}



\begin{hint}   
\begin{enumerate}
\item Test de méthodologie : quelles droites peut-on tracer à partir de ce qui est donné ?
\item Construire la droite équidistante (à distance $r$) des deux parallèles, puis les deux droites parallèles à la troisième et à distance $r$.
\end{enumerate}
\end{hint}      

\end{exo}  



\begin{exo}[Construction de cercles, CCC]
% tags : construction de cercles, cas particulier de CCC
On donne trois cercles distincts $\mathcal C_1$, $\mathcal C_2$ et $\mathcal C_3$ de même rayon et dont les centres ne sont pas alignés. Construire deux cercles tangents à $\mathcal C_1$, $\mathcal C_2$ et $\mathcal C_3$.


\begin{hint} 
Soient $O_1$, $O_2$ et $O_3$ les centres des trois cercles. Considérer le centre le centre du cercle circonscrit à $O_1O_2O_3$. \\
\end{hint}
  
\end{exo}  


\begin{exo}[Construction de cercles de rayon donné]%exo7
% tags : construction de cercles, cas particulier de CCC
Dans tout l'exercice, on fixe $R>0$. Dénombrer et construire les cercles de rayon $R$ tangents à :
\begin{enumerate}
\item deux cercles distincts $\mathcal C_1$ et $\mathcal C_2$;
\item deux droites sécantes $\mathcal D_1$ et $\mathcal D_2$;
\item un cercle $\mathcal C$ et une droite $\mathcal D$.
\end{enumerate}


\begin{sol} Tracer le lieu des points à distance $R$ des cercles et droites en présence. Leurs éventuels points d'intersection fournissent des solutions.
\end{sol}  
\end{exo}  



\begin{exo}[Tangentes communes à plusieurs cercles] % exo7
\begin{prerequis}Droite tangente à un cercle.\end{prerequis}


% homothéties, tangentes
%(\cite[I, prop. 17]{Euclide})
 On donne un cercle  $\mathcal C$ (de centre $O$), un point $M$ à l'extérieur du cercle, les deux tangentes $\mathcal D$ et $\mathcal D'$ à $\mathcal C$ passant par $M$. On notera $A$ et $B$ les points de tangence. 

Le cercle $\mathcal C$ coupe $(MO)$ en deux points $P$ et $Q$. D'autre part, soit $H$ l'intersection de la corde $[AB]$ avec $(OM)$. À l'aide d'homothéties, montrer que les cercles de centres $P$ et $Q$ et passant par $H$ sont tangents à $\mathcal D$ et $\mathcal D'$. 
\begin{center}
\includegraphics{images/img007091-1}
\end{center}
\end{exo}  



\begin{exo}[Tangentes communes]
\begin{prerequis}Droite tangente à un cercle.\end{prerequis}
 % exo7
% homothéties
On donne trois cercles disjoints, de rayons distincts et à l'extérieur les uns des autres. Chacune des trois paires de cercles fournit deux tangentes communes extérieures qui se croisent en un point.  Montrer que ces trois points sont alignés.

\begin{hint}   
Utiliser des homothéties.
\end{hint}

\end{exo} 
 
\begin{exo}[Construction de cercles DDP] % exo7
\begin{prerequis}Droite tangente à un cercle.\end{prerequis}

% tags : translation, homothétie
\begin{enumerate}
\item On donne deux droites parallèles et un point $A$ entre les deux droites. Dénombrer et tracer les cercles passant par $A$ et tangents aux deux droites.
\item On donne deux droites sécantes et un point $A$ n'appartenant pas aux deux droites. Dénombrer et tracer les cercles passant par $A$ et tangents aux deux droites.

\end{enumerate}


\begin{hint}   
Sans la condition sur $A$, l'exercice est facile. Tracer n'importe quel cercle tangent aux droites. Ensuite, appliquer la méthodologie classique. % translation dans un cas, homothétie dans l'autre.
\end{hint}
\begin{sol}
Commencer par tracer la bissectrice, puis $(OA)$. Ensuite, tracer un cercle quelconque tangent aux deux droites (dans le même secteur angulaire), et utiliser une homothétie. 

Note : les exercices faisant intervenir des homothéties se résolvent plus facilement en \og partant de la fin\fg, c'est-à-dire en procédant par analyse-synthèse et en faisant une figure approximative de ce que sera la solution.
\end{sol} 
\end{exo}  




\begin{exo}[distance au centre du cercle inscrit]
\begin{prerequis}
cercle inscrit, somme des angles d'un triangle
\end{prerequis}
Soit $ABC$ un triangle et $I$ le centre de son cercle inscrit, dont on note $r$ le rayon. Montrer qu'un des sommets du triangle est à distance $\geq 2r$ de $I$, et qu'un autre est à distance $\leq 2r$.
\begin{hint}   
\'Ecrire les distances aux sommets en fonction des angles du triangle.
\end{hint}
\begin{sol}
Deuxième indication: un des angles du triangle a une mesure $\geq \pi/3$, et un autre a une mesure $\leq \pi/3$.
\end{sol}
\end{exo}


\begin{exo}[Bissectrices extérieures]
\begin{prerequis}triangle rectangle $\Leftrightarrow$ inscrit ds un demi-cercle.
\end{prerequis}
% Debart
À l'extérieur d'un triangle $ABC$, on construit deux triangles rectangles $ADB$ et $ACE$, rectangles en $D$ et $E$, et tels que $D$ et $E$ soient sur les bissectrices extérieures des angles $\widehat B$ et $\widehat C$ du triangle $ABC$.

Que peut-on dire de $(DE)$ ? De la distance $DE$ ?
\begin{center}
\definecolor{qqwuqq}{rgb}{0.,0.39215686274509803,0.}
\definecolor{uuuuuu}{rgb}{0.26666666666666666,0.26666666666666666,0.26666666666666666}
\definecolor{qqqqff}{rgb}{0.,0.,1.}
\begin{tikzpicture}[line cap=round,line join=round,>=triangle 45,x=1.0cm,y=1.0cm]
\clip(-3.74,-0.7) rectangle (6.28,4.38);
\draw[color=qqwuqq,fill=qqwuqq,fill opacity=0.10000000149011612] (-2.655518409384939,1.7157301140179493) -- (-2.317600870221469,1.972268097025739) -- (-2.574138853229259,2.3101856361892086) -- (-2.9120563923927287,2.053647653181419) -- cycle; 
\draw[color=qqwuqq,fill=qqwuqq,fill opacity=0.10000000149011612] (4.696770762642875,2.2118050837777394) -- (4.5652318867527795,1.8084472722386917) -- (4.968589698291828,1.676908396348596) -- (5.100128574181923,2.080266207887644) -- cycle; 
\draw [shift={(-1.52,0.22)},color=qqwuqq,fill=qqwuqq,fill opacity=0.10000000149011612] (0,0) -- (74.21924669048461:0.6) arc (74.21924669048461:127.2047987085572:0.6) -- cycle;
\draw [shift={(-1.52,0.22)},color=qqwuqq,fill=qqwuqq,fill opacity=0.10000000149011612] (0,0) -- (127.2047987085572:0.6) arc (127.2047987085572:180.19035072662976:0.6) -- cycle;
\draw [shift={(4.5,0.24)},color=qqwuqq,fill=qqwuqq,fill opacity=0.10000000149011612] (0,0) -- (0.19035072662975322:0.6) arc (0.19035072662975322:71.93835265393318:0.6) -- cycle;
\draw [shift={(4.5,0.24)},color=qqwuqq,fill=qqwuqq,fill opacity=0.10000000149011612] (0,0) -- (71.9383526539332:0.6) arc (71.9383526539332:143.68635458123666:0.6) -- cycle;
\draw (-0.48,3.9)-- (-1.52,0.22);
\draw (-1.52,0.22)-- (4.5,0.24);
\draw (4.5,0.24)-- (-0.48,3.9);
\draw [domain=-3.74:6.28] plot(\x,{(--1.3548--0.02*\x)/6.02});
\draw [domain=-3.74:6.28] plot(\x,{(-1.0776220212444918-0.7964792780812006*\x)/0.6046658247224244});
\draw [domain=-3.74:6.28] plot(\x,{(-4.2038460318521595--0.9507234792794206*\x)/0.31004010377180713});
\draw [line width=1.6pt,dash pattern=on 2pt off 2pt,domain=-3.74:6.28] plot(\x,{(--16.53171958585--0.026618554706224984*\x)/8.012184966574651});
\draw (-2.9120563923927287,2.053647653181419)-- (-0.48,3.9);
\draw (-0.48,3.9)-- (5.100128574181923,2.080266207887644);
\draw [shift={(-1.52,0.22)},color=qqwuqq] (74.21924669048461:0.6) arc (74.21924669048461:127.2047987085572:0.6);
\draw[color=qqwuqq] (-1.6203713112074283,0.7505898603314065) -- (-1.6426760470313013,0.8684987181828305);
\draw [shift={(-1.52,0.22)},color=qqwuqq] (127.2047987085572:0.6) arc (127.2047987085572:180.19035072662976:0.6);
\draw[color=qqwuqq] (-2.0040925443636612,0.4592789344917715) -- (-2.1116686653333634,0.5124520310454985);
\draw [shift={(4.5,0.24)},color=qqwuqq] (0.19035072662975322:0.6) arc (0.19035072662975322:71.93835265393318:0.6);
\draw[color=qqwuqq] (4.893390370544686,0.6099243387001123) -- (4.980810452887949,0.6921297473001377);
\draw[color=qqwuqq] (4.972798427862551,0.5008862714109371) -- (5.077864745165339,0.558860998391145);
\draw [shift={(4.5,0.24)},color=qqwuqq] (71.9383526539332:0.6) arc (71.9383526539332:143.68635458123666:0.6);
\draw[color=qqwuqq] (4.271895873331519,0.7294573601416263) -- (4.22120606740519,0.8382256623953205);
\draw[color=qqwuqq] (4.400318399077286,0.7707198681390068) -- (4.378166932205572,0.8886576166143417);
\begin{scriptsize}
\draw [fill=qqqqff] (-0.48,3.9) circle (2.5pt);
\draw[color=qqqqff] (-1.16,4.07) node {$A$};
\draw [fill=qqqqff] (-1.52,0.22) circle (2.5pt);
\draw[color=qqqqff] (-1.68,-0.17) node {$B$};
\draw [fill=qqqqff] (4.5,0.24) circle (2.5pt);
\draw[color=qqqqff] (4.62,-0.07) node {$C$};
\draw [fill=uuuuuu] (-2.9120563923927287,2.053647653181419) circle (1.5pt);
\draw[color=uuuuuu] (-3.,2.51) node {$D$};
\draw [fill=uuuuuu] (5.100128574181923,2.080266207887644) circle (1.5pt);
\draw[color=uuuuuu] (4.88,2.69) node {$E$};
\end{scriptsize}
\end{tikzpicture}
\end{center}

\begin{hint}
Elle est parallèle à $(BC)$, et elle coupe $[AB]$ et $[AC]$ en leur milieu. La distance $DE$ vaut la moitié du périmètre de $ABC$.
\end{hint}
\begin{sol}
Soit $A'$ le point d'intersection de $(AD)$ avec $(BC)$. Alors $(BD)$ est à la fois une bissectrice et une hauteur de $BAA'$, donc $BAA'$ est isocèle en $B$.

On en déduit par le théorème des milieux (ou Thalès) que la droite passant par $D$ et le milieu de $[AB]$ est parallèle à $(BC)$.

Le même raisonnement pour le triangle $ACE$ montre que $(DE) // (BC)$, et que $(DE)$ passe par les milieux de $[AB]$ et $[AC]$.
\end{sol}
\end{exo}








\section{Sorti du programme de 3ème en 2016}


\subsection{Angles inscrits, angles au centre}

Ce petit bijou est sorti du programme de troisième à la dernière réforme, mais de toute façon il était sous-exploité. Pour des exercices de base sur le théorème des angles inscrits, voir le manuel Sesamath pré-2016. On a regroupé ici des exercices un peu plus stimulants.


\begin{exo}[Construction de l'arc capable]
% application directe
On donne un segment $[AB]$ et un réel $\alpha \in ]-\pi,\pi[$. On suppose que l'on dispose également d'un triangle auxiliaire $XYZ$ avec $\widehat{(\overrightarrow{XY}, \overrightarrow{XZ})}=\alpha$, de sorte que les angles de mesure $\alpha$ sont constructibles.

 Construire le lieu des points $M$ tels que $\widehat{(\overrightarrow{MA}, \overrightarrow{MB})}=\alpha$.

\begin{sol} 
Par le théorème de l'angle inscrit, c'est un arc de cercle, dont le centre est sur la médiatrice de $[AB]$. 

Par le cas limite du théorème de l'angle inscrit, on sait aussi que si $\mathcal T$ est la tangente à ce cercle en $A$, alors $(\mathcal T,AB)=\alpha$.

On trace donc la droite $\mathcal T$ faisant un angle $\alpha$ avec $(AB)$ en $A$, puis la perpendiculaire à $\mathcal T$ passant par $A$. Cette droite coupe la médiatrice en un point $O$ qui est donc le centre du cercle recherché.
\end{sol}  
\end{exo}  

\begin{exo}[Octogone sur un segment]
% angle au centre, inscrit. application directe
 Construire un octogone convexe régulier dont un des côtés est un segment $[AB]$ donné.

\begin{hint}   
Il y a deux tels octogones. En notant $O$ le centre d'un tel octogone, on doit avoir $\widehat{AOB}=\pm \pi/4$.
\end{hint}      
\begin{sol}  
Construisons un triangle $AIB$ isocèle rectangle en $I$ et le cercle de centre $I$ et de rayon $IA$. Ce cercle intersecte la médiatrice de $[AB]$ en un point $O$ qui vérifie $\widehat{AOB}=\pm \pi/4$, par le théorème de l'angle au centre. C'est donc le centre d'un octogone appuyé sur $[AB]$. En traçant le cercle de centre $O$ et de rayon $OA$, on peut terminer la construction de cet octogone.
\end{sol}  
\end{exo}  

\begin{exo}[Trapèzes inscriptibles]
% angles inscrits, facile, application directe
Montrer qu'un trapèze est isocèle si et seulement s'il est inscriptible.


\begin{sol}
Commençons par rappeler deux points:
\begin{enumerate}
\item dans un trapèze, deux angles non adjacents à une même base sont supplémentaires, puisque les deux bases sont parallèles.
\item un quadrilatère non croisé est inscriptible ssi les angles opposés sont supplémentaires.
\end{enumerate}

Un trapèze est isocèle ssi les angles adjacents à une même base sont égaux, donc (par le premier point ci-dessus) ssi les angles opposés sont supplémentaires, donc (par le deuxième point) ssi il est inscriptible.

\begin{center}
\includegraphics{images/img007121-1}
\end{center}

\end{sol}
\end{exo}  


\begin{exo}[antiparallélogramme]
% angles inscrits
Un antiparallélogramme est un quadrilatère croisé dont les cotés opposés sont deux à deux de même longueur. Soit $ABCD$ un antiparallélogramme. Montrer les assertion suivantes.

\begin{center}
\includegraphics{images/img007122-1}
\end{center}

\begin{enumerate}
\item Les angles opposés ont la même mesure.
\item Les diagonales $(AC)$ et $(BD)$ sont parallèles.
\item La médiatrice des diagonales est un axe de symétrie de $ABCD$.
\item Deux côtés opposés ont leur point d'intersection situé sur cette médiatrice.
\item Le quadrilatère convexe $ADBC$ formé par les deux côtés non croisés et les diagonales est un trapèze isocèle.
\item $ABCD$ est inscriptible.
\end{enumerate}


\end{exo}  



%------------------
\begin{exo}[Théorème de Reim]
% application directe
Soient $\mathcal C_1$ et $\mathcal C_2$ deux cercles sécants en $A$ et $B$, et $\mathcal D_A$ (respectivement $\mathcal D_B$) une droite passant par $A$  (resp. $B$). On note $C$ et $E$ (resp. $D$ et $F$) l'intersection de $\mathcal D_A$ (resp. $\mathcal D_B$) avec 
 les deux cercles. Montrer que $(CD) // (EF)$.
 

\begin{sol} 
Traçons une figure. \emph{On marque dès à présent quelques égalités d'angles obtenues par le théorème de l'angle inscrit:}

\begin{center}
\includegraphics{images/img007123-1}
\end{center}

\emph{Les égalités d'angles repérées sur la figure permettent de voir la solution, au moins dans la configuration particulière dessinée. On voit en effet que les angles $\widehat{ECD}$ et $\widehat{AEF}$ sont égaux. Attention toutefois, les angles géométriques sont trompeurs et les égalités que l'on voit sur une figure peuvent dépendre de la façon de tracer la figure. Sur la figure ci-dessous par exemple, les angles en question ne sont pas égaux mais supplémentaires.}

\begin{center}
\includegraphics{images/img007123-2}
\end{center}


\emph{Il ne reste plus qu'à rédiger rigoureusement la solution  avec des angles de droites, en s'appuyant sur l'intuition donnée par la figure.}

Pour montrer que $(CD)$ et $(EF)$ sont parallèles, il suffit par exemple de montrer qu'elles forment le même angle avec la droite $(CA)$. Or on a la suite d'égalités d'angles de droites :

\begin{align*}
(CD,CA) &= (BD,BA) \text{ car $CDAB$ est inscriptible}\\
&= (BF,BA) \text{ car $(BD)=(BF)$}\\
&=(EF,EA) \text{ car $BFAE$ est inscriptible}\\
&=(EF,CA)  \text{ car $(EA)=(CA)$.}
\end{align*}

\end{sol}   
\end{exo}  



\begin{exo}[Bissectrices et cercle circonscrit]%exo7
% exercice simple sur les angles inscrits
Les bissectrices intérieure et extérieure en $A$ d'un triangle $ABC$ non isocèle en $A$ recoupent le cercle $\Gamma$ circonscrit à ce triangle respectivement en $I$ et $J$. Montrer que $I$ et $J$ appartiennent à la médiatrice de $[BC]$.

\begin{hint}   
Rédiger avec des angles de droites et ne pas faire de distinction entre bissectrice extérieure et intérieure.
\end{hint}      

\begin{sol} 

Pour montrer le résultat, il suffit de montrer que $IBC$ et $JBC$ sont isocèles en $I$ et $J$. 

On commence par prouver le résultat pour $I$ :

\begin{center}
\includegraphics{images/img007124-1}
\end{center}

Pour montrer que $BCI$ est isocèle en $I$, il suffit de montrer que $(BC,BI) = (CI,CB)$. Or, on a 
\begin{align*}
(BC,BI) &= (AC,AI) \text{ car $ABIC$ est inscriptible}\\
&= (AI,AB) \text{ car $(AI)$ est une bissectrice de $(AC)$ et $(AB)$}\\
&= (CI,CB) \text{ car $ABIC$ est inscriptible.}
\end{align*}

On remarque qu'en rédigeant avec des angles de droites, on n'a pas eu besoin (ni en fait la possibilité) de préciser si la bissectrice était intérieure ou extérieure, ce qui implique que la preuve sera la même pour $J$. Traçons juste une figure pour visualiser la deuxième situation.


\begin{center}
\includegraphics{images/img007124-2}
\end{center}



\end{sol}  
\end{exo}  


\begin{exo}[Cas limite du théorème de Reim] %exo7
% application directe
% angle inscrit et angle au centre, cas limite
% ou homothéties
Soient $\mathcal C$ et $\mathcal C'$ deux cercles tangents en un point $T$, et $\mathcal D_1$, $\mathcal D_2$ deux droites sécantes en $T$. On note $A$ et $A'$ (resp. $B$ et $B'$) les points d'intersection de $\mathcal D_1$ (resp. $\mathcal D_2$) avec $\mathcal C$ et $\mathcal C'$. Montrer que les droites $(AB)$ et $(A'B')$ sont parallèles.


\begin{center}
\includegraphics{images/img007125-1}
\end{center}


\begin{hint}
Introduire la tangente commune $\mathcal T$ aux deux cercles. 
%Utiliser le cas limite du théorème de l'angle au centre.
\end{hint}
\begin{sol}
Soit $\mathcal T$ la tangente commune  aux deux cercles.

\begin{center}
\includegraphics{images/img007125-2}
\end{center}

Par le cas limite du théorème des angles inscrits, on a 
\[ (AB,AT) = (BT,\mathcal T)=(B'T,\mathcal T)=(A'B',A'T)\]

Comme $(AT) = (A'T)$, on en déduit que
\[ (AB,AT) = (A'B',AT),\]
et donc que $(AB)//(A'B')$. 

\underline{Autre preuve:} considérer une homothétie de centre $T$ qui envoie un cercle sur l'autre.

\end{sol}
\end{exo}



\begin{exo}[Théorème des trois cercles de Miquel] %exo7
% Mettre l'autre sens
% points cocycliques
Soit $ABC$ un triangle direct, et $P$, $Q$ $R$ trois points situés sur $[BC]$, $[CA]$ et $[AB]$ respectivement. Montrer que les cercles circonscrits à $ARQ$, $BPR$ et $CQP$ sont concourants.

\begin{center}
\includegraphics{images/img007126-1}
\end{center}


\begin{hint}   
Soient $\mathcal C$ et $\mathcal C'$ les cercles circonscrits à $ARQ$ et $BPR$. 
Ils se coupent en $R$ et en un deuxième point $T$. Montrer que $T$ est sur le cercle circonscrit à $CQP$.
\end{hint}      

\begin{sol} 



Soient $\mathcal C$ et $\mathcal C'$ les cercles circonscrits à $ARQ$ et $BPR$. 
Ils se coupent en $R$ et en un deuxième point $T$.

\begin{center}
\includegraphics{images/img007126-1}
\end{center}

Il s'agit de montrer que $T, P, C, Q$ sont cocycliques.


Par le cours,  il suffit de montrer l'égalité d'angles de droites $(QT,QC)=(PT,PC)$. Or on a :
\begin{align*}
(QT,QC)&=(QT,QA) \text{ car $(QC)=(QA)$}\\
&=(RT,RA) \text{ car $AQTR$ est inscriptible} \\
&= (RT,RB) \text{ car $(RA)=(RB)$} \\
&=(PT,PB) \text{ car $PTRB$ est inscriptible} \\
&=(PT,PC) \text{ car $(PB)=(PC)$.}
\end{align*}

Attention, si on utilise des angles géométriques au lieu des angles de droites pour rédiger la solution, on peut être amené à distinguer plusieurs configurations possibles, par exemple celle-ci:
\begin{center}
\includegraphics{images/img007126-2}
\end{center}

(Les angles $\widehat{BRT}$ et $\widehat{BPT}$ sont supplémentaires dans la première figure, et égaux dans la seconde.)

\end{sol}  
\end{exo}  



\begin{exo}[Triangle orthique $\heartsuit$]%exo7
% joli, points cocycliques
Soit $ABC$ un triangle non rectangle et $A'$, $B'$, $C'$ les pieds des hauteurs. Montrer que les hauteurs de $ABC$ sont des bissectrices du triangle $A'B'C'$, dit \emph{triangle orthique}. 

\begin{hint}   
Utiliser les angles droits pour montrer que des points sont cocycliques, puis utiliser le théorème de l'angle inscrit.
\end{hint}      
\begin{sol}

Le quadrilatère $ABA'B'$ est inscriptible dans un cercle de diamètre $[AB]$. En effet, les triangles $ABA'$ et $ABB'$ sont par définition rectangles en $A'$ et $B'$, et ont même hypoténuse $[AB]$.

De même, les quadrilatères $BCB'C'$ et $CAC'A'$ sont inscriptibles dans des cercles de diamètre $[BC]$ et $[CA]$.

\begin{center}
\includegraphics{images/img007131-1}
\end{center}

Montrons que la hauteur $(BB')$ est une bissectrice des droites $(B'C')$ et $(B'A')$. Pour cela, on montre que $(B'C',B'B)=(B'B,B'A')$.

On a :
\begin{align*}
(B'C',B'B)
&= (CC',CB) \text{ (car $BCB'C'$ est inscriptible)}\\
&= (CC',CA') \text{ (mêmes droites)}\\
&= (AC'AA') \text{ (car $ACA'C'$ est inscriptible)}\\
&= (AB,AA') \text{ (mêmes droites)}\\
&= (B'B,B'A') \text{ (car $ABA'B'$ est inscriptible)}\\
\end{align*}

\end{sol}  
\end{exo}  



\begin{exo}[Pentagramme] %exo7
% tags : chasse aux angles, un peu plus difficile

Soit $\mathcal C$ un cercle, $[BC]$ une corde, et $A \in \mathcal C$ tels que les arcs $AB$ et $AC$ soient égaux. Soient $[AD]$ et $[AE]$ deux autres cordes d'extrémités $A$, qui coupent $[BC]$ en $F$ et en $G$, respectivement. Montrer que $DEFG$ est inscriptible.

\begin{sol} 
Traçons une figure :

\begin{center}
\includegraphics{images/img007128-1}
\end{center}


\emph{[Sur la figure, on voit que les angles $\widehat{GFD}$ et $\widehat{GED}$ sont supplémentaires, car $\widehat{GFD}=\widehat{BFA}$ et $\widehat{GED}=\widehat{GEB}+\widehat{BED} = \widehat{FBA}+\widehat{BAF}$. Il ne reste plus qu'à rédiger cette preuve un peu plus rigoureusement avec des angles de droites.]}

Montrons que $(FD,FG)=(ED,EG)$, ce qui prouve que $EDFG$ est inscriptible.

Tout d'abord, comme $(FD)=(FA)$ et $(FG)=(FB)$, on a 
\[(FD,FG)=(FA,FB).\]

Ensuite, la somme des angles du triangle $ABF$ vaut $\pi$, donc en termes d'angles de droites on a la relation 
$(FA,FB)+(AB,AF)+(BF,BA)=0$, c'est-à-dire:
\[
(FA,FB) = (AF,AB)+(BA,BF).
\]
Calculons chacun de ces deux angles. D'une part, on a :
\begin{align*}
(AF,AB) &= (AD,AB) \text{ car $(AD)=(AF)$}\\
&=(ED,EB) \text{ car $ABDE$ est inscriptible}.
\end{align*}
Et d'autre part :
\begin{align*}
(BA,BF) &= (BA,BC) \text{ car $(BF)=(BC)$} \\
&= (CB,CA) \text{ car $ABC$ est isocèle en $A$}\\
&= (EB,EA) \text{ car $ABCE$ est inscriptible.}
\end{align*}

Finalement, on obtient donc:
\begin{align*}
(FD,FG)&=(FA,FB) \\
&= (AF,AB)+(BA,BF) \\
&= (ED,EB) + (EB,EA)\\
&= (ED,EA)\\
&= (ED,EG) \text{ car $(EG)=(EA)$,}
\end{align*}
ce qu'il fallait démontrer.
\begin{center}
\includegraphics{images/img007128-1}
\includegraphics{images/img007128-2}
\end{center}

\end{sol}  

\end{exo}  



\begin{exo}[Cercles sécants] % exo7
% initiative : tracer (PQ)


Soient $\mathcal C_1$ et $\mathcal C_2$ deux cercles se coupant en $P$ et $Q$, et considérons une droite $\mathcal D$ coupant $\mathcal C_1$ en $A$ et $B$, et coupant $\mathcal C_2$ en  $C$ et $D$. Montrer que $(PA,PC)=(DQ,BQ)$.


Plus précisément, montrer que les angles $\widehat{APC}$ et $\widehat{DQB}$ sont égaux si $\mathcal D$ coupe le segment $[PQ]$ et que $A$, $C$, $B$ et $D$ sont alignés dans cet ordre. Que peut-on dire dans les autres cas ?


\begin{sol}  Traçons une figure. \emph{[Le fait de marquer toutes les égalités d'angles disponibles donne le résultat. Sur la figure, on ne marque que celles utilisées dans la rédaction proposée.]}
\begin{center}
\includegraphics{images/img007129-1}
\end{center}

Montrons que $(PA,PC)=(DQ,BQ)$. On a:
\begin{align*}
(PA,PC)&= (PA,PQ)+(PQ,PC) \\
&= (BA,BQ)+(DQ,DC) \text{ par cocyclicité dans chaque cercle}\\
&= (BA,BQ)+(DQ,BA) \text{ car $(DC)=(BA)$}\\
&= (DQ,BQ).
\end{align*}

\end{sol}  
\end{exo}  








\begin{exo}[Une application de Ptolémée]%exo7
% source : Deschamps chap 2 ex 8
Soit $ABC$ un triangle équilatéral et $M$ un point du cercle circonscrit appartenant à l'arc $BC$ ne contenant pas $A$. Montrer que $MA= MB+MC$.

\begin{hint}Sans le théorème de Ptolémée, on peut aussi considérer le point $N \in [AM]$ tel que $\widehat{BNM}=\pi/3$.
\end{hint}
\end{exo}  

\begin{exo}[Trisection]%exo7
% cocyclicité
\begin{enumerate}
\item Soit $\Gamma$ un cercle, $O$ son centre et $M$ un point n'appartenant pas à $\Gamma$. Deux sécantes issues de $M$ coupent $\Gamma$ respectivement en $A$ et $B$, et en $C$ et $D$. Démontrer l'égalité:
\[2 \widehat{AMC} = \widehat{BOD} -  \widehat{COA}.\]
\item Soient $A$ et $B$ deux points d'un cercle $\Gamma$ de centre $O$ et de rayon $r$. Sur la droite $(OA)$, soit $C$ le point extérieur au cercle tel que la droite $(CB)$ recoupe le cercle en un point $M$ vérifiant $MC=r$. (On ne demande pas de construire ce point, le placer approximativement sur la figure.) Montrer que $\widehat{ACB} = \frac13 \widehat{AOB}$.
\end{enumerate}

\begin{hint}   
Décomposer $(\overrightarrow{MA},\overrightarrow{MC})$ en $(\overrightarrow{MA},\overrightarrow{AD}) + (\overrightarrow{AD},\overrightarrow{MC})$.% puis angles inscrits / angles au centre.
\end{hint}  
    
\end{exo}  





\begin{exo}[Problème \og DPP\fg ]%exo7

Soit $\mathcal D$ une droite et $A$ et $B$ deux points situés d'un seul côté de $\mathcal D$. L'objectif est de construire un cercle passant par les deux points et tangent à la droite. On suppose que les droites $(AB)$ et $\mathcal D$ ne sont pas parallèles. (Si elles le sont, le problème est plus facile, voir l'exercice \ref{DPPparallele}.)

\begin{enumerate}

\item (Analyse) Soit $\mathcal C$ un tel cercle et $T$ son point de tangence avec $\mathcal D$. Montrer que $(AB,AT) = (TB,\mathcal D)$. % angle inscrit avec le cas limite.
\item (Synthèse) Soit $I$ le point d'intersection de $(AB)$ avec $\mathcal D$, $B'$ le symétrique de $B$ par rapport à $I$, et $B''$ le symétrique de $B$ par rapport à $\mathcal D$. Montrer que le cercle circonscrit à $AB'B''$ (de diamètre $[AB']$ dans le cas où $B'=B''$) coupe $\mathcal D$ en deux points qui conviennent pour le choix de $T$.
\end{enumerate}


 

\end{exo}  


\begin{exo}[Bissectrices d'un quadrilatère convexe]%exo7
Soit $ABCD$ un quadrilatère convexe. On note $\mathcal B_A$ (resp. $\mathcal B_B$, $\mathcal B_C$, $\mathcal B_D$) la bissectrice intérieure en $A$ (resp. en $B$, $C$, $D$). Soient $I=\mathcal B_A\cap \mathcal B_B$, $J=\mathcal B_B\cap \mathcal B_C$, $K=\mathcal B_C\cap \mathcal B_D$ et $L=\mathcal B_D\cap \mathcal B_A$. Montrer que $IJKL$ est inscriptible.

\begin{center}
\includegraphics{images/img007136-1}
\end{center}


\begin{hint}
La somme des angles d'un quadrilatère convexe vaut $2\pi$.
\end{hint}
\begin{sol}
La somme des angles d'un quadrilatère convexe vaut $2\pi$ :
\begin{align*}
2\pi &=
\widehat{ABC}+\widehat{BCD}+\widehat{CDA}+\widehat{DAB}\\
&= 2\widehat{ABI}
+2\widehat{KCD}+2\widehat{CDK}+2\widehat{IAB}
\end{align*}
d'où \[\widehat{ABI}
+\widehat{KCD}+\widehat{CDK}+\widehat{IAB}=\pi,\]
autrement dit la somme des demi-angles vaut $\pi$.

On termine alors la preuve en utilisant le critère de cocyclicité.

\end{sol}
\end{exo}  


\begin{exo}[Médiatrices d'un quadrilatère convexe]%exo7
Soit $ABCD$ un quadrilatère convexe. Que dire du quadrilatère formé par les intersections des médiatrices des quatre côtés ? En particulier, que dire si $ABCD$ est un parallélogramme ? Un losange ?


\begin{sol} 
On trouve un parallélogramme dont les angles sont les supplémentaires de ceux de $ABCD$.
\end{sol}  
\end{exo}  


\begin{exo}[Carré invisible, bis]
On considère un carré $ABCD$ et on place quatre points $E$, $F$, $G$, et $H$ sur les côtés de ce carré (en-dehors des sommets). Puis, on efface le carré. L'objectif est de reconstruire le carré en utilisant le théorème de l'angle inscrit.

Si $A$ est le sommet entre $E$ et $F$, montrer que la diagonale du carré partant de $A$ passe par l'intersection du cercle de diamètre $[EF]$ avec la médiatrice de $[EF]$. % angle inscrit
En déduire une construction des diagonales du carré, puis du carré.


\end{exo}  



\begin{exo}[Symétrique de l'orthocentre]%exo7

Soit $ABC$ un triangle, $H$ son orthocentre et $\mathcal C$ son cercle circonscrit. La hauteur issue de $B$ recoupe $\mathcal C$ en $H'$. Montrer que $H'$ est le symétrique de $H$ par rapport à la droite $(AC)$. 

\begin{center}
\includegraphics{images/img007139-1}
\end{center}



\begin{hint}
Utiliser les différentes caractérisations des triangles isocèles.
\end{hint}

\begin{sol}

Si $ABC$ est rectangle, l'orthocentre coïncide avec un des sommets et la vérification de l'assertion est relativement facile. Dans la suite on suppose qu'on n'est pas dans ce cas.

Par définition, $H'$ est le symétrique de $H$ par rapport à $(AC)$ si $(AC)$ est la médiatrice de $[HH']$. C'est cela qu'on doit montrer.



D'autre part, par définition, on a $(AC) \bot (HH')$, donc $(AC)$ est la hauteur de $AHH'$ issue de $A$. 

Donc si $AHH'$ est isocèle en $A$, alors cette hauteur de $AHH'$  est aussi la médiane issue de $A$ et c'est encore la médiatrice du côté opposé à $A$ c'est-à-dire $[HH']$.

Il suffit donc de montrer que $AHH'$ est isocèle en $A$. Pour cela, il suffit de montrer que les angles adjacents à la base sont égaux, autrement dit $\widehat{AHH'} = \widehat{AH'H}$ avec des angles géométriques non orientés, ou plus précisément avec des angles orientés $(H'A,H'H)=(HH',AH)$.

Suivant la méthodologie habituelle, on marque de façon systématique les angles égaux (ou complémentaires, supplémentaires etc) sur la figure. Ceci indique la marche à suivre pour la preuve.


\begin{center}
\includegraphics{images/img007139-2}
\end{center}

Montrons que $(H'A,H'H)=(HH',AH)$. On a :
\begin{align*}
(H'A,H'H) &= (H'A,H'B) \text{ (car $(H'H)=(HB)$}\\
&= (CA,CB) \text{ (car $ABCH'$ est inscriptible)}\\
&= (CA,AH)+(AH,CB) \text{ (par Chasles)}\\
&= (CA,AH)+\pi/2 \text{ (car $(AH)$ est une hauteur de $ABC$)}\\
&= (CA,AH)+(HH',CA)\\
&= (HH',AH) \text{ (par Chasles)}
\end{align*}


\end{sol}
\end{exo}

\begin{exo}[Un théorème de Brahmagupta]%exo7
Soit $ABCD$ un quadrilatère convexe inscriptible dont les diagonales sont perpendiculaires, et soit $O$ leur point d'intersection. Soit $H$ le projeté orthogonal de $O$ sur $[CD]$, et $I$ l'intersection de $(OH)$ avec $[AB]$. L'objectif est de montrer que $I$ est le milieu de $[AB]$. 


\begin{center}
\includegraphics{images/img007140-1}
\end{center}

\begin{enumerate}
\item Montrer qu'il est suffisant d'établir que $IO=IA$.
\item Conclure.
\end{enumerate}


\begin{hint}
Où se trouve le centre du triangle circonscrit d'un triangle rectangle ?
\end{hint}
\begin{sol}

Rappelons la figure :

\begin{center}
\includegraphics{images/img007140-1}
\end{center}

\begin{enumerate}
\item On rappelle que le milieu de l'hypoténuse d'un triangle rectangle est le centre de son cercle circonscrit (une autre façon de le dire est que l'hypoténuse est un diamètre du cercle circonscrit).

Si $IO=IA$, cela signifie que $I$ est sur la médiatrice de $[OA]$. D'autre part, $AOB$ est rectangle en $O$ et $I$ est par définition sur l'hypoténuse $[AB]$. Donc $I$ est  l'intersection de l'hypoténuse et d'une médiatrice d'un autre côté, c'est donc le milieu de l'hypoténuse par la propriété rappelée plus haut. Il est donc suffisant de montrer que $IO=IA$.


\item Pour montrer que $IO=IA$, il suffit de montrer que $IOA$ est isocèle en $I$, c'est-à-dire que $(AI,AO)=(OA,OI)$. Or, on a :
\begin{align*}
(AI,AO)& = (AB,AC) \text{ (mêmes droites)}\\
&= (DB,DC) \text{ (car $ABCD$ est inscriptible}\\
&= (DO,DH) \text{ (mêmes droites)}\\
&= (DO,OH)+(OH,DH) \text{ (par Chasles)}\\
&= (DO,OH)+\pi/2 \text{ (par définition de $H$)}\\
&= (OB,OI)+\pi/2 \text{ (mêmes droites)}\\
&= (OB,OI) + (OA,OB) \text{ (car $(OA)\bot (OB)$ d'après l'énoncé)}\\
&= (OA,OI) \text{ (par Chasles)}
\end{align*}
\end{enumerate}

\end{sol}
\end{exo}

\begin{exo}[Puissance par rapport à un cercle]
Soit $\mathcal C$ un cercle, $\mathcal D$ et $\mathcal D'$ deux droites sécantes en $P$ et coupant chacune le cercle en deux points, $A$ et $B$ et $A'$ et $B'$.

Montrer que $PA\cdot PB = PA'\cdot PB'$.

\begin{hint}
Utiliser des triangles semblables.
\end{hint}
\begin{sol}
Il suffit de montrer que $\frac{PA}{PB'} = \frac{PA'}{PB}$. C'est le cas car $PAB'$ et $PBA'$ sont semblables.
\end{sol}
\end{exo}




\begin{exo}[Réflexion] % exo7

% fan de géom 159
% cocyclicité, bissectrice, puissance
Soit $ABC$ un triangle non isocèle en $A$. % acutangle? Pas besoin
La bissectrice intérieure $\Delta$ de $\widehat{BAC}$ coupe $[BC]$ en $A_1$ et le cercle circonscrit à $ABC$ en $A_2$.
\begin{enumerate}
\item Soit $D$ le symétrique de $B$ par rapport à $\Delta$. Justifier que $D \in (AC)$.
\item Montrer que $A_1A_2CD$ est inscriptible.
\item Montrer que $AA_1\cdot AA_2 = AB\cdot AC$. % puissance
\end{enumerate}
\begin{center}
\includegraphics{images/img007168-1}
\end{center}


\begin{sol}
\begin{enumerate}
\item Par construction, la droite $\Delta$ est la hauteur et la médiane de $ABD$. Le triangle est donc isocèle en $A$ et $\Delta$ est également sa bissectrice, ce qui montre que $\widehat{BAA_1} = \widehat{A_1AD}$ et donc que $D\in (AC)$.
\item Il suffit de prouver que $(A_2A_1,A_2C)=(DA_1,DC)$.
\begin{center}
\includegraphics{images/img007168-2}
\end{center}
Or on a :
\begin{align*}
(A_2A_1,A_2C)
&= (A_2A,A_2C)\\
&= (BA,BC) \text{ (car $ABCA_2$ est inscriptible)}\\
&= (BA,BA_1) \text{ (mêmes droites)}\\
&= (DA,DA_1) \text{ (par réflexion suivant $\Delta$)}\\
&= (DC,DA_1) \text{ (mêmes droites)},
\end{align*}
ce qu'il fallait démontrer.
\item Remarquons déjà que $AB=AD$.  D'autre part, il suffit de montrer que
\[ \frac{AA_1}{AD} = \frac{AC}{AA_2}.\]

La question précédente montre que les triangles $ADA_1$ et $AA_2C$ sont (inversement) semblables, puisqu'ils ont deux (et donc trois) angles en commun. Les rapports entre les côtés sont donc égaux, c'est-à-dire précisément
\[ \frac{AA_1}{AD} = \frac{AC}{AA_2}.\]

Remarque : si on connaît la notion de puissance d'un point par rapport à un cercle, on peut conclure plus vite : les deux produits égaux sont la puissance de $A$ par rapport au cercle $A_1A_2CD$.
\end{enumerate}
\end{sol}
\end{exo}



\fin % macros de fin : affiche corrections, indications, ou rien, selon

\end{document}

