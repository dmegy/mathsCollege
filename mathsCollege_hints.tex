\begin{Hint}{12}
Utiliser une ou plusieurs cordes.
\end{Hint}
\begin{Hint}{18}
Utiliser un triangle isocèle.
\end{Hint}
\begin{Hint}{19}
Décomposer l'aire comme la somme des aires de deux triangles.
\end{Hint}
\begin{Hint}{25}
 %Sans utiliser le "théorème des milieux", on peut compléter le trapèze rectangle en un rectangle grâce à la symétrie de centre $I$.
Les diagonales d'un rectangle sont égales et se coupent en leur milieu. % Y a-t-il plus simple ?
\end{Hint}
\begin{Hint}{27}
Raisonner d'abord par conditions nécessaires en faisant une figure approximative. Si de tels points existent, alors on peut dire un certain nombre de choses sur ce losange.
\end{Hint}
\begin{Hint}{29}
Prolonger le segment $[AO]$.
\end{Hint}
\begin{Hint}{32}
Considérer les symétries axiales d'axe $\mathcal D$ et $\mathcal D'$.
\end{Hint}
\begin{Hint}{36}
   Tracer le milieu $I$ de $AB$ puis procéder par analyse-synthèse.
\end{Hint}
\begin{Hint}{38}
Le triangle rectangle $ADA’$ est inscrit dans un demi-cercle.
% donc il y a un autre angle droit
\end{Hint}
\begin{Hint}{39}
Pour le premier cas, utiliser des angles complémentaires.
\end{Hint}
\begin{Hint}{40}
\end{Hint}
\begin{Hint}{41}
\begin{enumerate}
\item Penser à un trapèze.
\item On peut obtenir une telle droite comme hauteur d'un triangle $ABC$ adéquat.
\end{enumerate}
\end{Hint}
\begin{Hint}{42}
Montrer que $(AB) \bot (RC)$.
\end{Hint}
\begin{Hint}{43}
Déterminer les angles $\widehat{AMB}$ et $\widehat{ANB}$.
\end{Hint}
\begin{Hint}{45}
Reformulation de l'énoncé : le triangle $BMN$ doit être isocèle.
\end{Hint}
\begin{Hint}{47}
Considérer deux médianes ainsi que leur point d'intersection $G$, puis considérer le symétrique de $G$ par rapport au pied des deux médianes.
\end{Hint}
\begin{Hint}{49}
Il n'y a pas besoin du théorème de Thalès  : les médianes d'un triangle se croisent en un point situé aux deux-tiers des sommets.
\end{Hint}
\begin{Hint}{51}
Où se trouve le point $M$ sur le segment $[AC]$ ?
\end{Hint}
\begin{Hint}{54}
Considérer la translation $\tau$ de distance $a$ suivant la direction de la droite.
\end{Hint}
\begin{Hint}{55}
Considérer la translation de vecteur $\overrightarrow{CB}$ et l'image de $I$ par cette translation.
\end{Hint}
\begin{Hint}{56}
Des translations  commutent
\end{Hint}
\begin{Hint}{57}
Faire tourner le carré circonscrit par rapport au carré inscrit.
\end{Hint}
\begin{Hint}{58}
L'aire de l'intersection vaut le quart de l'aire du carré $ABCD$.
\end{Hint}
\begin{Hint}{59}
Considérer la rotation de centre $O$ et d'angle $\pi/2$.
\end{Hint}
\begin{Hint}{60}
Considérer la rotation de centre $O$ et d'angle $\pi/2$.
\end{Hint}
\begin{Hint}{61}
Procéder par analyse-synthèse et considérer des rotations.
\end{Hint}
\begin{Hint}{62}
Utiliser des rotations.
\end{Hint}
\begin{Hint}{64}
Considérer la rotation de centre $O$ et d'angle $\pi/3$.
\end{Hint}
\begin{Hint}{65}
Considérer  une rotation de centre $A$.
\end{Hint}
\begin{Hint}{66}
Si $ABC$ est un tel triangle, considérer les rotations d'angles $\pm \pi/3$ et centrées sur les sommets. Déterminer les images des différents points et droites par ces rotations.
\end{Hint}
\begin{Hint}{69}
Il suffit de montrer que cette bissectrice passe par le centre du carré.
\end{Hint}
\begin{Hint}{70}
Il suffit de construire une des droites d'appui du carré. Pour cela, il suffit de construire un deuxième point sur cette droite.
\end{Hint}
\begin{Hint}{71}
Ce sont les deux triangles rectangles en $C$.
\end{Hint}
\begin{Hint}{73}
 Utiliser le théorème de Pythagore.
\end{Hint}
\begin{Hint}{74}
Utiliser le théorème de Pythagore.
\end{Hint}
\begin{Hint}{75}
 Méthodologie : essayer avec plusieurs points $M$. Que remarque-t-on ?


% autre solution, voir vieux TD, rotations pour se ramener à un seul côté ?
\end{Hint}
\begin{Hint}{76}
Considérer $I$ et $J$ les milieux de $MA$ et $AM'$ ainsi qu'une projection orthogonale. Une telle projection réduit les distances.
\end{Hint}
\begin{Hint}{81}
Commencer par construire un cercle tangent aux deux droites.
\end{Hint}
\begin{Hint}{82}
\begin{enumerate}
\item Penser au triangle de l'écolier.
\end{enumerate}
\end{Hint}
\begin{Hint}{83}
Utiliser la caractérisation des triangles isocèles à l'aide d'angles.
% Utiliser des angles opposés par le sommet.
\end{Hint}
\begin{Hint}{84}
Pythagore.
\end{Hint}
\begin{Hint}{85}
Triangles isocèles et rectangles
\end{Hint}
\begin{Hint}{86}
La bissectrice de l'angle $\widehat A$ est la demi-droite composée des points à égale distance des demi-droites $[AB)$ et $[AC)$.
\end{Hint}
\begin{Hint}{89}
Centre du cercle inscrit.
\end{Hint}
\begin{Hint}{90}
\'Ecrire les distances aux sommets en fonction des angles du triangle.
\end{Hint}
\begin{Hint}{91}
Décomposer les longueurs suivant les points de tangence du cercle inscrit.
\end{Hint}
\begin{Hint}{93}
Partitionner le triangle en plusieurs triangles pour calculer l'aire.
\end{Hint}
\begin{Hint}{98}
 Méthodologie : de quels théorèmes dispose-t-on ? Lesquels concernent le parallélisme ? Pour l'aire, considérer l'aire du complémentaire de $IJKL$ par exemple, ou bien utiliser les diagonales de $ABCD$.
\end{Hint}
\begin{Hint}{99}
 Considérer l'application du segment $[AB]$ dans lui-même qui a un point $D$ sur le segment associe $G$ comme construit dans l'énoncé. Que dire si $D$ est une des extrémités du segment ? Que peut-on dire de cette application ? % affine, et involutive car échange deux points; en fait c'est le symétrique sur le segment
\end{Hint}
\begin{Hint}{101}
\begin{enumerate}
\item Utiliser des triangles particuliers.
\item Utiliser le théorème de Thalès.
\end{enumerate}
\end{Hint}
\begin{Hint}{105}
Écrire de façon plus simple le nombre dont $a_n$ est le dernier chiffre.
\end{Hint}
\begin{Hint}{113}
Les sommes semblent toujours donner des carrés.
\end{Hint}
\begin{Hint}{116}
Si $\phi$ est une homothétie envoyant $\mathcal C$ sur $\mathcal C'$, alors $O'=\phi(O)$. Il suffit d'avoir un deuxième couple $(M, \phi(M))$ pour pouvoir tracer le centre de l'homothétie $\phi$.
\end{Hint}
\begin{Hint}{117}
En faisant  une figure avec le carré déjà construit, on voit alors deux segments parallèles, ce qui invite à utiliser  une homothétie.
\end{Hint}
\begin{Hint}{118}
\begin{enumerate}
\item Considérer une homothétie de centre $G$.
\item Considérer une homothétie de centre $H$. On rappelle que $\overrightarrow{G\Omega}=-\frac12\overrightarrow{GH}$.
\end{enumerate}
\end{Hint}
\begin{Hint}{120}
Considérer le quadrilatère $ABB'A'$ et ses diagonales : elles se croisent sur la droite.
\end{Hint}
\begin{Hint}{122}
    Utiliser une homothétie et une symétrie centrale.
\end{Hint}
\begin{Hint}{123}
    Des homothéties de même centre commutent
\end{Hint}
\begin{Hint}{124}
Homothéties et translations
\end{Hint}
\begin{Hint}{126}
Utiliser des triangles semblables, ou bien utiliser la puissance d'un point par rapport à un cercle.
\end{Hint}
\begin{Hint}{131}
Que valent ces nombres ?
\end{Hint}
\begin{Hint}{137}
Utiliser des triangles isocèles.
\end{Hint}
\begin{Hint}{138}
\begin{enumerate}
\item À quelle distance de $O'$ se situe ce point d'intersection ?
\item Utiliser  le cercle de centre $O'$ et de rayon $r'-r$ pour une tangente \og extérieure\fg{} ou bien $r'+r$ pour une  tangente \og intérieure\fg, et un autre cercle.
\end{enumerate}
\end{Hint}
\begin{Hint}{139}
1. Commencer par trouver le point de la droite qui va appartenir au cercle.\\

2. Idem.
\end{Hint}
\begin{Hint}{140}
\begin{enumerate}
\item Test de méthodologie : quelles droites peut-on tracer à partir de ce qui est donné ?
\item Construire la droite équidistante (à distance $r$) des deux parallèles, puis les deux droites parallèles à la troisième et à distance $r$.
\end{enumerate}
\end{Hint}
\begin{Hint}{141}
Soient $O_1$, $O_2$ et $O_3$ les centres des trois cercles. Considérer le centre le centre du cercle circonscrit à $O_1O_2O_3$. \\
\end{Hint}
\begin{Hint}{144}
Utiliser des homothéties.
\end{Hint}
\begin{Hint}{145}
Sans la condition sur $A$, l'exercice est facile. Tracer n'importe quel cercle tangent aux droites. Ensuite, appliquer la méthodologie classique. % translation dans un cas, homothétie dans l'autre.
\end{Hint}
\begin{Hint}{146}
\'Ecrire les distances aux sommets en fonction des angles du triangle.
\end{Hint}
\begin{Hint}{147}
Elle est parallèle à $(BC)$, et elle coupe $[AB]$ et $[AC]$ en leur milieu. La distance $DE$ vaut la moitié du périmètre de $ABC$.
\end{Hint}
\begin{Hint}{149}
Il y a deux tels octogones. En notant $O$ le centre d'un tel octogone, on doit avoir $\widehat{AOB}=\pm \pi/4$.
\end{Hint}
\begin{Hint}{153}
Rédiger avec des angles de droites et ne pas faire de distinction entre bissectrice extérieure et intérieure.
\end{Hint}
\begin{Hint}{154}
Introduire la tangente commune $\mathcal T$ aux deux cercles.
%Utiliser le cas limite du théorème de l'angle au centre.
\end{Hint}
\begin{Hint}{155}
Soient $\mathcal C$ et $\mathcal C'$ les cercles circonscrits à $ARQ$ et $BPR$.
Ils se coupent en $R$ et en un deuxième point $T$. Montrer que $T$ est sur le cercle circonscrit à $CQP$.
\end{Hint}
\begin{Hint}{156}
Utiliser les angles droits pour montrer que des points sont cocycliques, puis utiliser le théorème de l'angle inscrit.
\end{Hint}
\begin{Hint}{159}
Sans le théorème de Ptolémée, on peut aussi considérer le point $N \in [AM]$ tel que $\widehat{BNM}=\pi/3$.
\end{Hint}
\begin{Hint}{160}
Décomposer $(\overrightarrow{MA},\overrightarrow{MC})$ en $(\overrightarrow{MA},\overrightarrow{AD}) + (\overrightarrow{AD},\overrightarrow{MC})$.% puis angles inscrits / angles au centre.
\end{Hint}
\begin{Hint}{162}
La somme des angles d'un quadrilatère convexe vaut $2\pi$.
\end{Hint}
\begin{Hint}{165}
Utiliser les différentes caractérisations des triangles isocèles.
\end{Hint}
\begin{Hint}{166}
Où se trouve le centre du triangle circonscrit d'un triangle rectangle ?
\end{Hint}
\begin{Hint}{167}
Utiliser des triangles semblables.
\end{Hint}
